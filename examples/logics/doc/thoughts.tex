\documentclass{article}
\usepackage[show]{ed}
\usepackage{url,draftstamp,letterwide}
\def\constlist{\enumerate}
\def\endconstlist{\endenumerate}
\def\openmath{{\sc OpenMath}}
\def\omdoc{{\sc OMDoc}}
\def\mathml{{\sc MathML}}
\title{An Overview and Todo-List for the Logic CDs}
\author{Michael Kohlhase\\ 
Computer Science, Carnegie Mellon University\\
\url{http://www.cs.cmul.edu/~kohlhase}}
\begin{document}
\maketitle

\section{Introduction}
In this note we give an overview over the logic CDs, a set of {\omdoc} documents
that supply the vocabularies for a variety of logics, used in computational logic
and for formalizing mathematics.  Currently, the logic CDs can be found at
{\url{http://omdoc.org/examples/logics}}, they will move
to a more conspicuous resting place when they are released. 

The main use of these vocabularies is in the {\openmath} and content {\mathml}
languages as symbols. In {\openmath} the symbol for falsity would be notated as 
{\tt{<OMS cd="truthval" name="false"/>}}, and in content {\mathml} as
{\tt{<csymbol DefinitionURL="\url{http://omdoc.org/examples/logics/truthval.omdoc#true}"}}.

\begin{todolist}{}
\item add inference rules everywhere. 
\item do we want $n$-ary connectives and quantifiers here?
\item add dynamic logics lik DRT, DPL,... for the computational linguists.
\item we should probably add non-classical logics like modal logics here?
\end{todolist}

 \section{Propositional Logics}
 In this section we will define some theories for and analyze some common
 propositional logics.

\subsection{{\tt{truthval}}: Truth Values}
This theory supplies the type of truth values and the object symbols for truth and
falsity. 

\begin{constlist}
  \item[bool] the type of truth values
  \item[true] the constant for truth
  \item[false] the constant for truth
  \item[eq] the equality of truth values
  \item[neq] the inequality of truth values
  \item[true-intro] the ND inference rule for truth introduction
  \item[true-intro] the ND inference rule for truth introduction
  \item[false-elim] the ND inference rule for false elimination.
\end{constlist}
  
\subsection{{\tt{pl0}}: Classical Propositional Connectives}
This theory provides all classical two-valued propositional connectives, it is
largely identical to the {\openmath} content dictionary {\tt{logic1}}. We have a
separate theory here, since we want to make the semantics more restrictive than
the inclusive treatment in {\tt{logic1}}. In particular, we insist that the
connectives are two-valued.

{\bf{Inclusions}}: {\tt{truthval}} for truth values
\begin{constlist}
  \item[not] classical negation
  \item[and] classical conjunction
  \item[or] classical disjunction
  \item[implies] classical implication
  \item[equivalent] biconditional
  \item[xor] exclusive or
  \item[forall] universal quantification over boolean values
  \item[exists] existential quantification over boolean values
\end{constlist}

\begin{todolist}{}
  \item include the truth tables
  \item find all other binary connectives and give them names. 
\end{todolist}


\subsection{{\tt{skl0}}: Strong Kleene Logic (Propositional)}
This theory provides the three-valued propositional connectives for strong Kleene
logic, which is often used as a basis for partial-function logics. 

{\bf Inclusions}: {\tt{truthval}} for the truth values.
\begin{constlist}
  \item[undefined] The undefined truth value.
  \item[not] strong Kleene negation
  \item[and] strong Kleene conjunction
  \item[or] strong Kleene disjunction
  \item[implies] strong Kleene implication
  \item[equivalent] strong Kleene biconditional
  \item[xor] strong Kleene exclusive or
  \item[tf] the definedness connective; it is true, if it's argument is true or
    false, it is false on undefined.
\end{constlist}

\begin{todolist}{Ask Manfred for help}
  \item include the truth tables.
  \item find other binary connectives and give them names, are all of the
    propositional ones meaningful?. 
  \item do all of the copied examples and lemmata make sense?
  \item Maybe read the Articles by Jones.
\end{todolist}

\section{First-Order Logics}
In this section we define constants for first-order logics, i.e. logics that
include quantifiers that range over individuals.

\subsection{{\tt{ind}}}: Individuals
This theory provides a type for individuals and equality on it. 

\begin{constlist}
  \item[ind] the type of individuals
  \item[eq] the equality on individuals
  \item[neq] inequality on individuals
\end{constlist}

\subsection{{\tt{pl1}}: Classical First-Order Logic}
This theory provides all classical two-valued first-order quantifiers, it is
largely identical to the {\openmath} content dictionary {\tt{quant1}}. We have a
separate theory here, since we want to make the semantics more restrictive than
the inclusive treatment in {\tt{quant1}}. In particular, we insist that the
connectives are two-valued.

{\bf Inclusions}: {\tt{pl0}} for the truth values and connectives. {\tt{ind}} for individuals.
\begin{constlist}
  \item[forall] classical universal quantification
  \item[exists] classical existential quantification
  \item[unique] uniqueness, 
  \item[exists-unique] unique existence
  \item[atmost] this quantifier takes a natural number $n$ as an argument, and
    postulates that there are at most $n$ distinct elements with a certain property.
  \item[atleast] this quantifier takes a natural number $n$ as an argument, and
    postulates that there are at least $n$ distinct elements with a certain property.
  \item[exactly] this quantifier takes a natural number $n$ as an argument, and
    postulates that there are exactly $n$ distinct elements with a certain property.
\end{constlist}
Note that we do not include generalized quantifiers like most, finitely many,
\ldots here, since they are not fist-order definable. 
\begin{todolist}{}
  \item copy and extend {\tt{quant1}}, include the truth tables.
  \item are there any other common first-order quantifiers?
\end{todolist}

\subsection{{\tt{ude1}}: Undefined Elements in First-Order Logic}
This theory introduces constants for  undefined individuals and a two-valued
definedness predicate.

{\bf Inclusions}: none
\begin{constlist}
  \item[undefined] The undefined individual
  \item[defined] the two-valued definedness predicate, it is true, when the
    argument is defined, and false else.
  \item[strongeq] strong equality
  \item[weakeq] weak equality
\end{constlist}
These new constants are enough to express a two-valued treatment of partial
functions, when added to first-order logic\ednote{cite Bill Farmer, and the other
  crowd here.}
\begin{todolist}{}
  \item look the equality notions up with Bill Farmer
  \item maybe call one of them (the more commonly used one) only {\tt{eq}}
\end{todolist}

\subsection{{\tt{skl1}}: Strong First-Order Kleene Logic}

Strong first-order Kleene logic, this is often used as a partial-function logic.

{\bf Inclusions}: {\tt{skl0}} for the truth values and connectives, {\tt{ude1}}
for undefined individuals.
\begin{constlist}
  \item[forall] strong Kleene universal quantification
  \item[exists] strong Kleene existential quantification
  \item[unique] uniqueness, 
  \item[exists-unique] unique existence
  \item[atmost] this quantifier takes a natural number $n$ as an argument, and
    postulates that there are at most $n$ distinct elements with a certain property.
  \item[atleast] this quantifier takes a natural number $n$ as an argument, and
    postulates that there are at least $n$ distinct elements with a certain property.
  \item[exactly] this quantifier takes a natural number $n$ as an argument, and
    postulates that there are exactly $n$ distinct elements with a certain property.
\end{constlist}

\begin{todolist}{Ask Manfred for Help}
  \item do all of these quantifiers make sense in the three-valued setting? Can we
    use the same definitions?
  \item what about equality here, do we need three-valued ones? Maybe the
    three-valued one should be called {\tt{eq}}.
\end{todolist}

\subsection{{\tt{spl1}}: Sorted First-Order Logic}

{\bf Inclusions}: {\tt{simpletypes}} for type constructors, {\tt{pl0}} for connectives.
\begin{constlist}
\item[forall] sorted universal quantifier. This $n$-ary quantifiers takes $n$
  sorts as argument\ednote{continue}
\item[exists] sorted existential quantifier. This $n$-ary quantifiers takes $n$
  sorts as argument\ednote{continue}
\end{constlist}
\begin{todolist}{}
  \item all the other quantifiers as well, what are their sorts.
  \item Can we do Bill Farmer's S2PPL1 with these?
  \item Can we do Kerber/Kohlhase's S3PPL1 with these?\ednote{don't think so.}
\end{todolist}


\section{Types}
In this section, we will define the basic infrastructure for type systems. These
include the sort systems known from first-order logic, and the type systems in
type theories.

\subsection{{\tt{simpletypes}}: Simple Types}
Simple types are characterized by the lack of dependence between the arguments of
the type constructors.

{\bf Inclusions}: none
\begin{constlist}
  \item[ind] base type of individuals, this is often used as the top sort in
    first-order logics
  \item[empty] the empty type, this is often called the ``unit'' type. It is
    sometimes used as the least sort in first-order logics.
  \item[funtype] function type constructor
  \item[pfuntype] partial function type constructor.
  \item[pair] pair type constructor
  \item[dunion] disjoint union type constructor.
  \item[union] union type constructor
  \item[intersection] intersection type constructor
  \item[complement] complement type constructor
  \item[tuple] the tuple type costructor
  \item[record] the record type constructor
\end{constlist}

\begin{todolist}{}
  \item are there more type constructors?
  \item do we really want all these type constructors in simpletypes, or would it
    be better to separate them out into a theory of simple type constructors.
  \item Should we have relations between types in this theory too? {\tt{subtype}},
  {\tt{supertype}}, {\tt{disjoint}}, \ldots?
\end{todolist}

\subsection{{\tt{deptypes}}: Dependent Types}
Dependent types allow dependencies between the arguments of the type constructors.

{\bf Inclusions}: {\tt{simpletypes}} for the base types. 
\begin{constlist}
  \item[funtype] function type constructor
  \item[pfuntype] partial function type constructor.
  \item[tuple] the tuple type costructor
  \item[record] the record type constructor
\end{constlist}

\begin{todolist}{}
  \item are there dependent versions of the other type constructors from
  \item[record] the record type constructor{\tt{simpletypes}}, does anybody use them?
\end{todolist}

\section{Type Theories}
Type theories are logics that use types as structurally governing principles. They
usually introduce term constructors for all type constructors and elaborate
equality theories that take into account the meaning of the objects in each type.\ednote{continue}

\section{Conclusion}
We have given an overview over the Logic CDs, a set of {\omdoc} documents
supplying standardized vocabularies for logics. 

\begin{appendix}
  \section{Translations of the OMEGA symbols}
  \begin{tabular}{|c|c||c|c||l|}\hline
    generic & o & bool & truthval & \\\hline
    generic & i & ind & ind & \\\hline
    generic & list & ???? & ???? & \\\hline\hline
    base & false & false & truthval & \\\hline
    base & true & true & truthval & \\\hline
    base & and & and & pl0 & \\\hline
    base & or  & or & pl0  & \\\hline
    base & implies & implies & pl0 & \\\hline
    base & equiv & equivalent & pl0 & \\\hline 
    base & not & not & pl0  & \\\hline
    base & = & eq & truthval & for type $o$ in untyped context\\\hline
    base & = & eq & ind & for type $i$ in untyped context\\\hline
    base & = & eq & hol & in typed context\\\hline
    base & exists & forall & pl1 & \\\hline
    base & forall & forall & pl1 & \\\hline 
    base & atmost-one & atmost @ 1 & pl1 & \\\hline
    base & exists-unique & exists-unique & pl1 & \\\hline
    base & that & ??? & ???& \\\hline 
    base & choose-from & ??? & ??? & \\\hline 
  \end{tabular}
  
  {\tt{void}} and {\tt{focus}} are defined in base.thy, but never used as far as I
  can tell. {\tt{focus}} is also defined in various other theories, but not used.

\end{appendix}
\end{document}
%%% Local Variables: 
%%% mode: latex
%%% TeX-master: t
%%% End: 
