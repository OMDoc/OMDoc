%%%%%%%%%%%%%%%%%%%%%%%%%%%%%%%%%%%%%%%%%%%%%%%%%%%%%%%%%%%%%%%%%%%%%%%%%
% This file is part of the LaTeX sources of the OMDoc 1.6 primer
% Copyright (c) 2006 Michael Kohlhase
% This work is licensed by the Creative Commons Share-Alike license
% see http://creativecommons.org/licenses/by-sa/2.5/ for details
\svnInfo $Id: xmlrpc.tex 8431 2009-07-19 11:33:00Z kohlhase $
\svnKeyword $HeadURL: https://svn.omdoc.org/repos/omdoc/trunk/doc/primer/xmlrpc.tex $
%%%%%%%%%%%%%%%%%%%%%%%%%%%%%%%%%%%%%%%%%%%%%%%%%%%%%%%%%%%%%%%%%%%%%%%%%

\begin{omgroup}[id=rpc,short=Communication between Systems]
                           {Communication with and between Mathematical Software Systems}

{\omdoc} can be used as content language for communication protocols between
mathematical software systems on the Internet. The ability to specify the context
and meaning of the mathematical objects makes the {\omdoc} format ideally suited
for this task.

In this chapter we will discuss a message interface in a fictitious software system
{\mathwebws}\footnote{``{\mathweb} {\bf{W}}eb {\bf{S}}ervices''; The examples discussed in
  this chapter are inspired by the {\mathwebsb}~\cite{FraKoh:mabdl99,ZimKoh:tmsbdmr02}
  (``{\mathweb} {\bf{S}}oftware {\bf{B}}us'') service infrastructure, which offers similar
  functionality based on the {\xmlrpc} protocol (an {\xml} encoding of Remote Procedure
  Calls (RPC)~\cite{xmlrpc}). We use the {\soap}-based formulation, since {\soap}
  ({\bf{S}}imple {\bf{O}}bject {\bf{A}}ccess {\bf{P}}rotocol) is the relevant
  {\indextoo{W3C}} standard and we can show the embedding of {\omdoc} fragments into other
  {\xml} namespaces. In {\xmlrpc}, the {\xml} representations of the content language
  {\omdoc} would be transported as base-64-encoded strings, not as embedded {\xml}
  fragments. }, which connects a wide-range of {\twintoo{reasoning}{system}s}
({\emph{\twintoo{mathematical}{service}s}}), such as {\atwintoo{automated}{theorem}
  {prover}s}, {\atwintoo{automated}{proof}{assistant}s}, {\twintoo{computer
    algebra}{system}s}, {\twintoo{model}{generator}s}, {\twintoo{constraint}{solver}s},
human interaction units, and {\atwintoo{automated}{concept formation} {system}s}, by a
common {\emph{\twintoo{mathematical}{software bus}}}.  Reasoning systems integrated in
{\mathwebws} can therefore offer new services to the pool of services, and can in turn use
all services offered by other systems.

On the protocol level, {\mathwebws} uses {\soap} remote procedure calls with the HTTP
binding~\cite{GudHad:soapad03} (see~\cite{Mitra:soapPrimer03} for an introduction to
{\soap}) interface that allows client applications to request service objects and to use
their service methods. For instance, a client can simply request a service object for the
automated theorem prover {\spass}~\cite{Weidenbach:sv97} via the {\http} {\sf{GET}}
request in {\mylstref{discover-spass}} to a {\mathwebws} broker node.

\begin{lstlisting}[label=lst:discover-spass,
          caption={Discovering Automated Theorem Provers (Request)}]
GET /ws.mathweb.org/broker/getService?name=SPASS  HTTP/1.1
Host: ws.mathweb.org
Accept: application/soap+xml
\end{lstlisting}

As a result, the client receives a {\soap} message like the one in
{\mylstref{rpc-spass}} containing information about various instances of
services embodying the {\spass} prover known to the broker service.

\begin{lstlisting}[label=lst:rpc-spass,
          caption={Discovering Automated Theorem Provers (Response)}]
HTTP/1.1 200 OK
Content-Type: application/soap+xml
Content-Length: 990

<?xml version='1.0'?>
<env:Envelope xmlns:env="http://www.w3.org/2003/05/soap-envelope">
  <env:Body>
    <ws:prover env:encodingStyle="http://www.w3.org/2003/05/soap-encoding"
        xmlns:ws="http://www.mathweb.org/ws-fictional">
      <ws:name>SPASS</ws:name>
      <ws:version>2.1</ws:version>
      <ws:URL>http://spass.mpi-sb.mpg.de/webspass/soap</ws:URL>
      <ws:uptime>P3D5H6M45S</ws:uptime>
      <ws:sysinfo>
        <ws:ostype>SunOS 5.6</ws:ostype>
        <ws:mips>3825</ws:mips>
      </ws:sysinfo>
    </ws:prover>
    <ws:prover env:encodingStyle="http://www.w3.org/2003/05/soap-encoding"
        xmlns:ws="http://www.mathweb.org/ws-fictional">
      <ws:name>SPASS</ws:name>
      <ws:version>2.0</ws:version>
      <ws:URL>http://asuka.mt.cs.cmu.edu/atp/spass/soap</ws:URL>
      <ws:uptime>P5M2D15H56M5S</ws:uptime>
      <ws:sysinfo>
        <ws:ostype>linux-2.4.20</ws:ostype>
        <ws:mips>1468</ws:mips>
      </ws:sysinfo>
    <ws:prover>
  </env:Body>
</end:Envelope>
\end{lstlisting}

The client can then select one of the provers (say the first one, because it runs
on the faster machine) and post theorem proving requests like the one in
{\mylstref{rpc-prover}}\footnote{We have made the namespaces involved explicit
  with prefixes in the examples, to show the mixing of different {\xml}
  languages.} to the URL which uniquely identifies the service object in the
Internet (this was part of the information given by the broker; see line 11 in
{\mylstref{rpc-spass}}).

\begin{lstlisting}[label=lst:rpc-prover,
  caption={A {\soap} RPC call to {\spass}}]
POST http://spass.mpi-sb.mpg.de/webspass/soap HTTP/1.1
Host: http://spass.mpi-sb.mpg.de/webspass/soap
Content-Type: application/soap+xml; 
Content-Length: 1123

<?xml version='1.0'?>
<env:Envelope xmlns:env="http://www.w3.org/2003/05/soap-envelope">
  <env:Body>
    <ws:prove env:encodingStyle="http://www.w3.org/2003/05/soap-encoding"
        xmlns:ws="http://www.mathweb.org/ws-fictional">
      <omdoc:assertion xml:id="peter-hates-somebody" type="conjecture"
             xmlns:omdoc="http://omdoc.org/ns"
             theory="http://mbase.mathweb.org:8080/RPC2#lovelife"> 
        <h:p>Peter hates somebody</h:p> 
        <omdoc:FMP> 
         <om:OMBIND xmlns:om="http://www.openmath.org/OpenMath"> 
           <om:OMS cd="quant1" name="exists"/> 
           <om:OMBVAR><om:OMV name="X"/></om:OMBVAR> 
           <om:OMA> 
             <om:OMS cd="lovelife" name="hate"/> 
              <om:OMS cd="lovelife" name="peter"/> 
              <om:OMV name="X"/> 
            </om:OMA> 
          </om:OMBIND> 
       </omdoc:FMP> 
      </omdoc:assertion> 
      <ws:replyWith><ws:state>proof</ws:state></ws:replyWith>
      <ws:timeout>20</ws:timeout>
    </ws:prove>
  </env:Body>
</env:Envelope>
\end{lstlisting}
This {\soap} remote procedure call uses a generic method ``{\snippet{prove}}'' that can be
understood by the {\atwintoo{first-order}{theorem}{prover}s} on {\mathwebsb}, and in
particular the {\spass} system. This method is encoded as a {\element[ns-elt=ws]{prove}}
element; its children describe the proof problem and are interpreted by the {\soap} RPC
node as a parameter list for the method invocation.  The first parameter is an {\omdoc}
representation of the assertion to be proven. The other parameters instruct the theorem
prover service to reply with the proof (instead of e.g. just a yes/no answer) and gives it
a time limit of 20 seconds to find it.

Note that {\omdoc} fragments can be seamlessly integrated into an {\xml} message format
like {\soap}. A {\soap} implementation in the client's implementation language simplifies
this process drastically since it abstracts from {\http} protocol details and offers
{\soap} nodes using data structures of the host language.  As a consequence, developing
{\mathweb} clients is quite simple in such languages.  Last but not least, both {\msie}
and the open source WWW browser {\firefox} now allow to perform {\soap} calls within
JavaScript. This opens new opportunities for building user interfaces based on web
browsers.

Note furthermore that the example in {\mylstref{rpc-prover}} depends on the information
given in the theory {\snippet{lovelife}} referenced in the {\attribute{theory}{assertion}}
attribute in the {\element{assertion}} element (see {\extref{spec}{theories}} for a
discussion of the theory structure in {\omdoc}). In our instance, this theory might
contain formalizations (in first-order logic) of the information that Peter hates
everybody that Mary loves and that Mary loves Peter, which would allow {\spass} to prove
the assertion. To get the information, the {\mathwebws} service based on {\spass} would
first have to retrieve the relevant information from a knowledge base like the {\mbase}
system described in {\extref{projects}{mbase}} and pass it to the {\spass} theorem prover
as background information. As {\mbase} is also a {\mathwebws} server, this can be done by
sending the query in {\mylstref{rpc-getTheory}} to the {\mbase} service at
\url{http://mbase.mathweb.org:8080}.

\begin{lstlisting}[label=lst:rpc-getTheory,  
  caption={Requesting a Theory from {\mbase}}]
GET /mbase.mathweb.org:8080/soap/getTheory?name=lovelife  HTTP/1.1
Host: mbase.mathweb.org:8080
Accept: application/soap+xml
\end{lstlisting}
The answer would be of the form given in {\mylstref{rpc-theory}}. Here, the
{\soap} envelope contains the {\omdoc} representation of the requested theory
(irrespective of what the internal representation of {\mbase} was).
\begin{lstlisting}[label=lst:rpc-theory,mathescape,
  caption={The Background Theory for Message {\ref{lst:rpc-prover}}}]
HTTP/1.1 200 OK
Content-Type: application/soap+xml
Content-Length: 602

<?xml version='1.0'?>
<env:Envelope xmlns:env="http://www.w3.org/2003/05/soap-envelope">
  <env:Body>
    <theory xml:id="lovelife" xmlns="http://omdoc.org/ns"> 
      <symbol name="peter"/><symbol name="mary"/> 
      <symbol name="love"/><symbol name="hate"/> 
      <axiom xml:id="opposite"> 
        <h:p>Peter hates everybody Mary loves</h:p> 
        <FMP>$\allcdot{x}{loves(mary,x)\Rightarrow hates(peter,x)}$</FMP> 
      </axiom> 
      <axiom xml:id="mary-loves-peter"> 
        <h:p>Mary loves Peter</h:p> 
        <FMP>$loves(mary,peter)$</FMP>
      </axiom>
    </theory>
  </env:Body>
</env:Envelope>
\end{lstlisting}
This information is sufficient to prove the theorem in {\mylstref{rpc-prover}}; and the
{\spass} service might reply to the request with the message in {\mylstref{rpc-proof}}
which contains an {\omdoc} representation of a proof (see {\extref{spec}{proofs}} for
details). Note that the {\attribute{for}{proof}} attribute in the {\element{proof}}
element points to the original assertion from {\mylstref{rpc-prover}}.
\begin{lstlisting}[label=lst:rpc-proof,mathescape,
  caption={A proof that Peter hates someone}]
HTTP/1.1 200 OK
Content-Type: application/soap+xml
Content-Length: 588

<?xml version='1.0'?>
<env:Envelope xmlns:env="http://www.w3.org/2003/05/soap-envelope">
  <env:Body>
    <proof xml:id="p347" for="#peter-hates-somebody"
           xmlns="http://omdoc.org/ns"> 
      <derive xml:id="d1">
        <FMP>$hates(peter,peter)$</FMP>
        <method xref="nd.omdoc#ND.chain">
          <premise xref="#lovelife.mary-loves-peter"/>
          <premise xref="#lovelife.opposite"/>
        </method>
      </derive>
      <derive xml:id="concl">
        <method xref="nd.omdoc#ND.ExI"><premise xref="#d1"/></method>
      </derive>
    </proof>
  </env:Body>
</env:Envelope>
\end{lstlisting} 
The proof has two steps: The first one is represented in the {\element{derive}} element,
which states that ``Peter hates Peter''. This fact is derived from the two axioms in the
theory {\snippet{lovelife}} in {\mylstref{rpc-theory}} (the {\element{premise}} elements
point to them) by the ``chaining rule'' of the
{\atwintoo{natural}{deduction}{calculus}}. This inference rule is represented by a symbol
in the theory {\snippet{ND}} and referred to by the {\attribute{xref}{method}} attribute
in the {\element{method}} element.  The second proof step is given in the second
{\element{derive}} element and concludes the proof. Since the assertion of the conclusion
is the statement of the proven assertion, we do not have a separate {\element{FMP}}
element that states this here. The sole premise of this proof step is the previous one.
For details on the representation of proofs in {\omdoc} see {\extref{spec}{proofs}}.

Note that the {\spass} theorem prover does not in itself give proofs in the
{\atwintoo{natural}{deduction}{calculus}}, so the {\spass} service that provided this
answer presumably enlisted the help of another {\mathwebws} service like the {\tramp}
system~\cite{Meier:sdttom00} that transforms resolution proofs (the native format of the
{\spass} prover) to natural deduction proofs.
\end{omgroup}

%%% Local Variables: 
%%% mode: latex
%%% TeX-master: "main"
%%% End: 

% LocalWords:  xmlrpc spass methodCall methodName getService params param h:p
% LocalWords:  struct lovelife FMP cd quant replyWith timeout JavaScript mary
% LocalWords:  getTheory mathescape methodResponse lst int xref concl ExI
% LocalWords:  nnnn env encodingStyle uptime sysinfo ostype SunOS mips omdoc om
% LocalWords:  mbase rpc eb ervices oftware imple bject ccess rotocol xmlns ws
% LocalWords:  ns elt OMBIND OMBVAR OMV OMA
