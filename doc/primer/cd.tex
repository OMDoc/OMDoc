%%%%%%%%%%%%%%%%%%%%%%%%%%%%%%%%%%%%%%%%%%%%%%%%%%%%%%%%%%%%%%%%%%%%%%%%%
% This file is part of the LaTeX sources of the OMDoc 1.6 primer
% Copyright (c) 2006 Michael Kohlhase
% This work is licensed by the Creative Commons Share-Alike license
% see http://creativecommons.org/licenses/by-sa/2.5/ for details
\svnInfo $Id: cd.tex 8431 2009-07-19 11:33:00Z kohlhase $
\svnKeyword $HeadURL: https://svn.omdoc.org/repos/omdoc/trunk/doc/primer/cd.tex $
%%%%%%%%%%%%%%%%%%%%%%%%%%%%%%%%%%%%%%%%%%%%%%%%%%%%%%%%%%%%%%%%%%%%%%%%%

\begin{omgroup}[id=omcds]{OpenMath Content Dictionaries}

  Content Dictionaries are structured documents used by the {\openmath}
  standard~\cite{BusCapCar:2oms04} to codify knowledge about mathematical symbols and
  concepts used in the representation of mathematical formulae. They differ from the
  mathematical documents discussed in the last chapter in that they are less geared
  towards introduction of a particular domain, but act as a reference/glossary document
  for implementing and specifying mathematical software systems. Content Dictionaries are
  important for the {\omdoc} format, since the {\omdoc} architecture, and in particular
  the integration of {\openmath} builds on the equivalence of {\openmath} content
  dictionaries and {\omdoc} theories.

  Concretely, we will look at the content dictionary {\snippet{arith1.ocd}} which defines
  the {\openmath} symbols {\snippet{abs}}, {\snippet{divide}}, {\snippet{gcd}},
  {\snippet{lcm}}, {\snippet{minus}}, {\snippet{plus}}, {\snippet{power}},
  {\snippet{product}}, {\snippet{root}}, {\snippet{sum}}, {\snippet{times}},
  {\snippet{unary\_minus}} (see~\cite{URL:omcd-core} for the original). We will discuss
  the transformation of the parts listed below into {\omdoc} and see from this process
  that the {\openmath} content dictionary format is (isomorphic to) a subset of the
  {\omdoc} format.  In fact, the {\openmath}2 standard only presents the content
  dictionary format used here as one of many encodings and specifies abstract conditions
  on content dictionaries that the {\omdoc} encoding below also meets. Thus {\omdoc} is a
  valid content dictionary encoding.

\begin{lstlisting}[language=omCD,label=lst:omcd,mathescape,
    caption={Part of the {\openmath} content dictionary {\snippet{arith1.ocd}}}]
<CD>
  <CDName> arith1 </CDName>
  <CDURL> http://www.openmath.org/cd/arith1.ocd </CDURL>
  <CDReviewDate> 2003-04-01 </CDReviewDate>
  <CDStatus> official </CDStatus>
  <CDDate> 2001-03-12 </CDDate>
  <CDVersion> 2 </CDVersion>
  <CDRevision> 0 </CDRevision>
  <dc:description> 
    This CD defines symbols for common arithmetic functions.
  </dc:description>

  <CDDefinition>
  <Name> lcm </Name>
  <Description> 
    The symbol to represent the n-ary function to return the least common
    multiple of its arguments.
  </Description>

  <h:p> lcm(a,b) = a*b/gcd(a,b) </h:p>
  <FMP>$\ldots$ </FMP>

  <h:p>
    for all integers a,b |
    There does not exist a c>0 such that c/a is an Integer and c/b is an
    Integer and lcm(a,b) > c.
  </h:p>
  <FMP>$\ldots$</FMP>
  $\ldots$
</CD>
\end{lstlisting}

\noindent Generally, {\openmath} content dictionaries are
represented as mathematical theories in {\omdoc}. These act as
containers for sets of symbol declarations and knowledge about them,
and are marked by {\element{theory}} elements. The result of the
transformation of the content dictionary in {\mylstref{omcd}} is the
{\omdoc} document in {\mylstref{omdoccd}}.

The first 25 lines in {\mylstref{omcd}} contain administrative information and
{\indextoo{metadata}} of the content dictionary, which is mostly incorporated into
the metadata of the {\element{theory}} element. The translation adds further
metadata to the {\element{omdoc}} element that were left implicit in the original,
or are external to the document itself. These data comprise information about the
translation process, the creator, and the terms of usage, and the source, from
which this document is derived (the content of the {\element[ns-elt=omcd]{CDURL}} element is
recycled in Dublin Core metadata element {\element[ns-elt=dc]{source}} in line 12.

The remaining administrative data is specific to the content dictionary per se, and
therefore belongs to the {\element{theory}} element. In particular, the
{\element[ns-elt=omcd]{CDName}} goes to the {\attribute[ns-attr=xml]{id}{omdoc}} attribute on the
{\element{theory}} element in line 36. The {\element[ns-elt=dc]{description}} element is
directly used in the {\element{metadata}} in line 38.  The remaining information is
encapsulated into the {\attribute{cd*}{theory}} attributes.

Note that we have used the {\omdoc} sub-language ``{\omdoc} Content Dictionaries''
described in {\extref{spec}{sub-languages.cd}} since it suffices in this case, this is
indicated by the {\attribute{modules}{omdoc}} attribute on the {\element{omdoc}} element.

\begin{lstlisting}[label=lst:omdoccd,escapechar=\|,mathescape,
    caption={The {\openmath} content dictionary {\snippet{arith1}} in {\omdoc} form},
    index={DOCTYPE,omdoc,metadata,dc:title,dc:creator,dc:date,dc:description,dc:source,dc:type,dc:format,theory
           omtext,h:p}]
<?xml version="1.0" encoding="utf-8"?>
<omdoc xml:id="arith1.omdoc" modules="@cd"
       xmlns:dc="http://purl.org/dc/elements/1.1/">

<metadata>
  <dc:title>The OpenMath Content Dictionary arith1.ocd in OMDoc Form</dc:title>
  <dc:creator role="trl">Michael Kohlhase</dc:creator>
  <dc:creator role="ant">The OpenMath Society</dc:creator>
  <dc:date action="updated"> 2004-01-17T09:04:03Z </dc:date>
  <dc:source>
    Derived from the OpenMath CD http://www.openmath.org/cd/arith1.ocd.
  </dc:source>
  <dc:type>Text</dc:type>
  <dc:format>application/omdoc+xml</dc:format>
  <dc:rights>Copyright (c) 2000 Michael Kohlhase;
    This OMDoc content dictionary is released under the OpenMath license:
    http://www.openmath.org/cdfiles/license.html
  </dc:rights>
</metadata>

<theory xml:id="arith1"
        cdstatus="official" cdreviewdate="2003-04-01" cdversion="2" cdrevision="0">
  <metadata>
    <dc:title>Common Arithmetic Functions</dc:title>
    <dc:description>This CD defines symbols for common arithmetic functions.</dc:description>
    <dc:date action="updated"> 2001-03-12 </dc:date>
  </metadata>
  <imports from="#sts"/>

  <symbol name="lcm">
    <metadata>
      <dc:description>The symbol to represent the $n$-ary function to return the least common
        multiple of its arguments.
      </dc:description>
      <dc:description xml:lang="de"> 
        Das Symbol f|\"u|r das kleinste gemeinsame Vielfache (als |$n$-\"a|re Funktion).
      </dc:description>
      <dc:subject>lcm, least common mean</dc:subject>
      <dc:subject xml:lang="de">kgV, kleinstes gemeinsames Vielfaches</dc:subject>
    </metadata>
    <type system="sts">
     <OMA><OMS name="mapsto" cd="sts"/>
        <OMA><OMS name="nassoc" cd="sts"/><OMV name="SemiGroup"/></OMA>
        <OMV name="SemiGroup"/>
      </OMA>
    </type>
  </symbol>

  <presentation xml:id="pr_lcm" for="#lcm">
    <use  format="default">lcm</use>
    <use  format="default" xml:lang="de">kgV</use>
    <use format="cmml" element="lcm"/>
  </presentation>

  <definition xml:id="lcm-def" for="#lcm" type="pattern">
    <h:p>We define <OMR href="#lcm-def.O"/> as <OMR href="#lcm-def.1"/></h:p>
    <h:p xml:lang="de">
      Wir definieren <OMR href="#lcm-def.O"/> als <OMR href="#lcm-def.1"/>.
    </h:p>
    <requation>
     <OMA id="lcm-def.O">
        <OMS cd="arith1" name="lcm"/>
        <OMV name="a"/><OMV name="b"/>
     </OMA>
     <OMA id="lcm-def.1">
        <OMS cd="arith1" name="divide"/>
        <OMA><OMS cd="arith1" name="times"/>
          <OMV name="a"/>
          <OMV name="b"/>
        </OMA>
        <OMA><OMS cd="arith1" name="gcd"/>
          <OMV name="a"/>
          <OMV name="b"/>
        </OMA>
      </OMA>
   </requation>
  </definition>

  <theory>
    <imports from="#relation1"/>
    <imports from="#quant1"/>
    <imports from="#logic1"/>

    <assertion xml:id="lcm-prop-3" type="lemma">
      <h:p>For all integers <OMV name="a"/>, <OMV name="b"/> there is no 
        <OMR href="#lcm-prop-3.1"/> such that <OMR href="#lcm-prop-3.2"/> and 
        <OMR href="#lcm-prop-3.3"/> and <OMR href="#lcm-prop-3.4"/>.
      </h:p>
      <h:p xml:lang="de">F|\"u|r alle ganzen Zahlen 
        <OMV name="a"/>, <OMV name="b"/> 
        gibt es kein <OMR href="#lcm-prop-3.1"/> mit   
        <OMR href="#lcm-prop-3.2"/> und 
        <OMR href="#lcm-prop-3.3"/> und 
        <OMR href="#lcm-prop-3.4"/>.
      </h:p>
      <FMP>
        <OMBIND><OMS cd="quant1" name="forall"/>
            <OMBVAR><OMV name="a"/><OMV name="b"/></OMBVAR>
            <OMA><OMS cd="logic1" name="implies"/>
              <OMA>$\ldots$</OMA>
              <OMA><OMS cd="logic1" name="not"/>
                <OMBIND><OMS cd="quant1" name="exists"/>
                  <OMBVAR><OMV name="c"/></OMBVAR>
                  <OMA><OMS cd="logic1" name="and"/>
                    <OMA id="lcm-prop-3.1">$\ldots$</OMA>
                    <OMA id="lcm-prop-3.2">$\ldots$</OMA>
                    <OMA id="lcm-prop-3.3">$\ldots$</OMA>
                    <OMA id="lcm-prop-3.4">$\ldots$</OMA>
                  </OMA>
                </OMBIND>
              </OMA>
            </OMA>
          </OMBIND>
     </FMP>
    </assertion> 
    $\ldots$
  </theory>
  $\ldots$
</theory>
\end{lstlisting}

\noindent One important difference between the original and the {\omdoc} version of the
{\openmath} content dictionary is that the latter is intended for machine manipulation,
and we can transform it into other formats. For instance, the human-oriented presentation
of the {\omdoc} version might look something like the following\footnote{These
  presentation was produced by the style sheets discussed in
  {\extref{processing}{omdoc2pres}}.}:

\begin{myfig}{result}{A human-oriented presentation of the OMDoc CD\vspace{-12pt}}
\hspace*{-12pt}\fbox{\begin{minipage}{11cm}
     \begin{center}
       The OpenMath Content Dictionary arith1.ocd in OMDoc Form\\
       Michael Kohlhase, The OpenMath Society\\
       January 17. 2004\\
       This CD defines symbols for common arithmetic functions.
     \end{center}
     
{\bf Concept 1.} {\tt{lcm}} (lcm, least common mean)\\
 {\bf Type} (sts): ${{SemiGroup}}^* \rightarrow{{SemiGroup}}$\\
 The symbol to represent the $n$-ary function to return the least common multiple of its arguments.\\[1ex]
{\bf Definition 2.}(lcm-def)\\
  We define ${lcm ({a},{b})}$ as ${{{{a}{\cdot} {b}}\over  gcd ({a},{b})}}$\\[1ex]
{\bf Lemma 3.} {\emph{For all integers $a$, $b$ there is no $c> 0$ such that
 $(a|c)$ and $(b|c)$ and $c< lcm (a,b)$.}}
  \end{minipage}}
\end{myfig}

\begin{myfig}{result-de}{A human-oriented presentation in German}
\hspace*{-12pt}\fbox{\begin{minipage}{11cm}
     \begin{center}
       The OpenMath Content Dictionary arith1.ocd in OMDoc form\\
       Michael Kohlhase, The OpenMath Society\\
       17. Januar 2004\\
       This CD defines symbols for common arithmetic functions.
     \end{center}
     
{\bf Konzept 1.} {\tt{lcm}} (kgV, kleinstes gemeinsames Vielfaches) \\
 {\bf Typ} (sts): $SemiGroup^* \rightarrow SemiGroup$\\
  Das Symbol f\"ur das kleinste gemeinsame Vielfache (als $n$-\"are Funktion).\\[1ex]
{\bf Definition 2.}(lcm-def)\\
  Wir definieren $kgV (a,b)$ als $a\cdot b\over  ggT (a,b)$\\[1ex]
{\bf Lemma 3.} {\emph{F\"ur alle ganzen Zahlen $a$, $b$ gibt es kein $c> 0$ mit
 $(a|c)$ und $(b|c)$ und $c< kgV(a,b)$.}}
  \end{minipage}}
\end{myfig}
\end{omgroup}

%%% Local Variables: 
%%% mode: latex
%%% TeX-master: "main"
%%% End: 

% LocalWords:  arith ocd abs gcd lcm omCD lst omcd CDName CDURL CDReviewDate ur
% LocalWords:  CDStatus CDDate CDVersion CDRevision CDUses alg fns linalg quant
% LocalWords:  setname transc omdoccd utf omcddtd trl sts ns elt attr omcd omcd
% LocalWords:  lang Das das kleinste gemeinsame Vielfache als aere Funktion kgV
% LocalWords:   kleinstes gemeinsames Vielfaches mapsto cd nassoc pr omcds dc
% LocalWords:  SemiGroup cmml def eq Fuer escapechar alle ganzen Zahlen es ary
% LocalWords:  gibt kein mit und Konzept kgv Typ mathescape loc se omcd omcd
% LocalWords:  catalgoue fco pres PCDATA xmlns OMR href Wir definieren cdstatus
% LocalWords:  requation cdreviewdate cdversion cdrevision Januar unary h:p FMP
% LocalWords:  CDDefinition omcd omcd metadata omdoc omcd omcd DOCTYPE omtext
% LocalWords:  cdstatus cdreviewdate cdversion cdrevision de das OMA OMV
% LocalWords:  OMBIND forall OMBVAR das ggT omcd omcd omcd omcd cdstatus das
% LocalWords:  cdreviewdate cdversion cdrevision das omcd omcd omcd omcd das
% LocalWords:  cdstatus cdreviewdate cdversion cdrevision das omcd omcd omcd
% LocalWords:  omcd cdstatus cdreviewdate cdversion cdrevision das das cdstatus
% LocalWords:  cdreviewdate cdversion cdrevision das das omcd omcd omcd omcd
% LocalWords:  cdstatus cdreviewdate cdversion cdrevision das das omcd omcd das
% LocalWords:  cdstatus cdreviewdate cdversion cdrevision das
