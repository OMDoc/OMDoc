%%%%%%%%%%%%%%%%%%%%%%%%%%%%%%%%%%%%%%%%%%%%%%%%%%%%%%%%%%%%%%%%%%%%%%%%%
% This file is part of the LaTeX sources of the OMDoc 1.6 book
% Copyright (c) 2006 Michael Kohlhase
% This work is licensed by the Creative Commons Share-Alike license
% see http://creativecommons.org/licenses/by-sa/2.5/ for details
\svnInfo $Id: foreword.tex 8904 2011-01-25 06:37:12Z kohlhase $
\svnKeyword $HeadURL: https://svn.omdoc.org/repos/omdoc/trunk/doc/book/foreword.tex $
%%%%%%%%%%%%%%%%%%%%%%%%%%%%%%%%%%%%%%%%%%%%%%%%%%%%%%%%%%%%%%%%%%%%%%%%%

\begin{omgroup}{Foreword}
\addcontentsline{toc}{section}{Foreword} 

\begin{omtext}
Computers are changing the way we think. Of course, nearly all desk-workers have access to
computers and use them to email their colleagues, search the web for information and
prepare documents.  But I'm not referring to that. I mean that people have begun to think
about what they do in computational terms and to exploit the power of computers to do
things that would previously have been unimaginable.
\end{omtext}

\begin{omtext}
This observation is especially true of mathematicians. Arithmetic computation is one of
the roots of mathematics. Since {\indextoo{Euclid's algorithm}} for finding
{\atwintoo{greatest}{common}{divisor}s}, many seminal mathematical contributions have
consisted of new procedures. But powerful {\twintoo{computer}{graphics}} have now enabled
mathematicians to envisage the behaviour of these procedures and, thereby, gain new
insights, make new conjectures and explore new avenues of research. Think of the explosive
interest in fractals, for instance.  This has been driven primarily by our new-found
ability rapidly to visualise fractal shapes, such as the {\twintoo{Mandelbrot}{set}}.
Taking advantage of these new opportunities has required the learning of new skills, such
as using {\indextoo{computer algebra}} and graphics packages.
\end{omtext}

\begin{omtext}
The argument is even stronger. It is not just that computational skills are a useful
adjunct to a mathematician's arsenal, but that they are becoming essential. Mathematical
knowledge is growing exponentially: following its own version of {\indextoo{Moore's Law}}.
Without computer-based {\twintoo{information}{retrieval}} techniques it will be impossible
to locate relevant theories and theorems, leading to a fragmentation and slowing down of
the field as each research area rediscovers knowledge that is already well-known in other
areas. Moreover, without the use of computers, there are potentially interesting theorems
that will remain unproved. It is an immediate corollary of
{\atwintoo{G\"odel's}{Incompleteness}{Theorem}} that, however huge a proof you think of,
there is a short theorem whose smallest proof {\emph{is}} that huge.  Without a computer
to automate the discovery of the bulk of these huge proofs, then we have no hope of
proving these simple-stated theorems.  We have already seen early examples of this
phenomenon in the {\twintoo{Four-Colour}{Theorem}} and {\twintoo{Kepler's}{Conjecture}} on
{\twintoo{sphere}{packing}}.  Perhaps computers can also help us to navigate, abstract
and, hence, understand these huge proofs.
\end{omtext}

\begin{omtext}
Realising this dream of: computer access to a world repository of
{\twintoo{mathematical}{knowledge}}; visualising and understanding this knowledge; reusing
and combining it to discover new knowledge, presents a major challenge to mathematicians
and informaticians.  The first part of this challenge arises because mathematical
knowledge will be distributed across multiple sources and represented in diverse ways. We
need a {\indextoo{lingua franca}} that will enable this babel of mathematical languages to
communicate with each other. This is why this book --- proposing just such a lingua franca
--- is so important. It lays the foundations for realising the rest of the dream.
\end{omtext}

\begin{omtext}
{\omdoc} is an open markup language for mathematical documents. The `markup' aspect of
{\omdoc} means that we can take existing knowledge and annotate it with the information
required to retrieve and combine it automatically. The `open' aspect of {\omdoc} means
that it is extensible, so {\indextoo{future-proof}ed} against new developments in
mathematics, which is essential in such a rapidly growing and complex field of
knowledge. These are both essential features. Mathematical knowledge is growing too fast
and is too distributed for any centrally controlled solution to its management. Control
must be distributed to the mathematical communities that produce it. We must provide
{\twintoo{lightweight}{mechanism}s} under local control that will enable those communities
to put the produce of their labours into the commonwealth with minimal effort. Standards
are required to enable interaction between these diverse knowledge sources, but they must
be flexible and simple to use. These requirements have informed {\omdoc}'s
development. This book will explain to the
{\atwintoo{international}{mathematics}{community}} what they need to do to contribute to
and to exploit this growing body of distributed mathematical knowledge. It will become
essentially reading for all working mathematicians and mathematics students aspiring to
take part in this new world of shared mathematical knowledge.
\end{omtext}

\begin{omtext}
{\omdoc} is one of the first fruits of the {\atwintoo{Mathematical}{Knowledge}{Management}}
({\sc mkm}) Network (\url{http://www.mkm-ig.org/}). This network combines researchers in
mathematics, informatics and library science. It is attempting to realise the dream of
creating a {\atwintoo{universal}{digital}{mathematics library}} of all mathematical
knowledge accessible to all via the world-wide-web. Of course, this is one of those dreams
that is never fully realised, but remains as a source of inspiration. Nevertheless, even
its partial realisation would transform the way that mathematics is practised and
learned. It would be a dynamic library, providing not just text, but allowing users to run
computer software that would provide visualisations, calculate solutions, reveal
counter-examples and prove theorems. It would not just be a passive source of knowledge
but a partner in mathematical discovery. One major application of this library will be to
{\indextoo{teaching}}. Many of the participants in the {\sc mkm} Network are building
teaching aids that exploit the initial versions of the library. There will be a seamless
transition between teaching aids and research assistants --- as the library adjusts its
contribution to match the mathematical user's current needs. The library will be freely
available to all: all nations, all age groups and all ability levels.
\end{omtext}

\begin{omtext}
I'm delighted to write this foreword to one of the first steps in realising this vision.
\hfill {\emph{Alan Bundy, Edinburgh, 25. May 2006}}
\end{omtext}
\end{omgroup}
%%% Local Variables: 
%%% mode: LaTeX
%%% TeX-master: "main"
%%% End: 

% LocalWords:  odel's mkm Bundy
