%%%%%%%%%%%%%%%%%%%%%%%%%%%%%%%%%%%%%%%%%%%%%%%%%%%%%%%%%%%%%%%%%%%%%%%%%
% This file is part of the LaTeX sources of the OMDoc 1.6 book
% Copyright (c) 2006 Michael Kohlhase
% This work is licensed by the Creative Commons Share-Alike license
% see http://creativecommons.org/licenses/by-sa/2.5/ for details
\svnInfo $Id: abstract.tex 8904 2011-01-25 06:37:12Z kohlhase $
\svnKeyword $HeadURL: https://svn.omdoc.org/repos/omdoc/trunk/doc/book/abstract.tex $
%%%%%%%%%%%%%%%%%%%%%%%%%%%%%%%%%%%%%%%%%%%%%%%%%%%%%%%%%%%%%%%%%%%%%%%%%

\begin{omgroup}[display=flow]{Abstract}
\begin{omtext}[type=introduction]
The {\omdoc} (Open Mathematical Documents) format is a content markup scheme for
(collections of) mathematical documents including articles, textbooks, interactive
books, and courses.  {\omdoc} also serves as the content language for agent
communication of mathematical services on a mathematical software bus.
\end{omtext}

\begin{omtext}
This document describes version 1.6 of the {\omdoc} format, the final and mature
release of {\omdocv{1}}. The format features a modularized language design,
{\openmath} and {\mathml} for representing mathematical objects, and has been
employed and validated in various applications.
\end{omtext}

\begin{omtext}
This {\report} contains the rigorous specification of the {\omdoc} document format, an
{\omdoc} primer with paradigmatic examples for many kinds of mathematical documents.
Furthermore we discuss applications, projects and tool support for {\omdoc}.
\end{omtext}
\end{omgroup}

%%% Local Variables: 
%%% mode: LaTeX
%%% TeX-master: "main"
%%% End: 
