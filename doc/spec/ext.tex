%%%%%%%%%%%%%%%%%%%%%%%%%%%%%%%%%%%%%%%%%%%%%%%%%%%%%%%%%%%%%%%%%%%%%%%%%
% This file is part of the LaTeX sources of the OMDoc 1.6 specification
% Copyright (c) 2006 Michael Kohlhase
% This work is licensed by the Creative Commons Share-Alike license
% see http://creativecommons.org/licenses/by-sa/2.5/ for details
\svnInfo $Id: ext.tex 8427 2009-07-19 10:01:23Z kohlhase $
\svnKeyword $HeadURL: https://svn.omdoc.org/repos/omdoc/trunk/doc/spec/ext.tex $
%%%%%%%%%%%%%%%%%%%%%%%%%%%%%%%%%%%%%%%%%%%%%%%%%%%%%%%%%%%%%%%%%%%%%%%%%

\begin{module}[id=ext]
\begin{omgroup}[id=ext,short=Auxiliary Elements]
                           {Auxiliary Elements (Module {\EXTmodule{spec}})}

Up to now, we have been mainly concerned with providing elements for marking up the
inherent structure of mathematical knowledge in mathematical statements and theories. Now,
we interface {\omdoc} documents with the Internet in general and mathematical software
systems in particular. We can thereby generate presentations from {\omdoc} documents where
formulae, statements or even theories that are active components that can directly be
manipulated by the user or mathematical software systems. \inlinedef{We call these
  documents {\twindef{active}{document}s}}.  For this we have to solve two problems: an
abstract interface for calls to external (web) services\footnote{Compare {\extref{primer}{rpc}}
  in the {\omdoc} Primer.} and a way of storing application-specific data in {\omdoc}
documents (e.g. as arguments to the system calls).

The module {\EXTmodule{spec}} provides a basic infrastructure for these tasks in
{\omdoc}. The main purpose of this module is to serve as an initial point of entry. We
envision that over time, more sophisticated replacements will be developed driven by
applications.

\begin{presonly}
\begin{myfig}{qtcode}{The {\omdoc} Auxiliary Elements for Non-{\xml} Data}
\begin{scriptsize}
\begin{tabular}{|>{\tt}l|>{\tt}l|>{\tt}p{4.5truecm}|c|>{\tt}p{3truecm}|}\hline
{\rm Element}& \multicolumn{2}{l|}{Attributes\hspace*{2.25cm}} & M & Content  \\\hline
             & {\rm Req.} & {\rm Optional}                     & D & \\\hline\hline
 private     &            & xml:id, for, theory, 
                               requires, type, reformulates, class, style
                                               & +  &  data+ \\\hline
% theory is optional for code, isn't it?? --clange
 code        &            & xml:id, for, theory, 
                              requires, type, class, style
                              & +  & input?, output?, effect?, data+ \\\hline
 input       &            & xml:id, style, class   & + & h:p*\\\hline
 output      &            & xml:id, style, class   & + & h:p*\\\hline
 effect      &            & xml:id, style, class   & + & h:p*\\\hline
 data        &            & format, href, size, original, pto, pto-version 
                                               & -- & <![CDATA[\ldots]]> \\\hline
\end{tabular}
\end{scriptsize}
\end{myfig}
\end{presonly}

\begin{omgroup}[id=private]{Non-XML Data and Program Code in OMDoc}

The representational infrastructure for mathematical knowledge provided by {\omdoc} is
sufficient as an output- and library format for
{\atwintoo{mathematical}{software}{system}s} like {\twintoo{computer algebra}{system}s},
{\twintoo{theorem}{prover}s}, or {\atwintoo{theory}{development}{system}s}. In particular,
having a standardized output- and library format like {\omdoc} will enhance system
interoperability, and allows to build and deploy general storage and library management
systems (see {\extref{projects}{mbase}} for an {\omdoc} example). In fact this was one of the
original motivations for developing the format.
  
However, most mathematical software systems need to store and communicate system-specific
data that cannot be standardized in a general knowledge-representation format like
{\omdoc}. Examples of this are pieces of {\indextoo{program}} code, like tactics or proof
search heuristics of tactical theorem provers or linguistic data of
{\atwintoo{proof}{presentation}{system}s}.  Only if these data can be integrated into
{\omdoc}, it will become a full storage and communication format for mathematical software
systems. One characteristic of such system-specific data is that it is often not in {\xml}
syntax, or its format is not fixed enough to warrant for a general {\xml} encoding.
  
\begin{definition}[id=privatecode.def]
  For this kind of data, {\omdoc} provides the {\eldef{private}} and {\eldef{code}}
  elements. As the name suggests, the latter is intended for program code\footnote{There
    is a more elaborate proposal for treating program code in the {\omdoc} arena
    at~\cite{Kohlhase:codemlspec}, which may be integrated into {\omdoc} as a separate
    module in the future, for the moment we stick to the basic approach.} and the former
  for system-specific data that is not program code.
  
The attributes of these elements are almost identical and contain metadata information
identifying system requirements and relations to other {\omdoc} elements. We will first
describe the shared attributes and then describe the elements themselves.
\begin{description}
\item[{\attribute[ns-attr=xml]{id}{private, code}}] for identification.
\item[{\attribute{theory}{private, code}}] specifies the mathematical theory (see
  {\sref{theories-contexts}}) that the data is associated with.
\item[{\attribute{for}{private, code}}] allows to attach data to some other {\omdoc}
  element. Attaching {\element{private}} elements to {\omdoc} elements is the main
  mechanism for system-specific extension of {\omdoc}.
\item[{\attribute{requires}{private, code}}] specifies other data this element depends
  upon as a whitespace-separated list of {\twintoo{URI}{reference}s}.  This allows to
  factor private data into smaller parts, allowing more flexible data storage and
  retrieval which is useful for program code or private data that relies on program
  code. Such data can be broken up into procedures and the call-hierarchy can be encoded
  in {\attribute{requires}{private, code}} attributes. With this information, a storage
  application based on {\omdoc} can always communicate a minimal complete code set to the
  requesting application.
\item[{\attribute{reformulates}{private}}] ({\element{private}} only) specifies a set of
  {\omdoc} elements whose knowledge content is reformulated by the {\element{private}}
  element as a whitespace-separated list of {\twintoo{URI}{reference}s}. For instance, the
  knowledge in the assertion in {\mylstref{private-simplify}} can be used as an algebraic
  simplification rule in the {\scsys{Analytica}}
  {\twintoo{theorem}{prover}}~\cite{ClaKoh:sda03} based on the {\scsys{Mathematica}}
  {\twintoo{computer algebra}{system}}.
\end{description}

The {\element{private}} and {\element{code}} elements contain an optional
{\element{metadata}} element and a set of {\element{data}} elements that contain or
reference the actual data.
\end{definition}

\begin{lstlisting}[label=lst:private-simplify,mathescape,
  caption={Reformulating Mathematical Knowledge},index={private,data}]
  <assertion xml:id="ALGX0">
    <h:p>If $a,b,c,d$ are numbers, then we have $a+b(c+d)=a+bc+bd$.</h:p>
  </assertion>
  <private xml:id="alg-expr-1" pto="Analytica" reformulates="ALGX0">
   <data format="mathematica-5.0">
     <![CDATA[SIMPLIFYRULES[a_ + b_*(c_ + d_) :> a + b*c + b*d /; NumberQ[b]]]]>
   </data>
  </private>
\end{lstlisting}

\begin{definition}[id=data.def]
  The {\eldef{data}} element contains the data in a {\snippetin{CDATA}} section.  Its
  {\attribute{pto}{data}} attribute contains a whitespace-separated list of
  {\twintoo{URI}{reference}s} which specifies the set of systems to which the data are
  related.  The intention of this field is that the data is visible to all systems, but
  should only manipulated by a system that is mentioned here. The
  {\attribute{pto-version}{data}} attribute contains a whitespace-separated list of
  version number strings; this only makes sense, if the value of the corresponding
  {\attribute{pto}{data}} is a singleton. Specifying this may be necessary, if the data or
  even their format change with versions.

  If the content of the {\element{data}} element is too large to store directly in the
  {\omdoc} or changes often, then the {\element{data}} element can be augmented by a link,
  specified by a {\twintoo{URI}{reference}} in the {\attribute{href}{data}} attribute. If
  the {\element{data}} element is non-empty and there is a
  {\attribute{href}{data}}\footnote{e.g. if the {\element{data}} content serves as a cache
    for the data at the URI, or the {\element{data}} content fixes a snapshot of the
    resource at the URI}, then the optional attribute {\attribute{original}{data}}
  specifies whether the {\element{data}} content (value {\attval{local}{original}{data}})
  or the external resource (value {\attval{external}{original}{data}}) is the
  original. The optional {\attribute{size}{data}} attribute can be used to specify the
  content size (if known) or the resource identified in the {\attribute{href}{data}}
  attribute. The {\element{data}} element has the (optional) attribute
  {\attribute{format}{data}} to specify the format the data are in, e.g.
  {\attval{image/jpeg}{format}{data}} or {\attval{image/gif}{format}{data}} for image
  data, {\attval{text/plain}{format}{data}} for text data, {\attval{binary}{format}{data}}
  for system-specific binary data, etc. It is good practice to use the
  {\twintoo{MIME}{type}s}~\cite{FreBor:MIME96} for this purpose whenever applicable.  Note
  that in a {\element{private}} or {\element{code}} element, the {\element{data}} elements
  must differ in their {\attribute{format}{data}} attribute. Their order carries no
  meaning.
\end{definition}

In {\mylstref{private-fig}} we use a {\element{private}} element to specify data for an
image\footnote{actually {\myfigref{bourbaki}} from {\extref{primer}{algebra}}} in various
formats, which is useful in a content markup format like {\omdoc} as the transformation
process can then choose the most suitable one for the target.

\begin{lstlisting}[label=lst:private-fig,index={private,data},
  caption={A {\element{private}} Element for an Image}]
<private xml:id="legacy">
  <metadata>
    <dc:title>A fragment of Bourbaki's Algebra</dc:title>
    <dc:creator role="trl">Michael Kohlhase</dc:creator> 
    <dc:date action="created">2002-01-03T0703</dc:date>
    <dc:description>A fragment of Bourbaki's Algebra</dc:description>
    <dc:source>Nicolas Bourbaki, Algebra, Springer Verlag 1974</dc:source>
    <dc:type>Text</dc:type>
  </metadata>
  <data format="application/x-latex" href="legacy.tex"/>
  <data format="image/jpg" href="legacy.jpeg"/>
  <data format="application/postscript" href="legacy.ps"/>
  <data format="application/pdf" href="legacy.pdf"/>
</private>
\end{lstlisting}

\begin{definition}[id=xputeffect.def]
  The {\element{code}} element is used for embedding pieces of program code into an
  {\omdoc} document.  It contains the documentation elements {\eldef{input}},
  {\eldef{output}}, and {\eldef{effect}} that specify the behavior of the procedure
  defined by the code fragment.  The {\element{input}} element describes the structure and
  scope of the input arguments, {\element{output}} the outputs produced by calling this
  code on these elements, and {\element{effect}} any side effects the procedure may have.
  They contain a {\twintoo{mathematical}{vernacular}} marked up in a sequence of
  {\element[ns-elt=h]{p}} elements\ednote{OMDoc1.2 had the possibility to express FMPs in
    there. The latter may be used for program verification purposes.  Do we want to enable
    this again?}. If any of these elements are missing it means that we may not make any
  assumptions about them, not that there are no inputs, outputs or effects.\ednote{Maybe }
\end{definition}
For instance, to specify that a procedure has no side-effects we need to specify something
like
\begin{lstlisting}
<effect><h:p>None.</h:p></effect>
\end{lstlisting}

These documentation elements are followed by a set of {\element{data}} elements
that contain or reference the program code itself. {\Mylstref{omlet1}} shows an
example of a {\element{code}} element used to store {\indextoo{Java}} code for an
applet.

\begin{lstlisting}[label=lst:callMint,mathescape,escapechar=\%,
  caption={The Program Code for a Java Applet},
  index={code,input,output,effect,data}]
<code xml:id="callMint" requires="org.riaca.cas"> 
  <metadata>
    <dc:description>
      The multiple integrator applet. It puts up a user interface, queries the user for a 
      function, which it then integrates by calling one of several computer algebra systems. 
    </dc:description>
  </metadata>
  <data format="application/x-java-applet">
    <![CDATA[$\ldots$ %\llquote{the {\snippet{callMint}} code goes here}% $\ldots$]]>
  </data> 
  <input><h:p>None: the applet handles input itself.</h:p></input> 
  <output><h:p>The result of the integration.</h:p></output>
  <effect><h:p>None.</h:p></effect> 
</code>
\end{lstlisting}
\end{omgroup}

\begin{oldpart}{MK: this to be rethought and possibly integrated with the
    {\element{h:object}} element, see \tracticket{1547} for a discussion}
\begin{omgroup}[id=applets]{Applets and External Objects in OMDoc}

Web-based text markup formats like {\html} have the concept of an
{\twintoo{external}{object}} or ``{\indextoo{applet}}'', i.e.  a program that can in some
way be executed in the browser or web client during document manipulation. This is one of
the primary format-independent ways used to enliven parts of the document.  Other ways are
to change the {\twintoo{document}{object model}} via an embedded programming language
(e.g.  {\indextoo{JavaScript}}). As this method ({\twintoo{dynamic}{HTML}}) is
format-dependent\footnote{In particular, the JavaScript references the {\html} DOM, which
  in our model is created by a presentation engine on the fly.}, it seems difficult to
support in a content markup format like {\omdoc}.
  
The challenge here is to come up with a format-independent representation of the applet
functionality, so that the {\omdoc} representation can be transformed into the specific
form needed by the respective presentation format.  Most user agents for these
presentation formats have built-in mechanisms for processing common data types such as
text and various image types. In some instances the user agent may pass the processing to
an external application (``{\indextoo{plug-in}s}'').  These need information about the
location of the object data, the {\twintoo{MIME}{type}} associated with the object data,
and additional values required for the appropriate processing of the object data by the
object handler at run-time.

\begin{presonly}
\begin{myfig}{qtomlet}{The {\omdoc} Elements for External Objects}
\begin{scriptsize}
\begin{tabular}{|>{\tt}l|>{\tt}l|>{\tt}p{4cm}|c|>{\tt}l|}\hline
{\rm Element}& \multicolumn{2}{l|}{Attributes} & M & Content  \\\hline
             & {\rm Req.} & {\rm Optional}     & D &           \\\hline\hline
 omlet       & data,  & xml:id, action, show, actuate, class, style & +  &
                    \element{h:p}* \& param* \& data*\\
 param   & name & value, valuetype & - & EMPTY\\\hline
\end{tabular}
\end{scriptsize}
\end{myfig}
\end{presonly}

\begin{definition}[id=omlet.def]
  In {\omdoc}, we use the {\eldef{omlet}} element for applets. It generalizes the {\html}
  {\indextoo{applet}} concept in two ways: The computational engine is not restricted to
  {\indextoo{plug-in}s} of the browser (we do not know what the result format and
  presentation engine will be) and the program code can be included in the {\omdoc}
  document, making document-centered computation easier to manage.
\end{definition}  

Like the {\element[ns-elt=xhtml]{object}} tag, the {\element{omlet}} element can be used
to wrap any text. In the {\omdoc} context, this means that the children of the
{\element{omlet}} element can be any elements or text that can occur in the
{\element{h:p}} element together with {\element{param}} elements to specify the
arguments. The main presentation intuition is that the applet reserves a rectangular space
of a given pre-defined size (specified in the {\css} \twinalt{markup}{CSS}{markup} in the
{\attribute{style}{omlet}} attribute; see {\mylstref{omlet1}}) in the result document
presentation, and hands off the presentation and interaction with the document in this
space to the applet process. The data for the external object is referenced in two
possible ways. Either via the {\attribute{data}{omlet}} attribute, which contains a
{\twintoo{URI}{reference}} that points to an {\omdoc} {\element{code}} or
{\element{private}} element that is accessible (e.g. in the same {\omdoc}) or by embedding
the respective {\element{code}} or {\element{private}} elements as children at the end of
the {\element{omlet}} element. This indirection allows us to reuse the machinery for
storing code in {\omdoc}s. For a simple example see {\mylstref{omlet1}}.
  
The behavior of the external object is specified in the attributes
{\attribute{action}{omlet}}, {\attribute{show}{omlet}} and {\attribute{actuate}{omlet}}
attributes\footnote{These latter two attributes are modeled after the
  {\xlink}~\cite{DeRMal:xlink01} attributes {\attributeshort{show}} and
  {\attributeshort{actuate}}.}.
  
The {\attribute{action}{omlet}} specified the intended action to be performed with the
data. For most objects, this is clear from the {\twintoo{MIME}{type}}. Images are to be
displayed, audio formats will be played, and application-specific formats are passed on to
the appropriate {\indextoo{plug-in}}. However, for the latter (and in particular for
program code), we might actually be interested to display the data in its raw (or suitably
presented) form. The {\attribute{action}{omlet}} addresses this need, it has the possible
values {\attval{execute}{action}{omlet}} (pass the data to the appropriate plug-in or
execute the program code), {\attval{display}{action}{omlet}} (display it to the user in
audio- or visual form), and {\attval{other}{action}{omlet}} (the action is left
unspecified).
  
The {\attribute{show}{omlet}} attribute is used to communicate the desired presentation of
the ending resource on traversal from the starting resource. It has one of the values
{\attval{new}{show}{omlet}} (display the object in a new document),
{\attval{replace}{show}{omlet}} (replace the current document with the presentation of the
external object), {\attval{embed}{show}{omlet}} (replace the {\element{omlet}} element
with the presentation of the external object in the current document), and
{\attval{other}{show}{omlet}} (the presentation is left unspecified).
  
The {\attribute{actuate}{omlet}} attribute is used to communicate the desired timing of
the action specified in the {\attribute{action}{omlet}} attribute.  Recall that {\omdoc}
documents as content representations are not intended for direct viewing by the user, but
appropriate presentation formats are derived from it by a ``presentation process'' (which
may or may not be incorporated into the user agent). Therefore the
{\attribute{actuate}{omlet}} attribute can take the values
{\attval{onPresent}{action}{omlet}} (when the presentation document is generated),
{\attval{onLoad}{action}{omlet}} (when the user loads the presentation document),
{\attval{onRequest}{action}{omlet}} (when the user requests it, e.g. by clicking in the
presentation document), and {\attval{other}{action}{omlet}} (the timing is left
unspecified).
  
The simplest form of an {\element{omlet}} is just the embedding of an external object like
an image as in {\mylstref{omlet-image}}, where the {\attribute{data}{omlet}} attribute
points to the {\element{private}} element in {\mylstref{private-fig}}. For presentation,
e.g. as {\xhtml} in a modern browser, this would be transformed into an
{\element[ns-elt=xhtml]{object}} element~\cite{W3C:xhtml2000}, whose specific attributes
are determined by the information in the {\element{omlet}} element here and those
{\element{data}} children of the {\element{private}} element specified in the
{\attribute{data}{omlet}} attribute of the {\element{omlet}} that are chosen for
presentation in {\xhtml}. If the action specified in the {\attribute{action}{omlet}}
attribute is impossible (e.g. if the contents of the {\attribute{data}{omlet}} target
cannot be presented), then the content of the {\element{omlet}} element is processed as a
fallback.

\begin{lstlisting}[label=lst:omlet-image,
  caption={An {\element{omlet}} for an Image},index={omlet}]
<omlet data="#legacy" show="embed">A Fragment of Bourbaki's Algebra</omlet>
\end{lstlisting}

In {\mylstref{omlet1}} we present an example of a conventional
{\twintoo{Java}{applet}} in a mathematical text: the {\attribute{data}{omlet}}
attribute points to a {\element{code}} element, which will be executed (if the
value of the {\attribute{action}{omlet}} attribute were
{\attval{display}{action}{omlet}}, the code would be displayed).

\begin{lstlisting}[label=lst:omlet1,
  caption={An {\element{omlet}} that Calls the Java Applet from {\mylstref{callMint}}.},
  index={omlet}]
<omtext xml:id="monp_1">    
 <h:p>Let practice integration!</h:p>
  <h:p><omlet data="#callMint" action="execute" style="width:320;height:200">
       No plug-in found for callMint!
    </omlet></h:p>
</omtext>
\end{lstlisting}

In this example, the Java applet did not need any parameters (compare the
documentation in the {\element{input}} element in {\mylstref{callMint}}). 

In the applet in {\mylstref{omlet2}} we assume a code fragment or plug-in (in a
{\element{code}} element whose {\attribute[ns-attr=xml]{id}{code}} attribute has the value
{\snippet{sendtoTP}}, which we have not shown) that processes a set of named arguments
(parameter passing with keywords) and calls the theorem prover, e.g. via a web-service as
described in {\extref{primer}{rpc}}.

\begin{lstlisting}[label=lst:omlet2,mathescape,
  caption={An {\element{omlet}} for Connecting to a Theorem Prover},
  index={omlet}]
<h:p> Let us prove it interactively:
  <omlet data="#sendtoTP" action="display">
    <param name="timeout" value="30" valuetype="data"/>
    <param name="performative" value="prove"/>
    <param name="problem" value="#ALGX0" valuetype="object"/>
    <param name="description" value="http://example.org/prob17.html" valuetype="ref"/>
    <param name="instance">
     <om:OMA><OMS name="root" cd="arith1"/>
        <om:OMI>3</om:OMI><om:OMI>3</om:OMI>
      </om:OMA>
   </param>   
    Sorry, no theorem prover available!
  </omlet>
</h:p>
\end{lstlisting}

\begin{definition}[id=parameter.def]
  For parameter passing, we use the {\eldef{param}} elements which specify a set of values
  that may be required to process the object data by a plug-in at run-time. Any number of
  {\element{param}} elements may appear in the content of an {\element{omlet}}
  element. Their order does not carry any meaning. The {\element{param}} element carries
  the attributes
\begin{description}
\item[{\attribute{name}{param}}] This required attribute defines the name of a
  run-time parameter, assumed to be known by the plug-in. Any two {\element{param}}
  children of an {\element{omlet}} element must have different
  {\attribute{name}{param}} values.
\item[{\attribute{value}{param}}] This attribute specifies the value of a run-time
  parameter passed to the plug-in for the key {\attribute{name}{param}}. Property
  values have no meaning to {\omdoc}; their meaning is determined by the plug-in in
  question.
\item[{\attribute{valuetype}{param}}] This attribute specifies the type of the
  {\attribute{value}{param}} attribute. The value
  {\attval{data}{valuetype}{param}} (the default) means that the value of the
  {\attribute{value}{param}} will be passed to the plug-in as a string.  The value
  {\attval{ref}{valuetype}{param}} specifies that the value of the
  {\attribute{value}{param}} attribute is to be interpreted as a
  {\twintoo{URI}{reference}} that designates a resource where run-time values are
  stored. Finally, the value {\attval{object}{valuetype}{param}} specifies that
  the {\attribute{value}{param}} value points to a {\element{private}} or
  {\element{code}} element that contains a {\twintoo{multi-format}{collection}} of
  {\element{data}} elements that carry the data.
\end{description}
If the {\element{param}} element does not have a {\attribute{value}{param}} attribute,
then it may contain a list of mathematical objects encoded as {\openmath}, content
{\mathml}, or {\element{legacy}} elements.
\end{definition}
\end{omgroup}
\end{oldpart}
\end{omgroup}
\end{module}

%%% Local Variables: 
%%% mode: latex
%%% TeX-master: "main"
%%% End: 

% LocalWords:  qtcode pto gif href codebase classid omlet lst argstr preslink
% LocalWords:  escapechar callMint java monp sendtoloui bla sendtotp uniq ext
% LocalWords:  Req mathescape ALGX alg expr mathematica SIMPLIFYRULES NumberQ
% LocalWords:  PPT bourbaki trl tex jpg ps pdf JavaScript XHTML xlink param rpc
% LocalWords:  ECMAScript sendtoTP valuetype performative ref cd arith mathml
% LocalWords:  MathDoc truecm FMP CDATA mbase Analytica HTML html omtext dc
% LocalWords: OMA OMS OMI javascript omdoc DOM qtomlet onPresent onLoad
% LocalWords:  onRequest om ns attr elt metadata bc bd Verlag sendtoTP sendtoTP
% LocalWords:  multi sendtoTP sendtoTP sendtoTP sendtoTP sendtoTP sendtoTP
% LocalWords:  sendtoTP sendtoTP sendtoTP sendtoTP
