%%%%%%%%%%%%%%%%%%%%%%%%%%%%%%%%%%%%%%%%%%%%%%%%%%%%%%%%%%%%%%%%%%%%%%%%%
% This file is part of the LaTeX sources of the OMDoc 1.3 specification
% Copyright (c) 2006 Michael Kohlhase
% This work is licensed by the Creative Commons Share-Alike license
% see http://creativecommons.org/licenses/by-sa/2.5/ for details
%%%%%%%%%%%%%%%%%%%%%%%%%%%%%%%%%%%%%%%%%%%%%%%%%%%%%%%%%%%%%%%%%%%%%%%%%

\begin{tchapter}[id=mtxt,short=Mathematical Text]{Mathematical Text (Modules
  {\MTXTmodule{spec}} and {\RTmodule{spec}})}

The everyday mathematical language used in textbooks, conversations, and written onto
blackboards all over the world consists of a {\indextoo{rigorous}}, slightly stylized
version of {\twintoo{natural}{language}} interspersed with mathematical formulae, that is
sometimes called {\twindef{mathematical}{vernacular}}\footnote{The term ``mathematical
  vernacular'' was first introduced by Nicolaas Govert de Bruijn in the 1970s
  (see~\cite{DeBruijn:tmv94} for a discussion). It derives from the word ``vernacular''
  used in the {\twintoo{Catholic}{church}} to distinguish the language used by
  {\indextoo{laymen}} from the official {\indextoo{Latin}}.}.

\begin{myfig}{qtcfmp}{ Mathematical Text}
\begin{scriptsize}
\begin{tabular}{|>{\tt}l|>{\tt}l|>{\tt}p{2.8truecm}|c|>{\tt}p{4truecm}|}\hline
{\rm Element}& \multicolumn{2}{l|}{Attributes\hspace*{2.25cm}} & D & Content  \\\hline
             & {\rm Required}  & {\rm Optional}  & C &           \\\hline\hline
 omtext      &  & xml:id, type, for, from, class, style, verbalizes    & +  & uses*, CMP+, FMP* \\\hline
 CMP         &  & xml:id, xml:lang                   & -- & \llquote{XHTML Block
   Level}\\\hline
 uses     & from            & id, type, class, style        & + & \\\hline
\end{tabular}
\end{scriptsize}
\end{myfig}


\begin{tsection}[id=mtext]{Multilingual Mathematical Vernacular}
  {\omdoc} models mathematical vernacular as parsed text interspersed with
  content-carrying elements. Most prominently, the {\element[ns-elt=om]{OMOBJ}},
  {\element[ns-elt=m]{math}}, and {\element{legacy}} elements are used for mathematical
  objects, see {\mychapref{mobj}}. Other elements structure the text. In
  {\myfigref{modstable}} we have given an overview over the ones described in this
  book. The last two modules in {\myfigref{modstable}} are optional (see
  {\mysecref{sub-languages}}).  Other (external or future) {\omdoc} modules can introduce
  further elements; natural extensions come when {\omdoc} is applied to areas outside
  mathematics, for instance {\twintoo{computer science}{vernacular}} needs to talk about
  {\twintoo{code}{fragment}s} (see {\mysecref{private}} and~\cite{Kohlhase:codemlspec}),
  {\twintoo{chemistry}{vernacular}} about chemical formulae (e.g. represented in Chemical
  Markup Language~\cite{CML:online}).

  \begin{newpart}{new element description, rethink}
    Recall that \openmath objects are only well-formed, if all the content dictionaries
    are declared in a local catalog\ednote{MK@MK: introduce that earlier, and reference it
      here}. given by \omdoc theories (see \mysecref{theories}). As mathematical
    vernacular contains \openmath objects, we need a mechanism that allows us to define a
    CD catalog. \omdoc provides the \eldef{uses} elements for this. An element
\begin{lstlisting}[label=lst:uses,
  caption={Opening a CD Catalog},
  index={uses}]
<uses from="http://example.org/cds/foo.omdoc#foo"/>
\end{lstlisting}
opens the \omdoc content dictionary \snippet{foo} from the \omdoc document
\url{http://example.org/cds/foo.omdoc}. Note that in contrast to the \element{imports}
element (see \mysecref{inheritance}) the \element{uses} element can be used outside a
\element{theory} element thus does not contribute to the available catalogs.

The \element{uses} element can be used as a direct child of any \omdoc element that can
take a metadata element, and its scope is limited to this element.
\end{newpart}

\begin{myfig}{modstable}{{\omdoc} Modules Contributing to Mathematical Vernacular}
\begin{small}
  \begin{tabular}{|l|p{4cm}|p{3.5cm}|l|}\hline
    Module & Elements & Comment & see\\\hline\hline
    \MOBJmodule{spec} &  {\element[ns-elt=om]{OMOBJ}}, {\element[ns-elt=m]{math}}, {\element{legacy}}
                      & mathematical Objects 
                      & p.~\pageref{chap:mobj}\\\hline
    \MTXTmodule{spec} & {\element{oref}}, {\element{term}}
                      & phrase-level markup
                      & below \\\hline                  
    \DOCmodule{spec}  & {\element{ignore}}
                      & document structure
                      & p.~\pageref{chap:omdoc-infrastructure}\\\hline                  
    \RTmodule{spec}   & Block/Inline level XHTML
                      & rich text structure & p.~\pageref{sec:rt}\\\hline
    \EXTmodule{spec}  & {\element{omlet}} & for applets, images, \ldots 
                      & p.~\pageref{eldef:omlet}\\\hline
  \end{tabular}
\end{small}
\end{myfig}

\begin{tsubsection}{Commented Mathematical Properties}
  To be able to support
  {\twintoo{multilingual}{documents}}\twin{multilingual}{support}\twin{languages}{multiple},
  the mathematical vernacular is represented as a groups of {\eldef{CMP}}\footnote{The
    name comes from ``Commented Mathematical Property'' and was originally taken from
    {\openmath} content dictionaries\index{content dictionary} for continuity
    reasons. Note that {\xml} does note confuse the two, since they are in different
    {\indextoo{namespace}s}.}  elements which contain the vernacular and have an optional
  {\attributeshort[ns-attr=xml]{lang}} attribute that specifies the language they are
  written in. Conforming with the {\xml} recommendation, we use the ISO
  639\atwin{ISO}{639}{norm} two-letter {\twintoo{country}{code}s}
  ({\snippetin{de}}$\;\widehat=\;$German, {\snippetin{en}}$\;\widehat=\;$English,
  {\snippetin{fr}}$\;\widehat=\;$French, {\snippetin{nl}}$\;\widehat=\;$Dutch, \ldots). If
  no {\attributeshort[ns-attr=xml]{lang}} is given, then {\attvalshort{en}{xml:lang}} is
  assumed as the default value. It is forbidden to have two or more sibling
  {\element{CMP}} with the same value of {\attribute[ns-attr=xml]{lang}{CMP}}, moreover,
  {\element{CMP}}s that are siblings must be translations of each other.\footnote{The
    translation requirement may be alleviated in the future, when further variant
    relations are encoded in {\element{CMP}} groups (see~\cite{KohKoh:copmem06} for a
    discussion in the context of ``communities of practice''). Then a generalized
    uniqueness condition must be observed in {\element{CMP}} groups, so that systems can
    choose between the supplied variants.} We speak of a {\twintoo{multilingual}{group}}
  of {\element{CMP}} elements if this is the case.
\begin{lstlisting}[escapechar=\%,label=lst:multiling,mathescape,
  caption={A Multilingual Group of {\element{CMP}} Elements},
  index={trl,xml:lang,CMP,FMP,OMOBJ}]
 <CMP><xhtml:p>
   Let <OMOBJ id="set"><OMV name="V"/></OMOBJ> be a set. 
   A <term role="definiendum">unary operation</term> on 
   <OMOBJ><OMR href="#set"/></OMOBJ> is a function  
   <OMOBJ id="fun"><OMV name="F"/></OMOBJ> with
   <OMOBJ id="im">
     <OMA>
       <OMS cd="relations1" name="eq"/>
       <OMA><OMS cd="fns1" name="domain"/><OMV name="F"/></OMA>
       <OMV name="V"/>
     </OMA>
   </OMOBJ> and 
   <OMOBJ id="ran">
     <OMA>
       <OMS cd="relations1" name="eq"/>
       <OMA><OMS cd="fns1" name="range"/><OMV name="F"/></OMA>
       <OMV name="V"/>
     </OMA>
   </OMOBJ>.</xhtml:p>
 </CMP>
 <CMP xml:lang="de"><xhtml:p>
   Sei <OMOBJ><OMR href="#set"/></OMOBJ> eine Menge. 
   Eine <term role="definiendum">un%\"a%re Operation</term> 
   ist eine Funktion <OMOBJ><OMR href="#fun"/></OMOBJ>, so dass
   <OMOBJ><OMR href="#im"/></OMOBJ> und 
   <OMOBJ><OMR href="#ran"/></OMOBJ>.</xhtml:p>
 </CMP>
 <CMP xml:lang="fr"><xhtml:p>
   Soit <OMOBJ><OMR href="#set"/></OMOBJ> un ensemble. 
   Une <term role="definiendum">op%\'e%ration unaire</term> s%\^u%r
   <OMOBJ><OMR href="#set"/></OMOBJ> est une fonction 
   <OMOBJ><OMR href="#fun"/></OMOBJ> avec 
   <OMOBJ><OMR href="#im"/></OMOBJ> et 
   <OMOBJ><OMR href="#ran"/></OMOBJ>.</xhtml:p>
 </CMP>
\end{lstlisting}

{\mylstref{multiling}} shows an example of such a {\twintoo{multilingual}{group}}. Here,
the {\openmath} extension by {\indextoo{DAG}}\atwin{directed}{acyclic}{graph}
representation (see {\mysecref{openmath}}) facilitates multi-language support: Only the
language-dependent parts of the text have to be rewritten, the (language-independent)
formulae can simply be re-used by cross-referencing\index{cross-reference}.
\end{tsubsection}

\begin{tsubsection}[id=rt]{Paragraph-Level Text Markup}

  Paragraph-level markup is given mostly given by block-level elements of the XHTML Text,
  Table, and List modules of XHTML 1.1~\cite{McCarron:xhtmlmods1.1}. Concretely, these are
  the elements \element{address}, \element{blockquote}, \element{div}, \element{p},
  \element{pre}, \element{ul}, \element{ol}, \element{dl}, and \element{table}. All of
  these elements have been updated to allow \element{metadata} elements and the
  \attributeshort{verbalizes}, \attributeshort{type}, and \attributeshort{index}
  attributes, which we explain next
\end{tsubsection}

\begin{tsubsection}{The Verbalization Relation}\ednote{Maybe this this should go into the
    RST section?}
  The value of the \attributeshort{verbalizes} attribute is a whitespace-separated list of
  URI references that act as pointers to other {\omdoc} elements. This has two
  applications: the first is another kind of parallel markup where we can state that a
  phrase corresponds to (and thus ``verbalizes'') a part of formula in a sibling
  {\element{FMP}} element.

\begin{lstlisting}[label=lst:parallel-formal-informal,mathescape,
  caption=Parallel Markup between Formal and Informal,
  index={xhtml:span,CMP,FMP}]
<CMP>
  <xhtml:p>
    If <xhtml:span verbalizes="#isaG">$\langle G,\circ\rangle$ is a group</xhtml:span>, then of course
       <xhtml:span verbalizes="#isaM">it is a monoid</xhtml:span> by construction.
  </xhtml:p>
</CMP>
<FMP>
  <OMOBJ>
    <OMA><OMS cd="logic1" name="implies"/>
      <OMA id="isaG"><OMS cd="algebra" name="group"/>
        <OMA id="GG"><OMS cd="set" name="pair">
          <OMV name="G"/><OMV name="op"/>
        </OMA>
      </OMA>
      <OMA xml:id="isaM"><OMS cd="algebra" name="monoid"/>
        <OMR href="GG"/>
      </OMA>
    </OMA>
  </OMOBJ>
</FMP>
\end{lstlisting}
Another important application of the \attributeshort{verbalizes} is the case of
inline mathematical statements, which we will discuss in {\mysecref{inline-statements}}.
\end{tsubsection}

\begin{tsubsection}{Parallel Multilingual Markup}
  Recall that sibling {\element{CMP}} elements form {\twintoo{multilingual}{group}s} of
  text fragments.  We can use the paragraph-level and phrase-level element to make the
  correspondence relation on text fragments more fine-grained: elements in sibling
  {\element{CMP}s} that have the same \attributeshort{index} value are considered to be
  equivalent.  Of course, the value of an \attributeshort{index} has to be unique in the
  dominating {\element{CMP}} element (but not beyond). Thus the \attributeshort{index}
  attributes simplify manipulation of {\indextoo{multilingual}} texts, see
  {\mylstref{parallel-multiling}} for an example at the discourse level.

\begin{lstlisting}[label=lst:parallel-multiling,
   caption={Multilingual Parallel Markup},
   index={omtext,CMP,ul,li,p}]
<omtext xml:id="animals.overview">
  <CMP>
    <xhtml:p index="intro">Consider the following animals:</xhtml:p>
    <xhtml:ul index="animals">
      <xhtml:li index="first">a tiger,</xhtml:li>
      <xhtml:li index="second">a dog.</xhtml:li>
    </xhtml:ul>
  </CMP>
  <CMP xml:lang="de">
    <xhtml:p index="intro">Betrachte die folgenden Tiere:</xhtml:p>
    <xhtml:ul index="animals">
      <xhtml:li index="first">Ein Tiger</xhtml:li>
      <xhtml:li index="second">Ein Hund</xhtml:li>
    </xhtml:ul>
  </CMP>
</omtext>
\end{lstlisting}
\end{tsubsection}
\end{tsection}

\begin{tsection}[id=phrases]{Phrase-Level Markup of Mathematical Vernacular}

  To make the sentence-internal structure of mathematical vernacular more explicit,
  {\omdoc} provides an infrastructure to mark up natural language phrases in
  sentences. Linguistically, a {\defin{phrase}} is a group of words that functions as a
  single unit in the syntax of a sentence. Examples include ``noun phrases, verb phrases,
  or prepositional phrases''. 

\begin{tsubsection}{XHTML Phrase-Level Markup}
  Phrase-level markup is given mostly given by inline-level elements of the XHTML Text,
  Applet, Bi-directional Text, Hypertext, Image, and Object, modules of XHTML
  1.1~\cite{McCarron:xhtmlmods1.1}. Concretely, these are the elements \element{abbr},
  \element{acronym}, \element{br}, \element{cite}, \element{code}, \element{dfn},
  \element{em}, \element{kbd}, \element{q}, \element{samp}, \element{span},
  \element{strong}, \element{var}.
\end{tsubsection}


\begin{newpart}{very preliminary, needs better definition}
\begin{tsubsection}{OMDoc References}
  {\omdoc} supplies the {\eldef{oref}} element for referencing fragments of other
  documents\footnote{{\omdocv{1.2}} used the {\oldelement{ref}{1.2}} element with
    {\oldattribute{type}{ref}{1.2}} {\oldattval{cite}{type}{ref}{1.2}} for this
    purpose.}. {\element{oref}} is an inline element that specifies the target element to
  be referenced via a {\attribute{oref}{href}} attribute. Its content is the default link
  text.  The processing of the {\element{oref}} is application specific. It is
  recommended to generate an appropriate label and (optionally) supply a
  hyper-reference. If that is not possible, the default text in the body of the
  \element{oref} element can be used.
\end{tsubsection}
\end{newpart}

\begin{myfig}{qthyper}{Hyperlinkds, Citations, \& References}
  \begin{scriptsize}
\begin{tabular}{|>{\tt}l|>{\tt}l|>{\tt}l|>{\tt}l|}\hline
{\rm Element}& \multicolumn{2}{l|}{Attributes\hspace*{2.25cm}} & Content  \\\hline
 citation & bibrefs & empty \\\hline
 oref & href & default text\\\hline
\end{tabular}
\end{scriptsize}
\end{myfig}

\begin{newpart}{very preliminary, needs better definition}
\begin{tsubsection}{Citations}
  The {\eldef{citation}}\index{citation} element is marks up a citation. Its
  {\attribute{bibrefs}{citation}} attribute references entries in a LaTeXML bibtex/XML
  file.
\end{tsubsection}
\end{newpart}


\begin{myfig}{qtcfmp}{Phrase-level Markup}
\begin{scriptsize}
\begin{tabular}{|>{\tt}l|>{\tt}l|>{\tt}p{2.8truecm}|c|>{\tt}p{4truecm}|}\hline
{\rm Element}& \multicolumn{2}{l|}{Attributes\hspace*{2.25cm}} & D & Content  \\\hline
             & {\rm Required}  & {\rm Optional}  & C &           \\\hline\hline
 term        & cd, name & cdbase, role, xml:id, class, style & -- & \llquote{math vernacular}\\\hline
 citation       &href | bibref & xml:id, class, style & -- & \llquote{math vernacular}\\\hline
 note        & &type, xml:id, style, class, index, verbalizes & + & \llquote{math  vernacular} \\\hline
\end{tabular}
\end{scriptsize}
\end{myfig}

The \attributeshort{type} attribute can be used to specify the (linguistic or
mathematical) type of the phrase, currently {\omdoc} does not make any restrictions on the
values of this attribute, for the mathematical type we recommend to use values for the
{\attribute{type}{omtext}} attribute specified in {\mysecref{omtext}}.


\begin{tsubsection}{Notes}
  The {\eldef{note}} element is the closest approximation to a {\indextoo{footnote}} or
  {\indextoo{endnote}}, where the kind of note is determined by the
  {\attribute{type}{note}} attribute. {\omdoc} supplies {\attval{footnote}{type}{note}} as
  a default value, but does not restrict the range of values. Its {\attribute{for}{note}}
  attribute allows it to be attached to other {\omdoc} elements externally where it is not
  allowed by the {\omdoc} document type. In our example, we have attached a footnote by
  reference to a table row, which does not allow {\element{note}}
  children.\ednote{inline/phrase-level markup}
\end{tsubsection}

\begin{tsubsection}{Index Markup}
  The {\eldef{idx}}\twin{index}{markup} element is used for index markup in {\omdoc}. It
  contains an optional {\eldef{idt}} element that contains the {\twintoo{index}{text}},
  i.e. the phrase that is indexed. 

\begin{myfig}{rt}{Index Markup}
  \begin{scriptsize}
\begin{tabular}{|>{\tt}l|>{\tt}l|>{\tt}l|c|>{\tt}l|}\hline
{\rm Element}& \multicolumn{2}{l|}{Attributes\hspace*{2.25cm}} & D & Content  \\\hline
idx         & &(xml:id|xref)                           & -- & idt?, ide+ \\\hline
 ide         & &index, sort-by, see, seealso, links    & -- & idp* \\\hline
 idt         & &style, class                            & --&  \llquote{math vernacular} \\\hline
 idp         & &sort-by, see, seealso, links            & --&  \llquote{math vernacular} \\\hline
\end{tabular}
\end{scriptsize}
\end{myfig}

The remaining content of the index element specifies what is entered into various
indexes. For every index this phrase is registered to there is one {\eldef{ide}} element
({\twintoo{index}{entry}}); the respective entry is specified by name in its optional
{\attribute{index}{ide}} attribute. The {\element{ide}} element contains a sequence of
{\twintoo{index}{phrase}s} given in {\eldef{idp}} elements. The content of an
{\element{idp}} element is regular mathematical text. Since index entries are usually
sorted, (and mathematical text is difficult to sort), they carry an attribute
{\attribute{sort-by}{idp}} whose value (a sequence of Unicode characters) can be sorted
lexically~\cite{Unicode:collation}. Moreover, each {\element{idp}} and {\element{ide}}
element carries the attributes {\attribute{see}{idp,ide}}, {\attribute{seealso}{idp,ide}},
and {\attribute{links}{idp,ide}}, that allow to specify extra information on these. The
values of the first ones are references to {\element{idx}} elements, while the value of
the {\attribute{links}{idp}} attribute is a whitespace-separated list of (external) URI
references.  The formatting of the {\twintoo{index}{text}} is governed by the attributes
{\attributeshort{style}} and {\attributeshort{class}} on the {\element{idt}} element. The
{\element{idx}} element can carry either an {\attribute[ns-attr=xml]{id}{idx}} attribute
(if this is the defining occurrence of the index text) or an {\attribute{xref}{idx}}
attribute. In the latter case, all the {\element{ide}} elements from the defining
{\element{idx}} (the one that has the {\attribute[ns-attr=xml]{id}{idx}} attribute) are
imported into the referring {\element{idx}} element (the one that has the
{\attribute{xref}{idx}} attribute).
\end{tsubsection}

\begin{oldpart}{MK: this should probably go into module ST.}
\begin{tsubsection}[id=terms]{Technical Terms}
  In {\omdoc} we can give the notion of a {\twindef{technical}{term}} a very precise
  meaning: it is a {\indextoo{phrase}} representing a {\indextoo{concept}} for which a
  {\indextoo{declaration}} exists in a {\twintoo{content}{dictionary}} (see
  {\mysubsecref{symbol-dec}}). In this respect it is the natural language equivalent for
  an {\openmath} symbol or a {\cmathml} token\footnote{and is subject to the same
    visibility and scoping conditions as those; see {\mysecref{theories}} for
    details}. Let us consider an example: We can equivalently say ``$0\in\NN$'' and ``the
  number zero is a natural number''. The first rendering in a formula, we would cast as
  the following {\openmath} object:
\begin{lstlisting}[language=OpenMath,numbers=none]
<OMOBJ>
  <OMA><OMS cd="set1" name="in"/>
    <OMS cd="nat" name="zero"/>
    <OMS cd="nat" name="Nats"/>
  </OMA>
</OMOBJ>
\end{lstlisting}
  with the effect that the components of the formula are disambiguated by pointing to the
  respective content dictionaries. Moreover, this information can be used by added-value
  services e.g. to cross-link the symbol presentations in the formula to their definition
  (see {\mychapref{transform-xsl}}), or to detect logical dependencies. To allow this for
  mathematical vernacular as well, we provide the {\element{term}} element: in our example
  we might use the following markup.
\begin{lstlisting}[language=OpenMath,numbers=none,mathescape]
$\ldots$<term  cd="nat" name="zero">the number zero</term> is an 
<term cd="nat" name="Nats">natural number</term>$\ldots$
\end{lstlisting}
The {\eldef{term}} element has two required attributes: {\attribute{cd}{term}} and
{\attribute{name}{term}}, and optionally {\attribute{cdbase}{term}}, which together determine the meaning of the phrase just like
they do for {\element[ns-elt=om]{OMS}} elements (see the discussion in
{\mysecref{openmath}} and {\mysubsecref{identifying}}). The {\element{term}} element also
allows the attribute {\attribute[ns-attr=xml]{id}{term}} for identification of the phrase
occurrence, the {\css} attributes\twin{CSS}{attribute} for styling and the optional
{\attribute{role}{term}} attribute that allows to specify the role the respective phrase
plays. We reserve the value {\attval{definiens}{role}{term}} for the defining occurrence
of a phrase in a definition.  This will in general mark the exact point to point to when
presenting other occurrences of the same\footnote{We understand this to mean with the same
  {\attribute{cd}{term}} and {\attribute{name}{term}} attributes.} phrase. Other attribute
values for the {\attribute{role}{term}} are possible, {\omdoc} does not fix them at the
current time.  Consider for instance the following text fragment from
{\myfigref{bourbaki}} in {\mychapref{algebra}}.

\begin{quote}
  {\sc{Definition 1.}} {\emph{Let $E$ be a set. A mapping of $E\times E$ is called a {\bf{law
        of composition}} on $E$. The value $f(x,y)$ of $f$ for an ordered pair $(x,y)\in
    E\times E$ is called the {\bf{composition of}} $x$ and $y$ under this law.  A set with
    a law of composition is called a magma.}}
\end{quote}
Here the first boldface term is the definiendum for a ``law of composition'', the second
one for the result of applying this to two arguments. It seems that this is not a totally
different concept that is defined here, but is derived systematically from the concept of
a ``law of composition'' defined before. Pending a thorough linguistic investigation we
will mark up such occurrences with {\tt{definiens-applied}}, for instance in

\begin{lstlisting}[mathescape,caption={Marking up the Technical Terms},label=lst:terms]
Let $E$ be a set. A mapping of $E\times E$ is called a 
<term cd="magmas" name="law_of_comp" role="definiendum">law of composition</term> on $E$. 
The value $f(x,y)$ of $f$ for an ordered pair $(x,y)\in E\times E$ is called the 
<term  cd="magmas"name="law_of_comp" role="definiendum-applied">composition of</term>
$x$ and $y$ under this law.
\end{lstlisting}
There are probably more such systematic correlations; we leave their categorization and
modeling in {\omdoc} to the future.
\end{tsubsection}
\end{oldpart}
\end{tsection}

\begin{newpart}{MK: re-read, extend, make more examples, check all with Frederik; move
    somewhere suitable}
\begin{tsection}[id=declarations]{Declarations and Discourse Referents}
  In mathematics we often see phrases like \nlex{Let $S$ be a set}, or \nlex{\ldots, where
    $f$ is a smooth function on the reals}, or \nlex{for all $\epsilon>0$, \ldots}. These
  introduce identifiers for the use in the text and formulae (with a given scope.)
  Traditionally, these are thought of as variables, but the concept of \defemph{discourse
    referents} from DRT (Discourse Representation Theory~\cite{KamRey:acffodrs96}) seems
  more suitable. 

  We use \eldef{declaration} element to markup such text fragments.\ednote{what about
    attributes? Think about this}. 
\begin{lstlisting}
<declaration for=''S''>
  Let <om:OMOBJ><om:OMV name=''S''/></om:OMOBJ> be a <term>set</term>.
</declaration>
\end{lstlisting}
\ednote{talk about identifiers and res}
\begin{myfig}{rt}{Index Markup}
  \begin{scriptsize}
\begin{tabular}{|>{\tt}l|>{\tt}l|>{\tt}l|c|>{\tt}l|}\hline
{\rm Element}& \multicolumn{2}{l|}{Attributes\hspace*{2.25cm}} & D & Content  \\\hline
  declaration        & & id, role, name & -- & \llquote{math vernacular},  identifiers*, restrictions* \\\hline
 identifiers        & &  id  & -- &x\llquote{math vernacular} \\\hline
 restriction         & &                             & --&  \llquote{math vernacular} \\\hline
\end{tabular}
\end{scriptsize}
\end{myfig}

\end{tsection}
\end{newpart}
\begin{tsection}[id=omtext]{Text Fragments and their Rhetoric/Mathematical Roles}

  As we have explicated above, all mathematical documents state properties of mathematical
  objects --- informally in mathematical vernacular or formally (as logical formulae), or
  both. {\omdoc} uses the {\eldef{omtext}} element to mark up text passages that form
  conceptual units, e.g. paragraphs, statements, or remarks.  {\element{omtext}} elements
  have an optional {\attribute[ns-attr=xml]{id}{omtext}} attribute, so that they can be
  {\indextoo{cross-reference}d}, the intended purpose of the text fragment in the larger
  document context can be described by the optional attribute {\attribute{type}{omtext}}.
  This can take e.g. the values {\attval{abstract}{type}{omtext}},
  {\attval{introduction}{type}{omtext}}, {\attval{conclusion}{type}{omtext}},
  {\attval{comment}{type}{omtext}}, {\attval{thesis}{type}{omtext}},
  {\attval{antithesis}{type}{omtext}}, {\attval{elaboration}{type}{omtext}},
  {\attval{motivation}{type}{omtext}}, {\attval{evidence}{type}{omtext}},
  {\attval{note}{type}{omtext}}, {\attval{transition}{type}{omtext}} with the obvious
  meanings. In the last five cases {\element{omtext}} also has the extra attribute
  {\attribute{for}{omtext}}, and in the last one, also an attribute
  {\attribute{from}{omtext}}, since these are in reference to other {\omdoc} elements.

  The content of an {\element{omtext}} element is {\twintoo{mathematical}{vernacular}}
  contained in a multi-lingual {\element{CMP}} group, followed by an (optional)
  multi-logic {\element{FMP}} group that expresses the same content.  This
  {\element{CMP}} group can be preceded by a {\element{metadata}} element that can be used
  to specify authorship, give the passage a title, etc. (see {\mysecref{dc-elements}}).

  We have used the {\attribute{type}{omtext}} attribute on {\element{omtext}} to classify
  text fragments by their {\twintoo{rhetoric}{role}}. This is adequate for much of the
  generic text that makes up the {\indextoo{narrative}} and explanatory text in a
  mathematical textbook. But many text fragments in mathematical documents directly state
  properties of mathematical objects (we will call them
  {\twintoo{mathematical}{statement}s}; see {\mychapref{statements}} for a more elaborated
  markup infrastructure). These are usually classified as {\indextoo{definition}s},
  {\indextoo{axiom}s}, etc.  Moreover, they are of a form that can (in principle) be
  formalized up to the level of logical formula; in fact, mathematical vernacular is seen
  by mathematicians as a more convenient form of communication for mathematical statements
  that can ultimately be translated into a foundational logical system like axiomatic set
  theory~\cite{Bernays:ast91}.  For such text fragments, {\omdoc} reserves the following
  values for the {\attribute{type}{omtext}} attribute:
\begin{description}
\item[{\attval{axiom}{type}{omtext}}] (fixes or restricts the meaning of certain
  symbols or concepts.) An axiom is a piece of mathematical knowledge that cannot
  be derived from anything else we know.
\item[{\attval{definition}{type}{omtext}}] (introduces new concepts or symbols.) A
  definition is an axiom that introduces a new symbol or construct, without restricting
  the meaning of others.
\item[{\attval{example}{type}{omtext}}] (for or against a mathematical property).
\item[{\attval{proof}{type}{omtext}}] (a proof), i.e. a rigorous --- but maybe informal
  --- argument that a mathematical statement holds.
\item[{\attval{hypothesis}{type}{omtext}}] (a local assumption in a proof that will be
  discharged later) for text fragments that come from (parts of) proofs.
\item[{\attval{derive}{type}{omtext}}] (a step in a proof), we will specify the exact
  meanings of this and the two above in {\mychapref{proofs}} and present more structured
  counterparts.
\end{description} 

Finally, {\omdoc} also reserves the values {\attval{assertion}{type}{omtext}},
{\attval{theorem}{type}{omtext}}, {\attval{proposition}{type}{omtext}},
{\attval{lemma}{type}{omtext}}, {\attval{corollary}{type}{omtext}},
{\attval{postulate}{type}{omtext}}, {\attval{conjecture}{type}{omtext}},
{\attval{false-conjecture}{type}{omtext}}, {\attval{assumption}{type}{omtext}},
{\attval{obligation}{type}{omtext}}, {\attval{rule}{type}{omtext}} and
{\attval{formula}{type}{omtext}} for statements that assert properties of mathematical
objects (see {\myfigref{assertion-types}} in {\mysubsecref{assertions}} for
explanations). Note that the differences between these values are largely pragmatic or
proof-theoretic (conjectures become theorems once there is a proof).  Mathematical
{\element{omtext}} elements (such with one of these types) can have additional
{\element{FMP}} elements (Formal Mathematical Property) that formally represents the
meaning of the descriptive text in the {\element{CMP}}s (if that is feasible).

Further types of text can be specified by providing a URI that points to a description of
the text type (much like the {\attribute[ns-elt=m]{definitionURL}{csymbol}} attribute on
the {\element[ns-elt=m]{csymbol}} elements in {\cmathml}).

Of course, the {\attribute{type}{omtext}} only allows a rough classification of the
{\twintoo{mathematical}{statement}s} at the text level, and does not make the underlying
{\twintoo{content}{structure}} explicit or reveals their contribution and interaction with
{\twintoo{mathematical}{context}}.  For that purpose {\omdoc} supplies a set of
specialized elements, which we will discuss in {\mychapref{statements}}.  Thus
{\element{omtext}} elements will be used to give informal accounts of mathematical
statements that are better and more fully annotated by the infrastructure introduced in
{\mychapref{statements}}. However, in narrative documents, we often want to be informal,
while maintaining a link to the formal element. For this purpose {\omdoc} provides the
optional {\attribute{verbalizes}{omtext}} attribute on the {\element{omtext}} element. Its
value is a whitespace-separated list of URI references to formal representations (see
{\mysecref{inline-statements}} for further discussion).
\end{tsection}

\begin{tsection}[id=FMP]{Formal Mathematical Properties}
  
  An {\eldef{FMP}}\footnote{The name comes from ``Formal Mathematical Properties'' and was
    originally taken from {\openmath} content dictionaries for continuity reasons.}
  element is the general element for representing formal mathematical content in the form
  of {\openmath} objects. {\element{FMP}}s always appear in groups, which can differ in
  the value of their {\attribute{logic}{FMP}} attribute, which specifies the logical
  formalism. The value of this attribute specifies the {\twintoo{logical}{system}} used in
  formalizing the content.  All members of the group have to formalize the same
  mathematical object or property, i.e. they have to be translations of each other, like
  siblings {\element{CMP}}s, we speak of a {\defemph{multi-logic {\element{FMP}}
      group\twin{multi-logic}{group}}} in this case. Furthermore, if an {\element{FMP}}
  group has {\element{CMP}} siblings, all must express the same content.

\begin{myfig}{qtfmp}{Formal Mathematical Properties}
\begin{scriptsize}
\begin{tabular}{|>{\tt}l|>{\tt}l|>{\tt}p{2.8truecm}|c|>{\tt}p{4truecm}|}\hline
{\rm Element}& \multicolumn{2}{l|}{Attributes\hspace*{2.25cm}} & D & Content  \\\hline
             & {\rm Required}  & {\rm Optional}  & C &           \\\hline\hline
FMP         &  & xml:id, logic                      & -- 
             & (assumption*, conclusion*) | {\mobjabbr} \\\hline
 assumption  &  & xml:id, inductive, class, style    & +  & (\mobjabbr)  \\\hline
 conclusion  &  & xml:id, class, style               & +  & (\mobjabbr)  \\\hline
\end{tabular}
\end{scriptsize}
\end{myfig}

  In {\mylstref{omtext-def}} we see two {\element{FMP}} elements, that state the property
  of being a unary operation in two logics. The first one ({\snippet{fol}} for
  {\twintoo{first-order}{logic}}) uses an equivalence to convey the restriction, the
  second one ({\snippet{hol}} for {\twintoo{higher-order}{logic}}) has
  $\lambda$-abstraction and can therefore define the binary predicate {\snippet{binop}}
  directly.

\begin{lstlisting}[escapechar=\%,label=lst:omtext-def,mathescape,
  caption={A multi-logic {\element{FMP}} group for {\mylstref{multiling}}.},
  index={trl,xml:lang,CMP,FMP,OMOBJ}]
<omtext xml:id="binop-def" type="definition">
   %$\ldots$ {\emph{the content of {\mylstref{multiling}}} here $\ldots$%
  <FMP logic="fol">$\allcdot{V,F}{binop(F,V)\Leftrightarrow{\bf{Im}}(F)=V\wedge{\bf{Dom}}(F)=V}$</FMP>
  <FMP logic="hol">$binop=\lambda{V,F}.{\bf{Im}}(F)=V\wedge{\bf{Dom}}(F)=V$</FMP>
</omtext>
\end{lstlisting}

As mathematical statements of properties of objects often come as {\defin{sequent}s},
i.e. as sets of conclusions drawn from a set of assumptions, {\omdoc} also allows the
content of an {\element{FMP}} to be a (possibly empty) set of {\eldef{assumption}}
elements followed by a (possibly empty) set of {\eldef{conclusion}} elements. The intended
meaning is that the {\element{FMP}} asserts that one of the conclusions is entailed by the
assumptions together in the current context.  As a consequence
\begin{lstlisting}[mathescape]
<FMP><conclusion>$A$</conclusion></FMP>
\end{lstlisting}
is equivalent to {\snippet{<FMP>}$A$\snippet{</FMP>}}, where $A$ is an {\openmath},
{\cmathml}, or {\element{legacy}} representation of a mathematical formula. The
{\element{assumption}} and {\element{conclusion}} elements allow to specify the content by
an {\element[ns-elt=om]{OMOBJ}}, {\element[ns-elt=m]{math}}, or {\element{legacy}}
element. The {\element{assumption}} and {\element{conclusion}} elements carry an optional
{\attributeshort[ns-attr=xml]{id}} attribute, which can be used for structure
sharing. This is important for specifying sequent-style proofs (see {\mychapref{proofs}}),
where the assumptions and conclusions of sequents are largely invariant over a proof and
would have to be copied otherwise. The {\element{assumption}} element carries an
additional optional attribute {\attribute{inductive}{assumption}} for inductive
hypotheses\twin{inductive}{hypothesis}.

  In the (somewhat contrived) example in {\mylstref{sequent}} we show a
  {\indextoo{sequent}} for a simple fact about {\twintoo{set}{intersection}}. Here the
  knowledge in both assumptions (together) is enough to entail one of the conclusions (the
  first in this case).

\begin{lstlisting}[mathescape,label=lst:sequent,
  caption={Representing Vernacular as an {\element{FMP}} Sequent},
  index={trl,xml:lang,CMP,FMP,OMOBJ}]
<CMP><xhtml:p>If $a\in{U}$ and $a\in{V}$, then $a\in{U}\cap{V}$ or 
  <xhtml:span index="moon_cheese">the moon is made of green cheese</xhtml:span>.
</xhtml:p></CMP>
<FMP>
  <assumption xml:id="A">$a\in{U}$</assumption>
  <assumption xml:id="B">$a\in{V}$</assumption>
  <conclusion xml:id="C">$a\in{U}\cap{V}$</conclusion>
  <conclusion xml:id="moon_cheese">$made\_of(moon,gc)$</conclusion>
</FMP>
\end{lstlisting}
\end{tsection}



\end{tchapter}
%%% Local Variables: 
%%% mode: latex
%%% TeX-master: "omdoc"
%%% End: 

% LocalWords:  RT ids multiling lst ul li testtext lang Hier testen wir ein ol
% LocalWords:  mehrsprachigen parallelen erster Strichelstrich etwas anderes tr
% LocalWords:  td href xlink rt mtxt Req dtd mobj richtext ednote bn modstable
% LocalWords:  Betrachte folgenden Tiere und Hund ns elt idx attr dl di dt dd
% LocalWords:  Nicolaas Govert ref omlet alsoinCMP alsoincmp lang en fr nl lst
% LocalWords:  escapechar multiling mathescape trl aut func im cd eq fns xref
% LocalWords:  Sei eine Menge ist Funktion da und Une unaire est une fonction
% LocalWords:  avec fol hol ple qtcfmp binop mtxt de Bruijn truecm FMP CMP xml
% LocalWords:  omtext mtext DTD OMOBJ ol ul mobj attribs omstyle une une
% LocalWords:  OMA OMS nat Nats xsl openmath bourbaki magmas comp OMV OMR href
% LocalWords:  un op defini tion jecture definitionURL csymbol def omdoc sump
% LocalWords:  idt idp seealso multi metadata une et ide th une restructurred
% LocalWords:  CoP Soit une isaG isaM une GG une dass une gc dec une
