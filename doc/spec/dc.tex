%%%%%%%%%%%%%%%%%%%%%%%%%%%%%%%%%%%%%%%%%%%%%%%%%%%%%%%%%%%%%%%%%%%%%%%%%
% This file is part of the LaTeX sources of the OMDoc 1.3 specification
% Copyright (c) 2016 Michael Kohlhase.
% Source at https://github.com/KWARC/OMDoc/tree/master/doc/spec
% This work is licensed by the Creative Commons Share-Alike license
% see http://creativecommons.org/licenses/by-sa/2.5/ for details
%%%%%%%%%%%%%%%%%%%%%%%%%%%%%%%%%%%%%%%%%%%%%%%%%%%%%%%%%%%%%%%%%%%%%%%%%

\begin{tchapter}[id=metadata,short=Metadata]{Metadata (Modules {\DCmodule{spec}} and  {\CCmodule{spec}})}

  Metadata\index{metadata} is ``{\twintoo{data}{about data}}'' --- in the case of {\omdoc}
  data about documents, such as titles, authorship, language usage, or administrative
  aspects like modification dates, distribution rights, and identifiers. To accommodate
  such data, {\omdoc} offers the {\element{metadata}} element in many places. The most
  commonly used metadata standard is the Dublin Core vocabulary, which is supported in
  some form by most formats. {\omdoc} uses this vocabulary for compatibility with other
  metadata applications and extends it for {\twintoo{document}{management}} purposes in
  {\omdoc}.  Most importantly {\omdoc} extends the use of metadata from documents to other
  (even mathematical) elements and {\twintoo{document}{fragment}s} to ensure a
  fine-grained authorship and {\twintoo{rights}{management}}.

\begin{tsection}{General Metadata}\label{sec:genmeta}
  \begin{oldpart}{CL: maybe delete or rephrase}
    {\omdocv{1.3}} already integrates the metadata framework for
    {\omdocv{2}} based on the recently stabilized
    RDFa~\cite{AdidaEtAl08:RDFa} a standard for flexibly embedding
    metadata into X(HT)ML documents. This design decision allows us to
    separate the {\emph{syntax}} (which is standardized in RDFa) from
    the {\emph{semantics}}, which we externalize in metadata
    ontologies, which can be encoded in {\omdoc}.
  \end{oldpart}

  \ednote{CL@MK: Do we want/need some motivation for the RDFa syntax
    here, e.\,g.\ some criticism of the \omdocv{1.2} metadata syntax?}

% Some ad-hoc macro definitions to make my LaTeX work here
% CL@MK: You may want to copy some of them to generic style files
\newcommand{\xcmd}[1]{#1\xspace }
\newcommand{\ncmd}[1]{\mbox{#1}}
\newcommand{\xncmd}[1]{\xcmd{\ncmd{#1}}}
\newcommand{\ccrel}{\xncmd{ccREL}}
\newcommand{\curie}{\xncmd{CURIE}}
\newcommand{\curies}{\xncmd{CURIEs}}
\newcommand{\dcmi}{\xncmd{DCMI}}
\newcommand{\mmt}{\xncmd{MMT}}
\newcommand{\rdfs}{\xncmd{RDFS}}
%\newcommand{\uri}{\xncmd{URI}}
\newcommand{\uris}{\xncmd{URIs}}
% linguistics
\newcommand{\cf}[1][\ ]{cf.#1}
\newcommand{\eg}[1][\ ]{e.\,g.#1}
\newcommand{\ie}[1][\ ]{i.\,e.#1}
\newcommand{\wrt}[1][\ ]{w.\,r.\,t.#1}
% semantics
\newcommand{\identifier}[1]{\ifmmode\mathit{#1}\else\textit{#1\/}\fi}
\newcommand{\person}[1]{\textsc{#1}}

Given the need to incorporate additional metadata into \omdoc, and considering the deficiencies of the metadata support in \omdoc 1.2, we developed a new framework.  The requirements were as follows:

\begin{enumerate}
\item Stay backwards-compatible with \omdoc 1.2 concerning expressivity.  That is,
  continue supporting Dublin Core and Creative Commons, and the custom extensions.
\item Expose the formal semantics of metadata vocabularies to \omdoc-based applications;
  additionally be compatible to semantic web applications.
\item Incorporate a vocabulary for versioning -- particularly aiming at technical
  specifications.
\item Do not hard-code a fixed set of vocabularies into the language but stay flexible and
  extensible for many applications, including future and unknown ones.
\end{enumerate}

Given the fact that many existing metadata vocabularies, including
Dublin Core and Creative Commons, have an \rdf semantics\ednote{CL:
  ref somewhere where these vocabularies in general are explained},
and that with \rdfa\ednote{CL: maybe a bit more background on that} a standard for flexibly embedding metadata into \xml had recently stabilized, we chose to incorporate \rdfa into \omdoc, and to look for metadata vocabularies with \rdf-based implementations to satisfy our further requirements.

So far, \rdfa has only been specified for the \enquote{host languages} \xhtml~\cite{AdidaEtAl08:RDFa}.  The specification is generally biased towards \xhtml but nevertheless foresees a future adoption of \rdfa as an annotation sublanguage by other \xml languages.  The vector graphics format \svg Tiny already includes \rdfa in the same way as \xhtml, referring to the \xhtml+\rdfa specification but making a few minor deviations from it.  Other languages are starting to adopt \rdfa as well~\cite{RDFaHostLanguages}.

\paragraph{Full \rdfa in \omdoc}
\label{sec:new-metadata-rdfa}

After initial discussions on how much of \rdfa to incorporate into \omdoc, we decided to give authors who want to model complex annotations freedom to use the full expressivity of \rdfa, but to particularly recommend a metadata syntax that resembles the one of \omdoc 1.2 and allows for expressing most metadata that could also be expressed there. The other reason for fully integrating \rdfa is compatibility to \rdfa tools.  When publishing the sources of \omdoc documents on the web, linked data crawlers like Sindice~\cite{TumDelOre:Sindice07} may find them.  While they would not be able to make any sense of \omdoc's own \xml vocabulary (\eg understanding that a \identifier{proof} element denotes an instance of the \identifier{oo:Proof} class), they would at least be able to understand the annotations made in \rdfa, and thus enable users to search for,  \eg[,] \omdoc resources having the \identifier{dc:creator} \person{Michael Kohlhase}.

A full integration of \rdfa means that the following attribute have to be added to \omdoc, with the same semantics as specified for \xhtml+\rdfa (quoted from~\cite{AdidaEtAl08:RDFa}; technical terms explained below):

\begin{description}
\item[\attribute{rel}] a whitespace-separated list of \curies, used for expressing relationships between two resources (\enquote*{predicates} in \rdf terminology);
\item[\attribute{rev}] a whitespace separated list of \curies, used for expressing reverse relationships between two resources (also \enquote*{predicates});
\item[\attribute{content}] a string, for supplying machine-readable content for a literal (a \enquote*{plain literal object}, in \rdf terminology);
\item[\textmd{[\xhtml-specific attributes omitted]}]
\item[\attribute{about}] a \uri or safe \curie, used for stating what the data is about (a \enquote*{subject} in \rdf terminology);
\item[\attribute{property}] a whitespace separated list of \curies, used for expressing relationships between a subject and some literal text (also a \enquote*{predicate});
\item[\attribute{resource}] a \uri or safe \curie for expressing the partner resource of a relationship that is not intended to be \enquote*{clickable} (also an \enquote*{object});
\item[\attribute{datatype}] a \curie representing a datatype, to express the datatype of a literal;
\item[\attribute{typeof}] a whitespace separated list of \curies that indicate the \rdf   type(s) to associate with a subject.
\end{description}

A \curie (Compact \uri, specified as a part of \rdfa, but also in a
specification of its own~\cite{W3C09:CURIE}) is a way of abbreviating
a \uri as \identifier{namespace:localname}, but in contrast to \xml
local names, the local name definition of
\sparql~\cite{PruSea08:sparql} is used, which is more liberal, \eg
permitting leading digits.  As in \sparql, the underscore prefix is
reserved for blank nodes, such as \identifier{\_:bnode-id}, and names
in the default namespace are written with an empty prefix, \ie as
\identifier{:localname}.  However, the latter namespace is \emph{not}
intended to be the default namespace declared in the surrounding \xml,
but a fixed namespace specified for the language.  In addition to
that, \curies also allow for completely unprefixed names, such as
\identifier{localname}, which can be reserved words whose mapping to
\uris is specified as a part of the language specification.  The
mappings to \uris for the default namespace and for unprefixed names
have been specified for \rdfa in \xhtml, but as there is currently no
standard way of declaring these mappings for a different host
language, \eg in its \xml schema, we do not anticipate that any
\rdfa-aware software\ednote{CL: except our own, maybe mention Krextor
  somewhere.\\ @MK: would it be worth to add a variant of the 6-page
  2009 paper on Krextor to the ``applications'' chapter?} would be able to interpret such \curies.  Therefore, we leave the specification of how \omdoc should handle such \curies as future work.  Some \rdfa attributes allow \uris and \curies, which are generally hard to distinguish.\footnote{The incoherent use of \uris vs.\ \curies in the \rdfa attributes is likely to change in future versions~\cite{Birbeck:ProposalForURIsEverywhere2009}.}  Therefore, a \curie in such an attribute has to be surrounded by square brackets.  This syntax is called \enquote{safe \curie}.

Also note that full \rdfa compatibility leads to a syntactical redundancy in all \omdoc elements that carry metadata.  In \omdoc 1.2, it was clear (by the human-readable specification, not necessarily for machines!) that metadata contained in an \xml element $E$ referred to the concept denoted by $E$, \eg[,] that the \identifier{dc:title} in listing~\ref{fig:fermat-omdoc1.2} is the title of the proof with the \uri \url{#fermat-proof}.  \rdfa requires the subject of annotations to be set explicitly, using the \attribute{about} attribute:

\begin{lstlisting}[language={[1.2]OMDoc}]
<proof xml:id="fermat-proof" about="#fermat-proof">
  <metadata>
     ...
  </metadata>
</proof>
\end{lstlisting}

Otherwise the parent subject would be reused, which is initially the
base \uri, \ie, unless specified otherwise, the \uri of the whole
document -- which may, of course, contain many other metadata records.
\rdfa in \xhtml is often used for talking about different things than
the elements of the \xhtml document itself, such as the book described
in a paragraph of the document, except for annotations on the top
level for expressing, \eg[,] the document's author and license.  In
contrast, metadata in \omdoc are always intended to be annotations for
the things modeled in the document, such as theories or statements.
It is recommended for all of these things to have a \uri, which is
defined by the \attribute{xml:id} attribute.\footnote{The \mmt \uris
  of \omdoc   1.6 will enable additional ways of giving \uris to
  \omdoc concepts, but from an \rdfa   point of view the principle
  remains the same.}\ednote{CL@MK: In this spec, will we have some
  outlook to the things that will definitely come in OMDoc 1.6?}

It would be tempting to specify that, for elements that have metadata
and an \attribute{xml:id}, the \rdfa subject of the metadata
annotations implicitly gets set to the \uri of the respective
element\ednote{CL: TODO HTML5 itemscope}.  One could even specify
that, if an element carrying metadata does not have an
\attribute{xml:id}, a blank node will be generated for it. However,
\xhtml is -- and will always be -- much more widespread than \omdoc,
\rdfa has first been designed for annotating \xhtml and is still
currently biased towards \xhtml, and \rdfa-aware software will
probably not be able to handle custom reinterpretations of the \rdfa
syntax and semantics soon, at least not as long as there is no way of
specifying them in a machine-understandable way\footnote{}.  Now
suppose we had an \omdoc document at an \uri $U$ containing a proof
with \rdfa metadata but without an explicit \attribute{about}
attribute.  Suppose the relation of the proof to the theorem it proves
were, for some reason, not modeled in \omdoc syntax, but in \rdfa,
using the \omdoc ontology\ednote{CL: either omit this aspect, as it's
  not relevant for ``metadata''.  Or describe the OMDoc ontology
  somewhere.  I think it deserves a (sub)section of its own.}, \ie as \texttt{<\textbf{link}   rel="oo:proves" resource="\#fermats-last-theorem">}, which is perfectly legal.  An \rdfa crawler not knowing \omdoc would extract the triple \texttt{<U> oo:proves   <\#fermats-last-theorem>} from that annotation.  From the domain of the \identifier{oo:proves} property, any \rdfs reasoner would then infer that $U$ is an instance of \identifier{oo:Proof}, which is clearly not the case; actually, this would even lead to a contradiction for an \owl reasoner, as \identifier{oo:Proof} is disjoint with \identifier{oo:Document}, of which $U$ actually is an instance.

Realizing that the web should not be polluted with such invalid \rdf
triples\footnote{See also the \enquote{Pedantic Web}
  initiative~\cite{PedanticWeb:FOP09}.}, we therefore specify that
\rdfa metadata in \omdoc must only be used together with correctly
placed \attribute{about} attributes.  A relaxation of this policy is
subject to future additions to the \rdfa specification that might
allow for defining parsing rules specific to particular host
languages\ednote{CL: TODO cite mail thread}.

\paragraph{Recommended Syntax for \rdfa Metadata}
\label{sec:new-metadata-recommended}

I will not cover full \rdfa in further detail here; for an
introduction, see~\cite{w3c:rdfa-primer,HHA:RDFaTutorial08}.  Instead,
I will continue with the recommended syntax for using metadata: We
introduce the elements \identifier{meta} and \identifier{link} as
children of any \identifier{metadata} block.\footnote{Actually, the
  \identifier{link} element has existed before, as a part of \omdoc's
  rich text (RT)   module~\cite[section~14.6]{Kohlhase:OMDoc1.2}.
  However, this usage does not conflict   with its usage as a
  \identifier{metadata} child.}\footnote{Note that the
  \identifier{metadata} element does not exist for \rdfa processors,
  as it does not   carry any \rdfa attributes.  It is merely a means
  of structuring the \omdoc syntax.} Their semantics is roughly
inspired by the namesake elements that can occur in the
\identifier{head} of an \xhtml document: \identifier{meta} is a
literal-valued metadata field, whereas \identifier{link} points to
another resource by referring to its \uri. Resources with
document-local identifiers only, i.\,e.\ \emph{blank nodes}, can be
created using the \identifier{resource} element.  The elements are
shown in table~\ref{tab:rdfa-metadata}; an example for using them is
given in listing~\ref{fig:fermat-omdoc-md}.\ednote{CL: TODO metadata element now optional}

\begin{table}
\begin{tabularx}{\linewidth}{|l|l|X|}
  \hline
  Element & Attributes & Children \\
  \hline \identifier{meta} & \attribute{property}, \attribute{content},   \attribute{datatype} &   literal text or XML (optional) \\
  \identifier{link} & \attribute{rel}, \attribute{rev}, \attribute{resource} &   (\identifier{resource}|\identifier{meta}|\identifier{link})\textsuperscript{*} \\
  \identifier{resource} & \attribute{about}, \attribute{typeof} &   (\identifier{meta}|\identifier{link})\textsuperscript{*} \\
  \hline
\end{tabularx}
  \caption{Elements of the recommended \rdfa syntax for \omdoc metadata}
  \label{tab:rdfa-metadata}
\end{table}

\paragraph{Relevant Metadata Vocabularies}
\label{sec:new-metadata-vocab}

Due to the inherent flexibility of \rdfa, any metadata vocabulary can
be used.  However, we give particular recommendations for metadata in
the above-mentioned domains of special interest.  Using Dublin Core
and Creative Commons metadata with the new \rdfa syntax for \omdoc is
largely trivial.  Concerning Dublin Core, we recommend using the more
modern DCMI terms vocabulary instead of the DCMES, which is now
possible by way of a simple namespace declaration.  While the MARC
roles had been used as annotations of triples with the
\identifier{dc:contributor} property in {\omdoc} 1.2, there is a
specification of how to use them in RDF, defining them as
sub-properties of
\identifier{dc:contributor}~\cite{Johnston:MARC-DC05}.  Most Creative
Commons license declarations will become much easier than in \omdoc
1.2, as we will follow the more recently recommended practice of not
always constructing licenses from scratch, but directly linking
resources to \emph{existing} Creative Commons licenses using the
\identifier{xhv:license} property\footnote{This property from the
  \xhtml vocabulary   supersedes the former \identifier{cc:license}
  property~\cite{AALY08:ccREL}.  By the   implementation of the \ccrel
  ontology, this property is also a subproperty of
  \identifier{dc:license}, which in turn is a subproperty of
  \identifier{dc:rights}.}\ednote{CL: refer to some section where DC
  is described}; for example
\texttt{<\textbf{link}   rel="xhv:license"
  resource="http://creativecommons.org/licenses/by/3.0/de/">}.  It
should also be noted that the \omdoc 1.2 syntax allowed for
constructing licenses that contradicted the \ccrel ontology.  For
example, it was possible to say \texttt{<\textbf{cc:permission}
  derivative\_works="prohibited">}, although
\identifier{cc:DerivativeWorks} is not in the range of the property
\identifier{cc:prohibits}.\footnote{Given that semantic web reasoning
  usually assumes an   open world, one cannot easily conclude from the
  \emph{absence} of the   \emph{permission} to create derivative works
  that it is   prohibited~\cite{W3C:ccREL-comment}.  Therefore, it is
  unclear whether one can   effectively prohibit derivative works
  using the \ccrel vocabulary.  This Orwellian   approach to
  restricting thinking about illiberal licenses by restricting
  language (\cf   \cite{Orwell:1984}) may be debatable, but the \ccrel
  ontology currently specifies it   like this, so we have to accept it
  for the sake of compatibility, or -- eventually --   model our own
  licensing ontology that extends \ccrel.}\ednote{CL: TODO other licensing ontology (CC list mail)}

The \omdoc 1.2 Dublin Core extensions for revision logs were not
immediately \rdf-compatible\ednote{CL: Is that review of the past
  necessary here?}.  We were able to partly replace them by the
revisioning vocabulary of \dcmi terms\ednote{CL: TODO need to describe
DCMI terms (compared to the good old DCMES) somewhere}.  Listing~\ref{fig:fermat-omdoc-md} shows the proof of Fermat's last theorem once more, now redone using \rdfa metadata, and using \dcmi terms for the revision history.  Comparing this to listing~\ref{fig:fermat-omdoc1.2}, particularly note the following features:

\begin{itemize}
\item We are able to link to resources, such as \foaf
  profiles\ednote{CL: explain somewhere}, that describe people (creators, contributors, etc.) in   further detail.
\item More than one predicate can be given per subject and objects.  This makes it   convenient to say that a person is both an editor and a publisher of a   document.\footnote{\identifier{marcrel:AUT} is only a subproperty of     \identifier{dc:contributor}.}
\item The complete revision history can be embedded into the document.
\item Versions (or persons, or licenses) can also be described (as blank nodes) if they   are only known in this document, \ie are not globally identifiable by a \uri.
\item The \dcmi Terms vocabulary allows for modeling the history of revisions more   faithfully than the Dublin Core extensions of \omdoc 1.2.  We can use more specific   subproperties of \identifier{dct:date}, such as \identifier{dct:created} or   \identifier{dct:issued}.  Date can be made really explicit to automated parsers by   declaring a datatype for them; otherwise the parser would have to know that   \identifier{dct:date} and its subproperties usually have an ISO 8601 date   value~\cite{BirMal:XMLSchema:Datatypes}, or it would have to apply heuristics.   Successive revisions can be modeled as a linked list via   \identifier{dct:replaces}, in addition to referring to them by   \identifier{dct:hasVersion}.  We did not model \person{Michael Kohlhase}'s   digitalization of Wiles's proof as such a replacement, but as a resource that is based   on Wiles's proof via the \identifier{dct:requires} and \identifier{dct:source}   properties.
\item The license of this document is a ready-to-use Creative Commons license that can   simply be referenced by its \uri.  Alternatively, we can construct it in place.
\end{itemize}

\begin{lstlisting}[float,caption={Proof of Fermat's last theorem, with \omdoc's new \rdfa metadata},label={fig:fermat-omdoc-md},language={[1.6]OMDoc},escapeinside={\{\}}]
<proof xml:id="fermat-proof" about="#fermat-proof" for="#fermats-last-theorem"
 xmlns:dct="http://purl.org/dc/terms/"
 xmlns:marcrel="http://www.loc.gov/loc.terms/relators/"
 xmlns:xsd="http://www.w3.org/2001/XMLSchema#"
 xmlns:xhv="http://www.w3.org/1999/xhtml/vocab#"
 xmlns:cc="http://creativecommons.org/ns#">
  <metadata>
    <meta property="dct:title">Proof of Fermat{'}s Last Theorem</meta>
    <link rel="dct:creator" {resource}="http://dbpedia.org/resource/Pierre_de_Fermat"/>
    <link rel="marcrel:AUT" {resource}="http://math.princeton.edu/~awiles/foaf.rdf#me"/>
    <link rel="marcrel:EDT dct:publisher"
     resource="http://kwarc.info/kohlhase/"/>
    <link rel="dct:hasVersion">
      <resource about="[_:initial]">
        <!-- Anonymous {resource} (bnode).  We could also point to a URI by which
             the previous {version} can actually be retrieved from a repository -->
        <link rel="dct:creator"
         {resource}="http://dbpedia.org/resource/Pierre_de_Fermat"/>
        <meta property="dct:created" datatype="xsd:date">1637-06-13T00:00:00</meta>
      </resource>
      <resource about="[_:correct]">
        <link rel="dct:replaces" {resource}="[_:initial]"/>
        <link rel="dct:creator"
         {resource}="http://math.princeton.edu/~awiles/foaf.rdf#me"/>
        <meta property="dct:date" datatype="xsd:date">1995-05-01T00:00:00</meta>
      </resource>
      <resource about="[_:digitalized]">
        <link rel="dct:requires dct:source"
         {resource}="[_:correct]"/>
        <link rel="dct:creator"
         {resource}="http://kwarc.info/kohlhase/"/>
        <meta property="dct:issued" datatype="xsd:date">2006-08-28T00:00:00</meta>
      </resource>
    </link>
    <link rel="xhv:license"><!-- actually recommended: directly using
     the pre-defined license http://creativecommons.org/licenses/by/3.0/de/,
     which is the same as what we are constructing here -->
      <meta property="cc:jurisdiction" {content}="de"/>
      <link rel="cc:permits">
        <resource about="[cc:Reproduction]"/>
        <resource about="[cc:Distribution]"/>
        <resource about="[cc:DerivativeWorks]"/>
      </link>
      <link rel="cc:requires">
        <resource about="[cc:Notice]"/>
        <resource about="[cc:Attribution]"/>
      </link>
    </link>
  </metadata>
  <!-- The actual body of the {proof} -->
</proof>
\end{lstlisting}

Compared to \omdoc 1.2, one aspect cannot be expressed with \dcmi
Terms: the actions that lead to new revisions.  One state-of-the-art
ontology that offers the desired expressivity is the Ontology Metadata
Vocabulary~\cite{OMV:on,PHCG:ChangeReprOWL2Onto09} for describing
ontologies\ednote{CL: refer somewhere where we say that by the MOLE
  approach OMDoc documents are basically the same as ontologies}. Instances of \identifier{omv:Ontology} can be
arranged into a list linked via \identifier{omv:hasPriorVersion}.  As
an overlay list to the mere sequence of revisions, a sequence of
changes can be given.  An \identifier{omv:ChangeSpecification}
connects two ontology versions by its properties
\identifier{omv:changeFromVersion} and
\identifier{omv:changeToVersion} and consists of a set of one or more
\identifier{omv:Change}s chained together by
\identifier{omv:hasPreviousChange}.  A change has an author (an
\identifier{omv:Person}), a date, and a few more properties.  OMV
offers a lot of change subclasses specific to \rdfs and \owl
ontologies; we could easily add change types for mathematical
documents, theories, or statements, \eg a change type for adding a
type declaration to a symbol.\ednote{CL: TODO morphism dct$\leftrightarrow$OMV}

\ednote{CL: TODO Also there is potential for interaction rules between DC and CC, \eg if BY(D) and dc:creator(D,A) then ....) -- Interesting. Yes, why not. But then I vote for the following plan     1. first do this as a part of the OMDoc spec to learn how it works    2. but then don't keep it within the OMDoc standard, but try to convince the CC developers    3. only if they don't like our approach, keep it in the OMDoc standard, otherwise contribute it to CC and refer to it from OMDoc. }

\begin{lstlisting}[language={[1.6]OMDoc},escapeinside={\{\}}]
    <!-- TODO: THIS IS OBSOLETE; I WILL REWORK IT INTO AN EXAMPLE USING OMV -->
    <link rel="rev:created_by_act" href="[_:creation]"/>
    <link rel="rev:current_version" href="[_:current]"/>
    <link rel="rev:has_version">
      <resource about="[_:v1]" typeof="rev:Revision">
        <link rel="rev:content" href="fermats-last-theorem?rev=1"/>
        <link rel="rev:created_by_act">
          <resource about="[_:creation]" typeof="chg:Creation">
            <link rel="event:agent" href="http://dbpedia.org/page/Pierre_de_Fermat"/>
            <dc:date>1637-06-13T00:00:00</dc:date>
          </resource>
        </link>
      </resource>
    </link>
    <!-- revision 2 (Wiles{'}s proof) left out to save space -->
    <link rel="rev:has_version">
      <resource about="[_:current]" typeof="rev:Revision">
        <link rel="rev:content" href="fermats-last-theorem?rev=3"/>
        <link rel="rev:created_by_act">
          <resource typeof="chg:Import">
            <link rel="event:agent" href="http://kwarc.info/kohlhase/foaf.rdf#me"/>
            <dc:date>2006-08-28T00:00:00</dc:date>
            <link rel="rev:prior_version" href="[_:v2]"/>
          </resource>
        </link>
      </resource>
    </link>
\end{lstlisting}

\paragraph{Pragmatic Metadata}
\label{sec:new-metadata-pragmatic}

\ednote{CL: I suppose we will keep the old sections on metadata mostly
unchanged from 1.2, so this has to be restructured.}

\begin{oldpart}{integrate}
  As the listing in Sect.~\ref{sec:new-metadata} shows, the new
  \rdfa-based metadata   syntax is much more verbose than the old one
  of {\omdoc} 1.2.  Therefore, we suggest   two ways of facilitating
  the annotation: For manual authoring, one can keep the old,
  \enquote{pragmatic} {\omdoc} 1.2 syntax and specify a transformation
  of such annotations to   the new, \enquote{strict} \rdfa syntax --
  implementable, e.\,g., in XSLT\ednote{CL: TODO align     with~\ref{sec:new-metadata}}.
\end{oldpart}

also consider {\sTeX} as an even more pragmatic metadata syntax .

\paragraph{Respecifying Metadata Inheritance}
\label{sec:new-metadata-inherit}

As I modeled our metadata ontologies in \omdoc, I am now able to extend it by a formal
specification of certain rules that had only informally been stated in the \omdoc 1.2
specification: for example, that most DC metadata propagate from document sections down
into subsections unless subsections specify different values, or that any
\identifier{dc:creator} of a subsection of a document becomes a \identifier{dc:contributor} to the
whole document.

\ednote{CL: TODO model formally in DL (give example): $insection \circ creator \sqsubseteq contributor$}

\ednote{CL: TODO @inherits, compare \activemath}

\ednote{CL: TODO new contribution: can also add metadata from RDF ontologies to OM terms (as attributions)}

\ednote{CL: TODO require importing ontologies when used as \curies?}

\end{tsection}


\begin{tsection}[id=dc-elements]{The Dublin Core Elements (Module {\DCmodule{spec}})}

In the following we will describe the variant of Dublin Core metadata elements used in
{\omdoc}\footnote{Note that {\omdocv{1.2}} systematically changes the Dublin Core {\xml}
  tags to synchronize with the tag syntax recommended by the Dublin Core Initiative. The
  tags were capitalized in {\omdoc}1.1}.  Here, the {\element{metadata}} element can
contain any number of instances of any Dublin Core elements described below in any
order. In fact, multiple instances of the same element type (multiple
{\element[ns-elt=dc]{creator}} elements for example) can be interspersed with other
elements without change of meaning.  {\omdoc} extends the Dublin Core framework with a set
of roles (from the MARC relator set~\cite{Marc:relators03}) on the authorship elements and
with a rights management system based on the Creative Commons Initiative.

\begin{myfig}{qtmetadata}{Dublin Core Metadata in {\omdoc}}
  \begin{scriptsize}
\begin{tabular}{|>{\tt}l|>{\tt}l|>{\tt}l|>{\tt}l|}\hline
{\rm Element}& \multicolumn{2}{l|}{Attributes\hspace*{2.25cm}} & Content  \\\hline
             & {\rm Req.}  & {\rm Optional}     &           \\\hline\hline
 dc:creator     &  & xml:id, class, style, role, type, scheme &  text \\\hline
 dc:contributor &  & xml:id, class, style, role, type, scheme    &  text \\hline
 dc:title       &  & xml:lang, type, scheme    &  \llquote{math vernacular}  \\\hline
 dc:subject     &  & xml:lang, type, scheme    &  \llquote{math vernacular}  \\\hline
 dc:description &  & xml:lang, type, scheme    &  \llquote{math vernacular}  \\\hline
 dc:publisher   &  & xml:id, class, style, type, scheme          &  ANY  \\\hline
 dc:date        &  & action, who, type, scheme &  {\twintoo{ISO}{8601}}  \\\hline
 dc:type        &  &  type, scheme  &  {\rm fixed:} "Dataset" {\rm or\ } "Text" \\\hline
 dc:format      &  & type, scheme            &  {\rm fixed:} "application/omdoc+xml"  \\\hline
 dc:identifier  &  & type,scheme      &  ANY  \\\hline
 dc:source      &  & type, scheme       &  ANY  \\\hline
 dc:language    &  & type, scheme      &  {\twintoo{ISO}{639}} \\\hline
 dc:relation    &  & type, scheme     &  ANY  \\\hline
 dc:rights      &  & type, scheme      &  ANY  \\\hline\hline
 \multicolumn{4}{|l|}{for \llquote{math vernacular} see {\mysecref{mtext}}}\\\hline
\end{tabular}
\end{scriptsize}
\end{myfig}


The descriptions in this section are adapted from~\cite{DCMI:dmt03}, and augmented for the
application in {\omdoc} where necessary. All these elements live in the {\twintoo{Dublin
    Core}{namespace}} \url{http://purl.org/dc/elements/1.1/}, for which we traditionally
use the {\twintoo{namespace}{prefix}} {\snippetin{dc:}}.\atwin{Dublin
  Core}{namespace}{URI}

\begin{description}
\item[{\element[ns-elt=dc]{title}}] The title of the element --- note that {\omdoc}
  metadata can be specified at multiple levels, not only at the document level, in
  particular, the Dublin Core {\eldef[ns-elt=dc]{title}} element can be given to assign a
  title to a theorem, e.g. the ``Substitution Value Theorem''.
  
  The {\element[ns-elt=dc]{title}} element can contain
  {\twintoo{mathematical}{vernacular}}, i.e. the same content as the {\element{CMP}}
  defined in {\mysecref{mtext}}. Also like the {\element{CMP}} element, the
  {\element[ns-elt=dc]{title}} element has an
  {\attribute[ns-elt=xml,ns-attr=dc]{lang}{title}} attribute that specifies the language
  of the content. Multiple {\element[ns-elt=dc]{title}} elements inside a
  {\element{metadata}} element are assumed to be translations of each other.
\item[{\element[ns-elt=dc]{creator}}] A primary creator or author of the publication.
  Additional contributors whose contributions are secondary to those listed in
  {\eldef[ns-elt=dc]{creator}} elements should be named in
  {\element[ns-elt=dc]{contributor}} elements.  Documents with multiple co-authors should
  provide multiple {\element[ns-elt=dc]{creator}} elements, each containing one author.
  The order of {\element[ns-elt=dc]{creator}} elements is presumed to define the order in
  which the creators' names should be presented.
  
  As markup for names across cultures is still un-standardized, {\omdoc} recommends that
  the content of a {\element[ns-elt=dc]{creator}} element consists in a single name (as it
  would be presented to the user). The {\element[ns-elt=dc]{creator}} element has an
  optional attribute {\attribute[ns-elt=xml,ns-attr=dc]{id}{creator}} so that it can be
  {\indextoo{cross-reference}d} and a {\attributeshort{role}} attribute to further
  classify the concrete contribution to the element. We will discuss its values in
  {\mysecref{dc-roles}}.
\item[{\element[ns-elt=dc]{contributor}}] A party whose contribution to the publication is
  secondary to those named in {\element[ns-elt=dc]{creator}} elements.  Apart from the
  significance of contribution, the semantics of the {\eldef[ns-elt=dc]{contributor}} is
  identical to that of {\element[ns-elt=dc]{creator}}, it has the same restriction content
  and carries the same attributes plus a
  {\attribute[ns-elt=xml,ns-attr=dc]{lang}{contributor}} attribute that specifies the
  target language in case the contribution is a translation.
\item[{\element[ns-elt=dc]{subject}}] This element contains an arbitrary phrase or keyword,
  the attribute {\attribute[ns-elt=xml,ns-attr=dc]{lang}{subject}} is used for the
  language. Multiple instances of the {\eldef[ns-elt=dc]{subject}} element are supported
  per {\attribute[ns-elt=xml,ns-attr=dc]{lang}{subject}} for multiple keywords.
\item[{\element[ns-elt=dc]{description}}] A text describing the containing element's
  content; the attribute {\attribute[ns-elt=xml,ns-attr=dc]{lang}{description}} is used
  for the language. As description of mathematical objects or {\omdoc} fragments may
  contain formulae, the content of this element is of the form
  ``{\twintoo{mathematical}{text}}'' described in {\mychapref{mtxt}}.  The
  {\eldef[ns-elt=dc]{description}} element is only recommended for {\element{omdoc}}
  elements that do not have a {\element{CMP}} group (see {\mysecref{mtext}}), or if the
  description is significantly shorter than the one in the {\element{CMP}s} (then it can
  be used as an {\indextoo{abstract}}).
\item[{\element[ns-elt=dc]{publisher}}] The entity for making the document available in
  its present form, such as a publishing house, a university department, or a corporate
  entity. The {\eldef[ns-elt=dc]{publisher}} element only applies if the
  {\element{metadata}} is a direct child of the root element ({\element{omdoc}}) of a
  document\twin{document}{root}.
\item[{\element[ns-elt=dc]{date}}] The date and time a certain action was performed on the
  element that contains this. The content is in the format defined by {\xml} Schema data
  type {\snippetin{date\-Time}} (see~\cite{BirMal:XMLSchema:Datatypes} for a discussion),
  which is based on the {\atwintoo{ISO}{8601}{norm}} for dates and times.

  Concretely, the format is
  {\snippet{\llquote{YYYY}-\llquote{MM}-\llquote{DD}T\llquote{hh}:\llquote{mm}:\llquote{ss}}}
  where {\llquote{YYYY}} represents the year, {\llquote{MM}} the month, and {\llquote{DD}}
  the day, preceded by an optional leading ``{\snippet{-}}'' sign to indicate a negative
  number. If the sign is omitted, ``{\snippet{+}}'' is assumed.  The letter
  ``{\snippet{T}}'' is the date/time separator and {\llquote{hh}}, {\llquote{mm}},
  {\llquote{ss}} represent hour, minutes, and seconds respectively.  Additional digits can
  be used to increase the precision of fractional seconds if desired, i.e the format
  {\snippet{\llquote{ss}.\llquote{sss\ldots}}} with any number of digits after the decimal
  point is supported.  The {\eldef[ns-elt=dc]{date}} element has the attributes
  {\attribute[ns-elt=dc]{action}{date}} and {\attribute[ns-elt=dc]{who}{date}} to specify
  who did what: The value of {\attribute[ns-elt=dc]{who}{date}} is a reference to a
  {\element[ns-elt=dc]{creator}} or {\element[ns-elt=dc]{contributor}} element and
  {\attribute[ns-elt=dc]{action}{date}} is a keyword for the action
  undertaken. Recommended values include the short forms
  {\attval[ns-elt=dc]{updated}{action}{date}},
  {\attval[ns-elt=dc]{created}{action}{date}},
  {\attval[ns-elt=dc]{imported}{action}{date}},
  {\attval[ns-elt=dc]{frozen}{action}{date}},
  {\attval[ns-elt=dc]{review-on}{action}{date}},
  {\attval[ns-elt=dc]{normed}{action}{date}} with the obvious meanings. Other actions may
  be specified by {\indextoo{URI}s} pointing to documents that explain the action.
\item[{\element[ns-elt=dc]{type}}] Dublin Core defines a vocabulary for the document types
  in {\cite{DCMI:dtv03}}. The best fit values for {\omdoc} are
  \begin{description}
  \item[{\snippetin{Dataset}}]\index{Dataset@{\snippet{Dataset}} as Dublin Core Type}
    defined as ``{\emph{information encoded in a defined structure (for example lists,
      tables, and databases), intended to be useful for direct machine processing}}.''
  \item[{\snippetin{Text}}]\index{Text@{\snippet{Text}} as Dublin Core Type} defined as
    ``{\emph{a resource whose content is primarily words for reading. For example -- books,
      letters, dissertations, poems, newspapers, articles, archives of mailing lists. Note
      that facsimiles or images of texts are still of the genre text.}}''
  \item[{\snippetin{Collection}}]\index{Collection@{\snippet{Collection} as Dublin Core
        Type}} defined as ``{\emph{an aggregation of items. The term collection means that
      the resource is described as a group; its parts may be separately described and
      navigated}}''.
  \end{description}
  The more appropriate should be selected for the element that contains the
  {\eldef[ns-elt=dc]{type}}. If it consists mainly of formal mathematical formulae, then
  {\snippetin{Dataset}} is better, if it is mainly given as text, then {\snippetin{Text}}
  should be used. More specifically, in {\omdoc} the value {\snippetin{Dataset}} signals
  that the order of children in the parent of the {\element{metadata}} is not relevant to
  the meaning. This is the case for instance in formal developments of mathematical
  theories, such as the specifications in {\mychapref{complex-theories}}.
\item[{\element[ns-elt=dc]{format}}] The physical or digital manifestation of the
  resource.  Dublin Core suggests using {\twintoo{MIME}{type}s}~\cite{FreBor:MIME96}.
  Following~\cite{MurLau:xmt01} we fix the content of the {\eldef[ns-elt=dc]{format}}
  element to be the string {\snippet{application/omdoc+xml}} as the {\twintoo{MIME}{type}}
  for {\omdoc}.
\item[{\element[ns-elt=dc]{identifier}}] A string or number used to uniquely identify the
  element.  The {\eldef[ns-elt=dc]{identifier}} element should only be used
  for public identifiers like {\indextoo{ISBN}} or {\indextoo{ISSN}} numbers. The
  numbering scheme can be specified in the {\attribute[ns-elt=dc]{scheme}{identifier}}
  attribute.
\item[{\element[ns-elt=dc]{source}}] Information regarding a prior resource from which the
  publication was derived. We recommend using either a {\indextoo{URI}} or a scientific
  reference including identifiers like ISBN numbers for the content of the
  {\eldef[ns-elt=dc]{source}} element.
\item[{\element[ns-elt=dc]{relation}}] Relation of this document to others.  The content
  model of the {\eldef[ns-elt=dc]{relation}} element is not specified in the {\omdoc}
  format.
\item[{\element[ns-elt=dc]{language}}] If there is a primary language of the document or
  element, this can be specified here. The content of the {\eldef[ns-elt=dc]{language}}
  element must be an {\atwintoo{ISO}{639}{norm}} two-letter language specifier, like
  {\snippetin{en}}$\;\widehat=\;$English, {\snippetin{de}}$\;\widehat=\;$German,
  {\snippetin{fr}}$\;\widehat=\;$French, {\snippetin{nl}}$\;\widehat=\;$Dutch, \ldots.
\item[{\element[ns-elt=dc]{rights}}] Information about rights held in and over the
  document or element content or a reference to such a statement. Typically, a
  {\eldef[ns-elt=dc]{rights}} element will contain a rights management statement, or
  reference a service providing such information. {\element[ns-elt=dc]{rights}}
  information often encompasses Intellectual Property rights (IPR), Copyright, and various
  other property rights. If the {\element[ns-elt=dc]{rights}} element is absent (and no
  {\element[ns-elt=dc]{rights}} information is inherited), no assumptions can be made
  about the status of these and other rights with respect to the document or element.
  
  {\omdoc} supplies specialized elements for the Creative Commons licenses to support the
  sharing of mathematical content. We will discuss them in {\mysecref{creativecommons}}.
\end{description}
Note that Dublin Core also defines a {\oldelement{Coverage}{1.1}} element that specifies
the place or time which the publication's contents addresses. This does not seem
appropriate for the mathematical content of {\omdoc}, which is largely independent of time
and geography.  \ednote{MK: all of them have a type and scheme attribute that can be
  filled e.g. with \url{http://dimes.lins.fju.edu.tw/dimes/meta-ref/DC-SubElements.html}}

\end{tsection}

\begin{tsection}[id=dc-roles]{Roles in Dublin Core Elements}

Because the Dublin Core metadata fields for {\element[ns-elt=dc]{creator}} and
{\element[ns-elt=dc]{contributor}} do not distinguish roles of specific parties (such as
author, editor, and illustrator), we will follow the {\indextoo{Open eBook}}
specification~\cite{OpenEBook:oeps99} and use an optional
{\attribute[ns-elt=dc]{role}{*}} attribute for this purpose, which is
adapted for {\omdoc} from the MARC relator code list~\cite{Marc:relators03}.
\begin{description}
\item[{\attval[ns-elt=dc]{aut}{role}{*}}] ({\indextoo{author}}) Use for a
  person or corporate body chiefly responsible for the intellectual
  content of an element. This term may also be used when more than one person or body
  bears such responsibility.
\item[{\attval[ns-elt=dc]{ant}{role}{*}}] (scientific/bibliographic
  antecedent\twin{bibliographic}{antecedent}\twin{scientific}{antecedent}) Use
for the author responsible for a work upon which the element is based.
\item[{\attval[ns-elt=dc]{clb}{role}{*}}] ({\indextoo{collaborator}}) Use
  for a person or corporate body that takes a limited part in the elaboration of a
  work of another author or that brings complements (e.g., appendices, notes) to
  the work of another author.
\item[{\attval[ns-elt=dc]{edt}{role}{*}}] ({\indextoo{editor}}) Use for a
  person who prepares a document not primarily his/her own for publication, such
  as by elucidating text, adding introductory or other critical matter, or
  technically directing an editorial staff.
\item[{\attval[ns-elt=dc]{ths}{role}{*}}] ({\twintoo{thesis}{advisor}}) Use for the person under
  whose supervision a degree candidate develops and presents a thesis, memoir, or text of
  a dissertation.
\item[{\attval[ns-elt=dc]{trc}{role}{*}}] ({\indextoo{transcriber}}) Use
  for a person who prepares a handwritten or typewritten copy from original
  material, including from dictated or orally recorded material. This is also the
  role (on the {\element[ns-elt=dc]{creator}} element) for someone who prepares the {\omdoc}
  version of some mathematical content.
\item[{\attval[ns-elt=dc]{trl}{role}{*}}] ({\indextoo{translator}}) Use
  for a person who renders a text from one language into another, or from an older
  form of a language into the modern form. The target language can be specified by
  {\attribute[ns-elt=xml,ns-attr=dc]{lang}{*}}.
\end{description}
As {\omdoc} documents are often used to formalize existing mathematical texts for use in
mechanized reasoning and computation systems, it is sometimes subtle to specify
authorship.  We will discuss some typical examples to give a guiding intuition.
{\mylstref{sec-edt}} shows metadata for a situation where editor $R$ gives the sources
(e.g. in {\LaTeX}) of an element written by author $A$ to secretary $S$ for conversion
into {\omdoc} format.
\begin{lstlisting}[label=lst:sec-edt,mathescape,
  caption={A Document with Editor ({\snippet{edt}}) and  Transcriber ({\snippet{trc}})},
  index={metadata,dc:title,dc:creator,dc:contributor}]
<metadata>
  <dc:title>The Joy of Jordan $C\sp{*}$ Triples</dc:title>
  <dc:creator role="aut">$A$</dc:creator>
  <dc:contributor role="edt">$R$</dc:contributor>
  <dc:contributor role="trc">$S$</dc:contributor>
</metadata>
\end{lstlisting}

In {\mylstref{formalize}} researcher $R$ formalizes the theory of natural numbers
following the standard textbook $B$ (written by author $A$). In this case we
recommend the first declaration for the whole document and the second one for
specific math elements, e.g. a definition inspired by or adapted from one in book
$B$.

\begin{lstlisting}[label=lst:formalize,mathescape,
  caption={A Formalization with Scientific Antecedent ({\snippet{ant}})},
  index={metadata,dc:title,dc:creator}]
<omdoc xml:id="NNat" version="1.3" xmlns:dc="http://purl.org/dc/elements/1.1/">
  <metadata><dc:title>Natural Numbers</dc:title></metadata>                              
  $\ldots$
  <theory xml:id="NNat.thy">
    <metadata>
      <dc:title>Natural Numbers</dc:title>
      <dc:creator role="aut">$R$</dc:creator>
      <dc:contributor role="ant">$A$</dc:contributor>
      <dc:source>$B$</dc:source>
    </metadata>
    $\ldots$
  </theory>
  $\ldots$
</omdoc>
\end{lstlisting}
\end{tsection}

\begin{tsection}[id=creativecommons,short=Managing Rights]{Managing Rights by Creative
    Commons Licenses (Module {\CCmodule{spec}})}

  The Dublin Core vocabulary provides the {\element[ns-elt=dc]{rights}} element for
  information about rights held in and over the document or element content, but leaves
  the content model unspecified. While it is legally sufficient to describe this
  information in natural language, a content markup format like {\omdoc} should support a
  machine-understandable format. As one of the purposes of the {\omdoc} format is to
  support the sharing and re-use of mathematical content, {\omdoc} provides markup for
  rights management via the {Creative Commons\twin{Creative Commons}{license}}
  (CC{\twin{CC}{license}}) licenses.  Digital rights
  management\atwin{Digital}{rights}{management} (\indextoo{DRM}) and licensing of
  {\twintoo{intellectual}{property}} has become a hotly debated topic in the last
  years. We feel that the {\twintoo{Creative Commons}{license}s} that encourage sharing of
  content and enhance the (scientific) public domain while giving authors some control
  over their intellectual property establish a good middle ground. Specifying rights is
  important, since in the absence of an explicit or implicit (via inheritance)
  {\element[ns-elt=dc]{rights}} element no assumptions can be made about the status of the
  document or fragment.  Therefore {\omdoc} adds another child to the {\element{metadata}}
  element.  This {\eldef[ns-elt=cc]{license}} element is a symbolic representation of the
  Creative Commons legal framework, adapted to the {\omdoc} setting: The Creative Commons
  Metadata Initiative specifies various ways of embedding {\twintoo{CC}{metadata}} into
  documents and {\twintoo{electronic}{artefacts}} like {\indextoo{picture}s} or
  {\twintoo{MP3}{recording}s}. As {\omdoc} is a source format, from which various
  presentation formats are generated, we need a content representation of the CC metadata
  from which the end-user representations for the respective formats can be generated.

\begin{myfig}{cctable}{The {\omdoc} Elements for Creative Commons Metadata}
\begin{scriptsize}
\begin{tabular}{|>{\tt}l|>{\tt}l|>{\tt}p{2.2truecm}|>{\tt}l|}\hline
{\rm Element}& \multicolumn{2}{l|}{Attributes\hspace*{2.25cm}} & Content  \\\hline
             & {\rm Req.}  & {\rm Optional}    &          \\\hline\hline
 cc:license      & & jurisdiction    &  permissions, prohibitions, requirements  \\\hline
 cc:permissions  & & reproduction, distribution, derivative\_works & EMPTY\\\hline
 cc:prohibitions & & commercial\_use & EMPTY \\\hline
 cc:requirements & & notice, copyleft,  attribution & EMPTY \\\hline
\end{tabular}
\end{scriptsize}
\end{myfig}

The Creative Commons Metadata Initiative~\cite{creative-commons:on} divides the license
characteristics in three types: {\defin{permissions}}, {\defin{prohibitions}} and
{\defin{requirements}}, which are represented by the three elements, which can occur as
children of the {\element[ns-elt=cc]{license}} element. The {\element[ns-elt=cc]{license}}
element has two optional argument:
\begin{description}
\item[{\attribute[ns-elt=cc]{jurisdiction}{license}}] which allows to specify the country
  in whose jurisdiction the license will be enforced\footnote{The {\twintoo{Creative
        Commons}{Initiative}} is currently in the process of adapting their licenses to
    jurisdictions other than the USA, where the licenses
    originated. See~\cite{URL:creativecommonsWorldwide} for details and to check for
    progress.}. It's value is one of the {\twintoo{top-level}{domain}} codes of the
  ``Internet Assigned Names Authority (IANA)''~\cite{IANA:TLD}. If this attribute is
  absent, then the original US version of the license is assumed.
\item[{\attribute[ns-elt=cc]{version}{license}}] which allows to specify the version of the
  license. If the attribute is not present, then the newest released version is assumed
  (version 2.0 at the time of writing this {\report})
\end{description}

The following three empty elements can occur as children of the
{\element[ns-elt=cc]{license}} element; their attribute specify the rights bestowed on the
user by the license.  All these elements have the {\twin{Creative Commons}{namespace}}
namespace \url{http://creativecommons.org/ns}\atwin{Creative Commons}{namespace}{URI},
for which we traditionally use the {\twintoo{namespace}{prefix}} {\snippetin{cc:}}.

\begin{itemize}
\item {\eldef[ns-elt=cc]{permissions}} are the rights granted by the license, to model
  them the element has three attributes, which can have the values
  {\attvalveryshort{permitted}} (the permission is granted by the license) and
  {\attvalveryshort{prohibited}} (the permission isn't):
  \begin{center}\scriptsize
    \begin{tabular}{|l|p{6truecm}|>{\tt}l|}\hline
      Attribute & Permission & Default\\\hline\hline
      {\attribute[ns-elt=cc]{reproduction}{permissions}} 
      & the work may be reproduced & permitted\\\hline
      {\attribute[ns-elt=cc]{distribution}{permissions}}  
      & the work may be distributed, publicly displayed, and
      publicly performed & permitted \\\hline
      {\attribute[ns-elt=cc]{derivative\_works}{permissions}}  
      & derivative works may be created and reproduced & permitted \\\hline
    \end{tabular}
  \end{center}
\item {\eldef[ns-elt=cc]{prohibitions}} are the things the license prohibits.
  \begin{center}\scriptsize
    \begin{tabular}{|l|p{6truecm}|>{\tt}l|}\hline
      Attribute & Prohibition & Default\\\hline\hline
      {\attribute[ns-elt=cc]{commercial\_use}{permission}} 
      &  stating that rights may be exercised for commercial purposes.
      & permitted \\\hline
    \end{tabular}
  \end{center}
\item {\eldef[ns-elt=cc]{requirements}} are restrictions imposed by the license.
    \begin{center}\scriptsize
      \begin{tabular}{|l|p{6.5truecm}|>{\tt}l|}\hline
      Attribute & Requirement & Default\\\hline\hline
      {\attribute[ns-elt=cc]{notice}{requirements}}  
      & copyright and license notices must be kept intact & required \\\hline
      {\attribute[ns-elt=cc]{attribution}{requirements}}  
      & credit must be given to copyright holder and/or author & required\\\hline
      {\attribute[ns-elt=cc]{copyleft}{requirements}}  
      & derivative works, if authorized, must be licensed under the same terms as
      the work & required \\\hline
    \end{tabular}
  \end{center}
\end{itemize}

This vocabulary is directly modeled after the Creative Commons
Metadata~\cite{URL:creativecommonsMetadata} which defines the meaning, and provides an
{\rdf}~\cite{LasSwi:rdf99} based implementation. As we have discussed in
{\mysecref{metadata}}, {\omdoc} follows an approach that specifies metadata in the
document itself; thus we have provided the elements described here. In contrast to many
other situations in {\omdoc}, the rights model is not extensible, since only the current
model is backed by legal licenses provided by the creative commons initiative.

{\Mylstref{ccc-copyleft}} specifies a license grant using the Creative Commons
``share-alike'' license: The copyright is retained by the author, who licenses the content
to the world, allowing others to reproduce and distribute it without restrictions as long
as the copyright notice is kept intact. Furthermore, it allows others to create derivative
works based on the content as long as it attributes the original work of the author and
licenses the derived work under the identical license (i.e. the Creative Commons
``share-alike'' as well).
\begin{lstlisting}[label=lst:ccc-copyleft,caption={A Creative Commons License},
  index={metadata,dc:rights,license,permissions,reproduction,distribution,
         derivative_works,prohibitions,commercial_use,requirements,
         notice,copyleft,attribution}]
<metadata>
  <dc:rights>Copyright (c) 2004 Michael Kohlhase</dc:rights>
  <license jurisdiction="de" xmlns="http://creativecommons.org/ns">
    <permissions reproduction="permitted" distribution="permitted" 
                 derivative_works="permitted"/>
    <prohibitions commercial_use="permitted"/>
    <requirements notice="required" copyleft="required" attribution="required"/>
  </license>
</metadata>
\end{lstlisting}
\end{tsection}

\end{tchapter}

%%% Local Variables: 
%%% mode: latex
%%% TeX-master: "omdoc"
%%% End: 

% LocalWords:  DCperson trl dateTime CC YY DD hh ss sss ISBN ISSN isbn IPR dc
% LocalWords:  MiKo aut clb edt ths trc lst sec qtmetadata lang mathescape Req
% LocalWords:  mtext camelcase natlist en fr nl creativecommons DRM genmeta mmt
% LocalWords:  cctable comercial RDF CNX NNat mtxt ref xmlns inheritancea ns un
% LocalWords:  inheritanceb elt attr CMP Dataset omdoc metadata YYYY de eBook
% LocalWords:  creativecommons dc CC DRM cc cctable Req IANA RDF ccc lst xmlns
% LocalWords:  th mbox xncmd ccrel rdfs wrt ifmmode mathit textit rdfa rdfa svg
% LocalWords:  enquote xhtml xhtml xhtml TumDelOre Sindice07 oo localname
% LocalWords:  PruSea08 sparql bnode-id
