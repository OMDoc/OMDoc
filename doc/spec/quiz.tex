%%%%%%%%%%%%%%%%%%%%%%%%%%%%%%%%%%%%%%%%%%%%%%%%%%%%%%%%%%%%%%%%%%%%%%%%%
% This file is part of the LaTeX sources of the OMDoc 1.6 specification
% Copyright (c) 2006 Michael Kohlhase
% This work is licensed by the Creative Commons Share-Alike license
% see http://creativecommons.org/licenses/by-sa/2.5/ for details
\svnInfo $Id: quiz.tex 8427 2009-07-19 10:01:23Z kohlhase $
\svnKeyword $HeadURL: https://svn.omdoc.org/repos/omdoc/trunk/doc/spec/quiz.tex $
%%%%%%%%%%%%%%%%%%%%%%%%%%%%%%%%%%%%%%%%%%%%%%%%%%%%%%%%%%%%%%%%%%%%%%%%%

\begin{module}[id=quiz]
\begin{omgroup}[id=quiz,short=Exercises]{Exercises (Module {\QUIZmodule{spec}})}

Exercises and study problems are vital parts of mathematical documents like textbooks or
exams, in particular, mathematical exercises contain mathematical vernacular and pose the
same requirements on context like mathematical statements. Therefore markup for exercises
has to be tightly integrated into the document format, so {\omdoc} provides a module for
them.

Note that the functionality provided in this module is very limited, and largely serves as
a place-holder for more pedagogically informed developments in the future (see
{\extref{projects}{activemath}} and~\cite{GogMelUllCai:psmmee03} for an example in the {\omdoc}
framework).

\begin{presonly}
\begin{myfig}{qtex}{The {\omdoc} Auxiliary Elements for Exercises}
\begin{scriptsize}
\begin{tabular}{|>{\tt}l|>{\tt}l|>{\tt}p{3.3truecm}|c|>{\tt}p{4truecm}|}\hline
{\rm Element}& \multicolumn{2}{l|}{Attributes\hspace*{2.25cm}} & M & Content  \\\hline
             & {\rm Req.} & {\rm Optional}                     & D &           \\\hline\hline
 exercise    &            & xml:id, class, style       & +  & h:p*,hint?,(solution*|mc*)\\\hline
 hint        &            & xml:id, class, style       & +  & h:p* \\\hline
 solution    &            & xml:id, for, class, style  & +  & \llquote{top-level element} \\\hline
 mc          &            & xml:id, for, class, style  & -- & choice, hint?, answer\\\hline
 choice      &            & xml:id, class, style       & +  & h:p*    \\\hline
 answer      & verdict    & xml:id, class, style       & +  & h:p*      \\\hline
\end{tabular}
\end{scriptsize}
\end{myfig}
\end{presonly}

\begin{definition}[id=exercise.def]
  The {\QUIZmodule{spec}} module provides the top-level elements {\eldef{exercise}},
  {\element{hint}}, and {\element{solution}}. The first one is used for
  {\indextoo{exercise}s} and {\indextoo{assessment}s}. The question statement is
  represented as mathematical vernacular in a sequence of {\element[ns-elt=h]{p}}
  elements. This information can be augmented by {\indextoo{hint}s}
  (using the {\element{hint}} element) and a {\indextoo{solution}}/{\indextoo{assessment}}
  block (using the {\element{solution}} and {\element{mc}} elements).
\end{definition}

The {\element{hint}} and {\element{solution}} elements can occur as children of
{\element{exercise}}; or outside, referencing it in their optional
{\attribute{for}{solution}} attribute. This allows a flexible positioning of the hints and
solutions, e.g. in separate documents that can be distributed separately from the
{\element{exercise}} elements. 

\begin{definition}[id=hint.def]
  The {\eldef{hint}} element contains mathematical vernacular as a sequence of
  {\element[ns-elt=h]{p}} elements for the hint text. The {\eldef{solution}} element can contain any
  number of {\omdoc} top-level elements to explain and justify the solution.  This is the
  case, where the question contains an assertion whose proof is not displayed and left to
  the reader. Here, the {\element{solution}} contains a proof.
\end{definition}

\begin{lstlisting}[label=lst:texbook-exercise,escapechar=\%,
  caption={An Exercise from the {\TeX}Book},
  index={exercise,hint,solution}]
<exercise xml:id="TeXBook-18-22">
  <h:p>Sometimes the condition that defines a set is given as a fairly long
    English description; for example consider `{p|p and p+2 are prime}'. An 
    hbox would do the job:</h:p>

  <h:p style="display:block;font-family:fixed">
    $\{\,p\mid\hbox{$p$ and $p+2$ are prime}\,\}$
  </h:p>

  <h:p>but a long formula like this is troublesome in a paragraph, since an hbox cannot
    be broken between lines, and since the glue inside the 
    <h:span style="font-family:fixed">\hbox</h:span> does not vary with the inter-word
    glue in the line that contains it. Explain how the given formula could be
    typeset with line breaks.</p>
  </h:p>
  <hint>
    <h:p>Go back and forth between math mode and horizontal mode.</h:p>
  </hint>
  <solution>
    <h:p>
      <h:span style="font-family:fixed">
       $\{\,p\mid p$~and $p+2$ are prime$\,\}$
      </h:span>,
      assuming that <h:span style="font-family:fixed">\mathsurround</h:span> is
      zero. The more difficult alternative '<h:span style="font-family:fixed">
      $\{\,p\mid p\\ {\rm and}\ p+2\rm\ are\ prime\,\}$</h:span>'
      is not a solution, because line breaks do not occur at 
      <h:span style="font-family:fixed">\%\textvisiblespace%</h:span> (or at glue of any
      kin) within math formulas. Of course it may be best to display a formula like
      this, instead of breaking it between lines.
    </h:p>
  </solution>
</exercise>
\end{lstlisting}

 {\indextoo{Multiple-choice exercise}s} (see {\mylstref{exercise}}) are represented by a
  group of {\element{mc}} elements inside an {\element{exercise}} element.

\begin{definition}[id=mc.def] 
  An {\eldef{mc}} element represents a single choice in a multiple choice element. It
  contains the elements below (in this order).
\begin{description}
\item[{\element{choice}}] for the description of the choice (the text the user gets to see
  and is asked to make a decision on). The {\eldef{choice}} element carries the
  {\attributeshort[ns-attr=xml]{id}}, {\attributeshort{style}}, and
  {\attributeshort{class}} attributes and contains mathematical vernacular in a sequence
  of {\element[ns-elt=h]{p}} elements.
\item[{\element{hint}}] (optional) for a hint to the user, see above for a description.
\item[{\element{answer}}] for the feedback to the user. This can be the correct answer, or
  some other feedback (e.g. another hint, without revealing the correct answer).  The
  {\attribute{verdict}{answer}} attribute specifies the truth of the answer, it can have
  the values {\attval{true}{verdict}{answer}} or {\attval{false}{verdict}{answer}}. This
  element is required, inside a {\element{mc}}, since the {\attribute{verdict}{answer}} is
  needed. It can be empty if no feedback is available. Furthermore, the {\eldef{answer}}
  element carries the {\attributeshort[ns-attr=xml]{id}}, {\attributeshort{style}}, and
  {\attributeshort{class}} attributes and contains mathematical vernacular as a sequence
  of {\element[ns-elt=h]{p}} elements.
\end{description}
\end{definition}

\begin{lstlisting}[label=lst:exercise,mathescape,
  caption={A Multiple-Choice Exercise in {\omdoc}},
  index={exercise,mc,choice,answer}]
<exercise for="#ida.c6s1p4.l1" xml:id="ida.c6s1p4.mc1">
  <h:p>
    What is the unit element of the semi-group $Q$ with operation $a*b = 3ab$?
  </h:p> 
  <mc>
    <choice><h:p><OMI>1</OMI></h:p></choice>
    <answer verdict="false"><h:p>No, $1*1=3$ and not 1</h:p></answer>
  </mc>
  <mc>
    <choice><h:p>1/3</h:p></choice>
    <answer verdict="true"></answer>
  </mc>
  <mc>
    <choice><h:p>It has no unit.</h:p></choice>
    <answer verdict="false"><h:p>No, try another answer</h:p></answer>
  </mc>
</exercise>
\end{lstlisting}
\end{omgroup}
\end{module}
%%% Local Variables: 
%%% mode: latex
%%% TeX-master: "main"
%%% End: 

% LocalWords:  qtex mc proofobject TeXBook hbox lst texbook truecm OMI
% LocalWords:  escapechar omdoc ns attr Req multi activemath
