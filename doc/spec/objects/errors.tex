\begin{omgroup}[id=om.error]{Reporting Errors in  Mathematical Objects}
\begin{definition}[id=ome.def]
  The {\eldef[ns-elt=om]{OME}} element is used for {\atwintoo{in-place}{error}{markup}} in
  {\openmath} objects, it can be used almost everywhere in {\openmath} elements. It has
  two children; the first one is an {\twintoo{error}{operator}}\footnote{An error operator
    is like a {\twintoo{binding}{operator}}, only the symbol has role
    \attval{error}{role}{symbol}.}, i.e. an {\openmath} symbol that specifies the kind
  of error, and the second one is the faulty {\openmath} object fragment. Note that since
  the whole object must be a valid {\openmath} object, the second child must be a
  well-formed {\openmath} object fragment.
\end{definition}
As a consequence, the \element[ns-elt=om]{OME} element can only be used for
``{\twintoo{semantic}{error}s}'' like non-existing content dictionaries, out-of-bounds
errors, etc.  {\xml}-well-formedness and DTD-validity errors will have to be handled by
the {\xml} tools involved. In the following example, we have marked up two errors in a
faulty representation of $\sin(\pi)$.  The outer error flags an arity violation (the
function $\sin$ only allows one argument), and the inner one flags the typo in the
representation of the constant $\pi$ (we used the name {\snippet{po}} instead of
{\snippet{pi}}).

\begin{lstlisting}[label=ome,language=OpenMath,numbers=none,index={OME}]
<OME>
  <OMS cd="type-error" name="arity-violation"/>
  <OMA>
    <OMS cd="transc1" name="sin"/>
    <OME>
      <OMS cd="error" name="unexpected_symbol"/>
      <OMS cd="nums1" name="po"/>
    </OME>
    <OMV name="x"/>
  </OMA>
</OME>
\end{lstlisting}
  As we can see in this example, errors can be nested to encode multiple faults found by
  an {\openmath} application.

  \ednote{need to talk about the \element[ns-elt=m]{cerror} element}
\end{omgroup}

%%% Local Variables:
%%% mode: latex
%%% TeX-master: t
%%% End:
