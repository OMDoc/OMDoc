\begin{omgroup}[id=constitutive-statements]{Theory-Constitutive Statements in OMDoc}
\begin{module}[id=constitutive-statements]

  The \omdoc format provides an infrastructure for four kinds of theory-constitutive
  statements: symbol declarations, type declarations, (proper) axioms, and definitions. We
  will take a look at all of them now.

\begin{presonly}
  \begin{myfig}{constitutive-theory}{Theory-Constitutive Elements in \omdoc}
    \begin{scriptsize}
      \begin{tabular}{|>{\snippet}l|>{\tt}l|>{\tt}p{3.6truecm}|c|>{\tt}p{3.2truecm}|}\hline
        {\rm Element}& \multicolumn{2}{l|}{Attributes\hspace*{2.25cm}} & M & Content  \\\hline
                     & {\rm Required}  & {\rm Optional}                & C &          \\\hline\hline
          symbol     & name      & xml:id, role, scope, style, class   & +  & type*\\\hline
          axiom      & name      & xml:id, for, type, style, class     & +  & h:*,FMP*   \\\hline
          definition & for & xml:id, type, style, class                & +  & h:p*, \llquote{mobj})?  \\\hline
 \multicolumn{5}{|l|}{where \llquote{mobj} is {\tt{(\mobjabbr)}}}\\\hline
\end{tabular}
\end{scriptsize}
\end{myfig}
\end{presonly}

\begin{omgroup}[id=symbol-dec]{Symbol Declarations}

  The {\element{symbol}} element declares a symbol for a
  {\twintoo{mathematical}{concept}}, such as 1 for the natural number ``one'', $+$ for
  addition, $=$ for equality, or {\snippetin{group}} for the property of being a
  group. Note that we not only use the \element{symbol} element for mathematical objects
  that are usually written with mathematical symbols, but also for any concept or object
  that has a definition or is restricted in its meaning by axioms.
  
We will refer to the mathematical object declared by a \element{symbol} element as a
``{\indextoo{symbol}}'', iff it is usually communicated by specialized notation in
mathematical practice, and as a ``{\indextoo{concept}}'' otherwise.  The name ``symbol''
of the \element{symbol} element in \omdoc is in accordance with usage in the
philosophical literature (see e.g.~\cite{NewSim:cseisas81}): A {\defemph{symbol}} is a
{\emph{{\twinalt{mental}{physical}{representation}} or
    {\twinalt{physical}{mental}{representation}}}} representation of a concept.  In
particular, a symbol may, but need not be representable by a (set of) {\indextoo{glyph}s}
(symbolic notation).  The definiendum objects in {\myfigref{math-def}} would be considered
as ``symbols'' while the concept of a ``group'' in mathematics would be called a
``concept''.
  
\begin{definition}[id=symbol.def]
  The {\eldef{symbol}} element has a required attribute \attribute{name}{symbol} whose
  value uniquely identifies it in a theory.  Since the value of this attribute will be
  used as an {\openmath} symbol name, it must be an {\xml} name\footnote{This limits the
    characters allowed in a name to a subset of the characters in Unicode 2.0; e.g. the
    colon $\colon$ is not allowed. Note that this is not a problem, since the name is just
    used for identification, and does not necessarily specify how a symbol is presented to
    the human reader. For that, \omdoc provides the notation definition infrastructure
    presented in \sref{pres}.} as defined in {\xml} 1.1~\cite{xml1.1:04}. The optional
  attribute \attribute{scope}{symbol} takes the values \attribute{global}{scope}{symbol}
  and \attval{local}{scope}{symbol}, and allows a simple specification of visibility
  conditions: if the \attribute{scope}{symbol} attribute of a \element{symbol} has
  value \attval{local}{scope}{symbol}, then it is not \twinalt{exported}{export}{symbol}
  outside the theory; The {\oldattribute{scope}{symbol}{1.2}} attribute is deprecated, a
  formalization using the \attribute{hiding}{imports} attribute on the
  \element{imports} element should be used instead.  Finally, the optional attribute
  \attribute{role}{symbol} that can take the values\footnote{The first six values come
    from the {\openmath}2 standard. They are specified in content dictionaries; therefore
    \omdoc also supplies them.}
\begin{description}
\item[\attval{binder}{role}{symbol}] The symbol may appear as a binding symbol of an
  {\twintoo{binding}{object}}, i.e. as the first child of an \element[ns-elt=om]{OMBIND}
  object, or as the first child of an \element[ns-elt=m]{apply} element that has an
  \element[ns-elt=m]{bvar} as a second child.
\item[\attval{attribution}{role}{symbol}] The symbol may be used as key in an
  {\openmath} \element[ns-elt=om]{OMATTR} element, i.e. as the first element of a
  key-value pair, or in an equivalent context (for example to refer to the value of an
  attribution).  This form of attribution may be ignored by an application, so should be
  used for information which does not change the meaning of the attributed {\openmath}
  object.
\item[\attval{semantic-attribution}{role}{symbol}] This is the same as
  {\attvalveryshort{attribution}} except that it modifies the meaning of the attributed
  {\openmath} object and thus cannot be ignored by an application.
\item[\attval{error}{role}{symbol}] The symbol can only appear as the first child of an
  {\openmath} error object.
\item[\attval{application}{role}{symbol}] The symbol may appear as the first child of an
  application object.
\item[\attval{constant}{role}{symbol}] The symbol cannot be used to construct a compound
  object.
\item[\attval{type}{role}{symbol}] The symbol denotes a sets that is used in a type
  systems to annotate mathematical objects.
\item[\attval{sort}{role}{symbol}] The symbol is used for a set that are inductively
  built up from constructor symbols; see \sref{adt}.
\end{description}
If the \attribute{role}{symbol} is not present, the value
\attval{object}{role}{symbol} is assumed.

The children of the \element{symbol} element consist of a
{\twintoo{multi-system}{group}} of \element{type} elements (see \sref{type-axioms} for
a discussion). For this group the order does not matter.  In {\mylstref{symbol}} we have a
symbol declaration for the concept of a {\indextoo{monoid}}.  Keywords or simple phrases
that describes the symbol in mathematical vernacular can be added in the
\element{metadata} child of \element{symbol} as \element[ns-elt=dc]{subject} and
{\element[ns-elt=dc]{description}s}; the latter have the same content model as the
\element{h:p} elements, see the discussion in \sref{mtext}). If the document
containing their parent \element{symbol} element were stored in a data base system, it
could be looked up via these metadata. As a consequence the symbol
\attribute{name}{symbol} need only be used for identification. In particular, it need
not be mnemonic, though it can be, and it need not be language-dependent, since this can
be done by suitable \element[ns-elt=dc]{subject} elements.
\end{definition}

\begin{lstlisting}[label=lst:symbol,mathescape,
  caption={An \omdoc \element{symbol} Declaration},
  index={symbol,type}]
<symbol name="monoid">
  <metadata>
    <dc:subject xml:lang="en">monoid</dc:subject>
    <dc:subject xml:lang="de">Monoid</dc:subject>
    <dc:subject xml:lang="it">monoide</dc:subject>
  </metadata>
  <type system="simply-typed">$set[any]\rightarrow(any\rightarrow any \rightarrow any)\rightarrow any\rightarrow bool$</type>
  <type system="props">
    <OMS cd="arities" name="ternary-relation"/>
  </type>
</symbol>
\end{lstlisting}
\end{omgroup}

\begin{omgroup}[id=axioms]{Axioms}
  
  The relation between the components of a monoid would typically be specified by a set of
  axioms (e.g. stating that the base set is closed under the operation).  For this purpose
  \omdoc uses the \element{axiom} element:

\begin{definition}[id=axiom.def]
  The {\eldef{axiom}} element contains mathematical vernacular as a sequence of
  \element[ns-elt=h]{p} elements a \element{FMP} that expresses this as a logical
  formula.  \element{axiom} elements may have a \attribute{generated-from}{axiom}
  attribute, which points to another \omdoc element (e.g. an \element{adt}, see
  \sref{adt}) which subsumes it, since it is a more succinct representation of the same
  mathematical content. Finally the \element{axiom} element has an optional
  \attribute{for}{axiom} attribute to specify salient semantic objects it uses as a
  {\indextoo{whitespace-separated list}} of {\indextoo{URI}} references to symbols
  declared in the same theory, see {\mylstref{axiom}} for an example. Finally, the
  \element{axiom} element can have an \attribute{type}{axiom} attribute, whose values
  we leave unspecified for the moment.
\end{definition}

\begin{lstlisting}[label=lst:axiom,mathescape,
  caption={An \omdoc \element{axiom}},index={axiom}]
<axiom xml:id="mon.ax" for="monoid">
  <h:p>If $\psom{(M,*)}$ is a semigroup with unit $\psom{e}$, then $\psom{(M,*,e)}$ is a monoid.</h:p>
</axiom>
\end{lstlisting}
\end{omgroup}

\begin{omgroup}[id=type-axioms]{Type Declarations}

\begin{omtext}
  {\indexalt{Types}{type}} (also called {\indextoo{sort}s} in some contexts) are
  representations of certain simple sets that are treated specially in (human or
  mechanical) {\twintoo{reasoning}{process}es}. \inlinedef{A \defii{type}{declaration}
    $e\colon t$ makes the information that a symbol or expression $e$ is in a set
    represented by a type $t$ available to a specified mathematical process}.  For
  instance, we might know that $7$ is a natural number, or that expressions of the form
  $\sum_{i=1}^n a_ix^{i}$ are polynomials, if the $a_i$ are real numbers, and exploit this
  information in mathematical processes like proving, pattern matching, or while choosing
  intuitive notations.  \inlinedef{If a type is declared for an expression that is not a
    symbol, we will speak of a \defii{term}{declaration}}.
\end{omtext}

\begin{definition} [id=type.def]
  \omdoc uses the {\eldef{type}} element for type declarations. The optional attribute
  \attribute{system}{type} contains a {\indextoo{URI}} reference that identifies the
  {\twintoo{type}{system}} which interprets the content. There may be various sources of
  the set membership information conveyed by a type declaration, to justify it this source
  may be specified in the optional \attribute{just-by}{type} attribute. The value of
  this attribute is a {\indextoo{URI}} reference that points to an \element{assertion}
  or \element{axiom} element that asserts $\allcdot{x_1,\ldots,x_n}{e\in t}$ for a type
  declaration $e\colon t$ with variables $x_1,\ldots,x_n$.  If the
  \attribute{just-by}{type} attribute is not present, then the type declaration is
  considered to be generated by an {\twintoo{implicit}{axiom}}, which is considered
  {\indextoo{theory-constitutive}}\footnote{It is considered good practice to make the
    axiom explicit in formal contexts, as this allows an extended automation of the
    knowledge management process.}.
  
  The \element{type} element contains one or two mathematical objects. In the first
  case, it represents a type declaration for a symbol (we call this a
  \defii{symbol}{declaration}), which can be specified in the optional
  \attribute{for}{type} attribute or by embedding the \element{type} element into the
  respective \element{symbol} element.  A \element{type} element with two mathematical
  objects represents a {\twintoo{term}{declaration}} $e\colon t$, where the first object
  represents the expression $e$ and the second one the type $t$ (see
  {\mylstref{term-declaration}} for an example). There the type declaration of
  {\snippet{monoid}} characterizes a monoid as a three-place predicate (taking as
  arguments the base set, the operation, and a neutral element).
\end{definition}    

\begin{omtext}
  As reasoning processes vary, information pertaining to multiple
  {\twintoo{type}{system}s} may be associated with a single symbol and there can be more
  than one \element{type} declaration per expression and {\twintoo{type}{system}}, this
  just means that the object has more than one type in the respective type system (not all
  type systems admit {\twintoo{principal}{type}s}).
\end{omtext}
\end{omgroup}

\begin{omgroup}[id=definitions]{Definitions}

Definitions are a special class {\indextoo{axioms}} that completely fix the meaning of
symbols. 

\begin{definition}[id=definition.def]
  Therefore {\eldef{definition}} elements that represent definitions carry the required
  \attribute{for}{definition} attribute, which contain a whitespace-separated list of
  names of symbols in the same theory.  We call these symbols
  \adefii{defined}{defined}{symbol} and \adefii{primitive}{primitive}{symbol}
  otherwise. A \element{definition} contains mathematical vernacular as a sequen ce of
  \element[ns-elt=h]{p} elements to describe the meaning of the defined symbols.

  A \element{definition} element contains a mathematical object that can be substituted
  for the symbol specified in the \attribute{for}{definition} attribute of the
  definition. The \attribute{type}{definition} is fixed to
  \attval{simple}{type}{definition}\ednote{maybe better leave it out}.  {\Mylstref{one}}
  gives an example of a simple definition of the number one from the successor function
  and zero. \omdoc treats the \attribute{type}{definition} attribute as an optional
  attribute. If it is not given explicitly, it defaults to
  \attval{simple}{type}{definition}.
\end{definition}

\begin{lstlisting}[label=lst:one,
  caption={A Simple \omdoc \element{definition}.},
  index={definition}]
<symbol name="one"/>
<definition xml:id="one.def" for="one" type="simple">
 <h:p><OMS cd="nat" name="one"/> is the successor of <om:OMS cd="nat" name="zero"/>.</h:p>
 <om:OMA>
   <om:OMS cd="int" name="suc"/>
   <om:OMS cd="nat" name="zero"/>
 </om:OMA>
</definition>
\end{lstlisting}
\end{omgroup}
\end{module}
\end{omgroup}

%%% Local Variables:
%%% mode: latex
%%% TeX-master: t
%%% End:
