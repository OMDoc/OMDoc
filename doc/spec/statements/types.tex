\begin{omgroup}[id=statements-constitutive]{Types of Statements in Mathematics}
\begin{module}[id=statementtypes]

  In the last chapter we introduced mathematical statements as special text fragments that
  state properties of the mathematical objects under discussion and categorized them as
  definitions, theorems, proofs,\ldots. A set of statements about a related set of objects
  make up the context that is needed to understand other statements.  For instance, to
  understand a particular theorem about finite groups, we need to understand the
  definition of a group, its properties, and some basic facts about finite groups
  first. Thus statements interact with context in two ways: the context is built up from
  (clusters of) statements, and statements only make sense with reference to a context. Of
  course this dual interaction of statements with {\emph{context}}\footnote{In linguistics
    and the philosophy of language this phenomenon is studied under the heading of
    ``{\indexalt{discourse theories}{discourse theory}}'', see e.g.~\cite{KamRey:fdtl93}
    for a start and references.}  applies to any text and to communication in general. In
  mathematics, where the problem is aggravated by the load of notation and the need for
  precision for the communicated concepts and objects, contexts are often discussed under
  the label of \twinalt{mathematical theories}{mathematical}{theory}. We will distinguish
  two classes of statements with respect to their interaction with theories: We view
  {\indextoo{axiom}s} and {\indextoo{definition}s} as {\emph{constitutive}} for a given
  theory, since changing this information will yield a different theory (with different
  mathematical properties, see the discussion in {\extref{book}{meta-math}}).  Other
  mathematical statements like {\indextoo{theorem}s} or the {\indextoo{proof}s} that
  support them are not constitutive, since they only illustrate the mathematical objects
  in the theory by explicitly stating the properties that are implicitly determined by the
  constitutive statements.
  
\begin{omtext}
To support this notion of context \omdoc supports an infrastructure for theories using
special \element{theory} elements, which we will introduce in \sref{theories-contexts} and
extend in \sref{complex-theories}.
\twinalt{Theory-constitutive}{theory-constitutive}{element} elements must be contained as
children in a \element{theory} element; we will discuss them in
\sref{definitions}, non-constitutive statements will be defined in
\sref{assertion}. They are allowed to occur outside a \element{theory} element in
\omdoc documents (e.g. as top-level elements), however, if they do they must reference a
theory, \inlinedef{which we will call their \defii{home}{theory}} in a special
\attribute{theory}{statement} attribute. This situates them into the context provided by
this theory and gives them access to all its knowledge. The home theory of
theory-constitutive statements is given by the theory they are contained in.
\end{omtext}

The division of statements into constitutive and non-constitutive ones and the
encapsulation of constitutive elements in \element{theory} elements add a certain
measure of safety to the knowledge management aspect of \omdoc.  Since {\xml} elements
cannot straddle document borders, all constitutive parts of a theory must be contained in
a single document; no constitutive elements can be added later (by other authors), since
this would change the meaning of the theory on which other documents may depend on.
  
Before we introduce the \omdoc elements for theory-constitutive statements, let us
fortify our intuition by considering some mathematical examples.  {\emph{Axioms}} are
assertions about (sets of) mathematical objects and concepts that are assumed to be
true. There are many forms of axiomatic restrictions of meaning in mathematics. Maybe the
best-known are the five Peano Axioms for natural numbers.

\begin{myfig}{peano}{The Peano Axioms}
  \fbox{\begin{minipage}{11cm}
 \begin{enumerate}
 \item 0 is a natural number.
 \item The successor $s(n)$ of a natural number $n$ is a natural number.
 \item 0 is not a successor of any natural number.
 \item The successor function is one-one (i.e. injective).
 \item The set $\NN$ of natural numbers contains only elements that can be
   constructed by axioms 1. and 2.
 \end{enumerate}
 \end{minipage}}
\end{myfig}

\begin{omtext}
The {\twintoo{Peano}{axioms}} in {\myfigref{peano}} (implicitly) introduce three
{\indextoo{symbol}s}: the number 0, the successor function $s$, and the set $\NN$ of
natural numbers. The five axioms in {\myfigref{peano}} jointly constrain their meaning
such that conforming structures exist (the natural numbers we all know and love) any two
structures that interpret 0, $s$, and $\NN$ and satisfy these axioms must be isomorphic.
This is an ideal situation --- the axioms are neither too lax (they allow too many
mathematical structures) or too strict (there are no mathematical structures) --- which is
difficult to obtain. The latter condition (\inlinedef{\defi{inconsistent} theories}) is
especially unsatisfactory, since any statement is a theorem in such theories. As
consistency can easily be lost by adding axioms, mathematicians try to keep
{\twintoo{axiom}{system}s} minimal and only add axioms that are safe.
\end{omtext}

\begin{omtext}
  Sometimes, we can determine that an axiom does not destroy {\indextoo{consistency}} of a
  theory $\cT$ by just looking at its form: for instance, axioms of the form $s=\bA$,
  where $s$ is a symbol that does not occur in $\cT$ and $\bA$ is a formula containing
  only symbols from $\cT$ will introduce no constraints on the meaning of
  $\cT$-symbols. The axiom $s=\bA$ only constrains the meaning of the
  {\twintoo{new}{symbol}} to be a unique object: the one denoted by $\bA$. \inlinedef{We
    speak of a \defii{conservative}{extension} in this case}. So, if $\cT$ was a
  consistent theory, the extension of $\cT$ with the symbol $s$ and the axiom $s=\bA$ must
  be one too. Thus axioms that result in {\twintoo{conservative}{extension}s} can be added
  safely --- i.e. without endangering consistency --- to theories.
\end{omtext}

\begin{definition}[display=flow,id=conservative-extension.def]
  Generally an axiom $\cA$ that results in a {\twintoo{conservative}{extension}} is called
  a \defi{definition} and any new symbol it introduces a \defi{definiendum} (usually
  marked e.g. in boldface font in mathematical texts), and we call \defi{definiens} the
  material in the definition that determines the meaning of the definiendum.
\end{definition}
\end{module}
\end{omgroup}

%%% Local Variables:
%%% mode: latex
%%% TeX-master: t
%%% End:
