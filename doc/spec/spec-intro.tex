%%%%%%%%%%%%%%%%%%%%%%%%%%%%%%%%%%%%%%%%%%%%%%%%%%%%%%%%%%%%%%%%%%%%%%%%%
% This file is part of the LaTeX sources of the OMDoc 1.3 specification
% Copyright (c) 2006 Michael Kohlhase
% This work is licensed by the Creative Commons Share-Alike license
% see http://creativecommons.org/licenses/by-sa/2.5/ for details
%%%%%%%%%%%%%%%%%%%%%%%%%%%%%%%%%%%%%%%%%%%%%%%%%%%%%%%%%%%%%%%%%%%%%%%%%
\begin{tchapter}[id=spec-intro]{General Aspects of the OMDoc Format}
\ednote{MK: some intro here.}
\begin{tsection}[id=modular]{OMDoc as a Modular Format}
  A modular approach to design is generally accepted as best practice in the development
  of any type of complex application. It separates the application's functionality into a
  number of "{\indextoo{building blocks}}" or "{\indextoo{module}s}", which are
  subsequently combined according to specific rules to form the entire application. This
  approach offers numerous advantages: The increased {\indextoo{conceptual clarity}}
  allows developers to share ideas and code, and it encourages reuse by creating
  well-defined modules that perform a particular task. Modularization also reduces
  complexity by decomposition of the application's functionality and thus decreases
  debugging time by localizing errors due to design changes. Finally, flexibility and
  maintainability of the application are increased because single modules can be upgraded
  or replaced independently of others.

  The {\omdoc} vocabulary has been split by thematic role, which we will briefly overview
  in {\myfigsref{core-modules}{ext-modules}} before we go into the specifics of the respective modules
  in {\mychaplref{mobj}{quiz}}. To avoid repetition, we will introduce some attributes
  already in this chapter that are shared by elements from all modules. In
  {\mychapref{document-model}} we will discuss the {\omdoc} document model and possible
  sub-languages of {\omdoc} that only make use of parts of the functionality
  (\mysecref{sub-languages}).

\begin{myfig}{core-modules}{The {\omdoc} Modules}
\begin{small}
\fbox{\begin{tabular}{|l|l|l|l|}\hline
  Module & Title & Required? & Chapter\\\hline\hline
  {\bf\MOBJmodule{spec}} &  Mathematical Objects & yes & {\mychapref{mobj}}\\\hline
    \multicolumn{4}{|p{11cm}|}{\em\footnotesize Formulae are a central part of mathematical
       documents; this module integrates the content-oriented representation
       formats {\openmath} and {\mathml} into {\omdoc}}\\\hline\hline
  {\bf\MTXTmodule{spec}} &  Mathematical Text & yes & {\mychapref{mtxt}}\\\hline
    \multicolumn{4}{|p{11cm}|}{\em\footnotesize Mathematical vernacular,
  i.e. natural language with embedded formulae}\\\hline\hline
  {\bf\DOCmodule{spec}} & Document Infrastructure & yes & {\mychapref{omdoc-infrastructure}}\\\hline
    \multicolumn{4}{|p{11cm}|}{\em\footnotesize  A basic infrastructure for
      assembling pieces of  mathematical knowledge into functional documents and 
      referencing their parts }\\\hline\hline
  {\bf\RTmodule{spec}} & Rich Text Structure & no & {\mysecref{rt}}\\\hline
    \multicolumn{4}{|p{11cm}|}{\em\footnotesize Rich text structure in
  mathematical vernacular (lists, paragraphs, tables, \ldots)}\\\hline\hline
  {\bf\STmodule{spec}} &  Mathematical Statements & no  & {\mychapref{statements}}\\\hline
    \multicolumn{4}{|p{11cm}|}{\em\footnotesize Markup for mathematical forms like 
      {\indextoo{theorem}s},  {\indextoo{axiom}s}, {\indextoo{definition}s}, 
      and {\indextoo{example}s} that can be used to specify or define properties
      of given mathematical objects and theories to group mathematical
  statements and provide a notion of context.}\\\hline\hline
  {\bf\PFmodule{spec}} &  Proofs and proof objects & no & {\mychapref{proofs}}\\\hline 
    \multicolumn{4}{|p{11cm}|}{\em\footnotesize Structure of proofs
     and argumentations at various levels of details and formality}\\\hline\hline
 {\bf\PRESmodule{spec}} & Presentation Information & no &  {\mychapref{pres}}\\\hline
    \multicolumn{4}{|p{11cm}|}{\em\footnotesize Limited functionality for
    specifying presentation and notation information for local typographic
      conventions  that cannot be determined by general principles alone}\\\hline\hline
 \end{tabular}}
\end{small}
\end{myfig}
The modules in {\myfigref{core-modules}} are required (mathematical documents without them
do not really make sense), the ones in {\myfigref{ext-modules}} are optional.

The document-structuring elements in module {\DOCmodule{spec}} have an attribute
{\attributeshort{modules}} that allows to specify which of the modules are used in a
particular document (see {\mychapref{omdoc-infrastructure}} and
{\mysecref{sub-languages}}).
\begin{myfig}{ext-modules}{The {\omdoc} Modules}
\begin{small}
\fbox{\begin{tabular}{|l|l|l|l|}\hline
  Module & Title & Required? & Chapter\\\hline\hline
 {\bf\DCmodule{spec}} & Dublin Core Metadata & yes &   {\mysecsref{dc-elements}{dc-roles}}\\\hline
    \multicolumn{4}{|p{11cm}|}{\em\footnotesize Contains bibliographical ``{\twindef{data}{about data}}'',
      which can be used to annotate many {\omdoc} elements by descriptive and
      administrative information that facilitates navigation and organization}\\\hline\hline 
  {\bf\CCmodule{spec}} & Creative Commons Metadata & yes & {\mysecref{creativecommons}}\\\hline
    \multicolumn{4}{|p{11cm}|}{\em\footnotesize Licenses for text use}\\\hline\hline
 {\bf\ADTmodule{spec}} &  Abstract Data Types & no & {\mychapref{adt}}\\\hline 
    \multicolumn{4}{|p{11cm}|}{\em\footnotesize  Definition schemata for
      sets that are built up inductively from constructor symbols}\\\hline\hline 
  {\bf\CTHmodule{spec}} & Complex Theories & no & {\mychapref{complex-theories}}\\\hline
    \multicolumn{4}{|p{11cm}|}{\em\footnotesize Theory morphisms; they can be used
    to structure mathematical theories}\\\hline\hline
  {\bf\DGmodule{spec}} & Development Graphs & no & {\mysecref{development-graphs}}\\\hline
    \multicolumn{4}{|p{11cm}|}{\em\footnotesize Infrastructure for managing theory
  inclusions, change management}\\\hline\hline
  {\bf\EXTmodule{spec}} & Applets, Code, and Data & no & {\mychapref{ext}}\\\hline
    \multicolumn{4}{|p{11cm}|}{\em\footnotesize Markup for applets, program code,
  and data (e.g. images, measurements, \ldots)}\\\hline\hline
 {\bf\QUIZmodule{spec}} &  Infrastructure for Assessments & no & {\mychapref{quiz}}\\\hline
    \multicolumn{4}{|p{11cm}|}{\em\footnotesize Markup for exercises integrated
    into the {\omdoc} document model}\\\hline 
  \end{tabular}}
\end{small}
\end{myfig}
\end{tsection}

\begin{tsection}[id=omdoc-ns]{The OMDoc Namespaces}
  
  The namespace for the {\omdoc} format is the URI\atwin{OMDoc}{namespace}{URI}
  \url{http://omdoc.org/ns}. Note that the {\omdoc}
  namespace\twin{OMDoc}{namespace} does not reflect the versions, this is done in the
  {\attributeshort{version}} attribute on the {\twintoo{document}{root}} element
  {\element{omdoc}} (see {\mychapref{omdoc-infrastructure}}).  As a consequence, the
  {\omdoc} vocabulary identified by this namespace is not static, it can change with each
  new {\omdoc} version. However, if it does, the changes will be documented in later
  versions of the specification: the latest released version can be found
  at~\cite{URL:omdocspec}.


  \begin{myfig}{omdoc-namespaces}{OMDoc Namespaces}\scriptsize
    \begin{tabular}{|l|l|l|}\hline
      Format      & namespace URI & see \\\hline\hline
      Dublin Core & \url{http://purl.org/dc/elements/1.1/} &   {\mysecsref{dc-elements}{dc-roles}}\\\hline
      Creative Commons & \url{http://creativecommons.org/ns} & {\mysecref{creativecommons}}\\\hline
      {\mathml} & \url{http://www.w3.org/1998/Math/MathML} & {\mysecref{cmml}}\\\hline
      {\openmath} & \url{http://www.openmath.org/OpenMath} & {\mysecref{openmath}}\\\hline
      {\xslt} & \url{http://www.w3.org/1999/XSL/Transform} & {\mychapref{pres}}\\\hline
    \end{tabular}
  \end{myfig}
  In an {\omdoc} document, the {\omdoc} namespace must be specified either using a
  {\twintoo{namespace}{declaration}} of the form
  {\snippet{xmlns="}}\url{http://omdoc.org/ns}{\snippet{"}} on the {\element{omdoc}}
  element or by prefixing the {\twintoo{local}{name}s} of the {\omdoc} elements by a
  namespace prefix ({\omdoc} customarily use the prefixes {\snippet{omdoc:}} or
  {\snippet{o:}}) that is declared by a {\atwintoo{namespace}{prefix}{declaration}} of the
  form {\snippet{xmlns:o="}}\url{http://omdoc.org/ns}{\snippet{"}} on some element
  dominating the {\omdoc} element in question (see {\mysecref{xml}} for an
  introduction). {\omdoc} also uses the namespaces in
  \myfigref{omdoc-namespaces}\footnote{In this specification we will use the
    {\twintoo{namespace}{prefix}es} above on all the elements we reference in text unless
    they are in the {\omdoc} namespace.}  Thus a typical document root of an {\omdoc}
  document looks as follows:
  \begin{lstlisting}[mathescape]
<?xml version="1.0" encoding="utf-8"?>
<omdoc xml:id="test.omdoc" version="1.3"
  xmlns="http://omdoc.org/ns"
  xmlns:cc="http://creativecommons.org/ns"
  xmlns:dc="http://purl.org/dc/elements/1.1/"
  xmlns:om="http://www.openmath.org/OpenMath"
  xmlns:m="http://www.w3.org/1998/Math/MathML">
$\ldots$
</omdoc>
\end{lstlisting}  
\end{tsection}

\begin{tsection}[id=common-attribs]{Common Attributes in OMDoc}
  There are some attributes that are common to many {\omdoc} elements, so we will describe
  them here before we go into the specifics of the respective elements themselves

  \begin{tsubsection}[id=foreign-attribs]{Foreign-Namespace Attributes}

  Generally, the {\omdoc} format allows any attributes from foreign (i.e. non-{\omdoc})
  namespaces\twin{foreign}{namespace} on the {\omdoc} elements. This is a commonly found
  feature that makes the {\xml} encoding of the {\omdoc} format extensible. Note that the
  attributes defined in this specification are in the default (empty)
  namespace\twin{default}{namespace}\twin{empty}{namespace}: they do not carry a namespace
  prefix. So any attribute of the form {\snippet{na:xxx}} is allowed as long as it is in
  the scope of a suitable {\atwintoo{namespace}{prefix}{declaration}}.
\end{tsubsection}

\begin{tsubsection}[id=identifiers]{XML Identifiers}
  Many {\omdoc} elements have optional {\attributeshort[ns-attr=xml]{id}} attributes that
  can be used as identifiers to reference them. These attributes are of type
  {\snippet{ID}}\twin{type}{ID}, they must be unique in the document which is important,
  since many {\xml} applications\twin{XML}{application} offer functionality for
  referencing and retrieving elements by {\snippet{ID}}-type\twin{type}{ID} attributes.
  Note that unlike other {\snippet{ID}}{\twin{ID}{type}}-attributes, in this special case
  it is the name {\attributeshort[ns-attr=xml]{id}}~\cite{XML:id05} that defines the
  {\indextoo{referencing}} and {\indextoo{uniqueness}} functionality, not the type
  declaration in the {\indextoo{DTD}} or {\twintoo{XML}{schema}} (see
  {\mysubsecref{xml-validation}} for a discussion).

  Note that in the {\omdoc} format proper, all {\twintoo{ID}{type}} attributes are of the
  form {\attributeshort[ns-attr=xml]{id}}. However in the older {\openmath} and {\mathml}
  standards, they still have the form {\attributeshort{id}}. The latter are only
  recognized to be of type {\snippet{ID}}, if a document type or {\xml}schema is
  present. Therefore it depends on the application context, whether a DTD should be
  supplied with the {\omdoc} document.
\end{tsubsection}

\begin{oldpart}{MK:this needs some rework, in particular, we have to cite CSS 2.1 and say
    that the cascading rules of that apply.}
\begin{tsubsection}[id=css-attribs]{CSS Attributes}
  For many occasions (e.g. for printing {\omdoc} documents), authors want to control a
  wide variety of aspects of the presentation. {\omdoc} is a content-oriented format, and
  as such only supplies an infrastructure to mark up content-relevant information in
  {\omdoc} elements. To address this dilemma {\xml} offers an interface to 
  {\twintoo{Cascading}{Style Sheet}s} ({\css})~\cite{BosHak:css98}, which allow to specify
  presentational traits like {\twintoo{text}{color}}, {\twintoo{font}{variant}},
  {\indextoo{positioning}}, {\indextoo{padding}}, or {\indextoo{frame}s} of
  {\twintoo{layout}{box}es}, and even {\indextoo{aural}} aspects of the text.
  
  To make use of {\css}, most {\omdoc} elements (all that have
  {\attributeshort[ns-attr=xml]{id}} attributes) have {\attributeshort{style}}
  attributes\footnote{The treatment of the {\css} attributes has changed from
    {\omdocv{1.1}}, see the discussion on
    page~\pageref{style/class-comment}.}\twin{CSS}{attribute} that can be used to specify
  {\css} directives\twin{CSS}{directive} for them. In the {\omdoc} fragment in
  {\mylstref{css-basic}} we have used the {\attribute{style}{omtext}} attribute to specify
  that the text content of the {\element{omtext}} element should be formatted in a
  centered box whose width is 80\% of the surrounding box (probably the page box), and
  that has a 2 pixel wide solid frame of the specified RGB color. Generally {\css}
  directives are of the form {\snippet{A:V}}, where {\snippet{A}} is the name of the
  aspect, and {\snippet{V}} is the value, several {\css} directives\twin{CSS}{directive}
  can be combined in one {\attributeshort{style}} attribute as a
  {\twintoo{semicolon-separated}{list}} (see {\cite{BosHak:css98} and the emerging {\css}
    3} standard).

\begin{lstlisting}[label=lst:css-basic,mathescape,
   caption={Basic {\css} Directives in a {\attributeshort{style}} Attribute},
   index={style,class}]
<?xml version="1.0" encoding="utf-8"?>
<?xml-stylesheet type="text/css" href="http://example.org/style.css"?>
<omdoc xml:id="stylish">
  $\ldots$
  <omtext xml:id="t1" style="width:80%;align:center;border:2px #006699 solid">
    <CMP><xhtml:p>Here comes something 
      <xhtml:span style="font-weight:bold;color:green" class="emphasize">stylish</xhtml:span>!
    </xhtml:p></CMP>
  </omtext>
  $\ldots$
</omdoc>
\end{lstlisting}

Note that many {\css} properties of parent elements are inherited by the children,
if they are not explicitly specified in the child. We could for instance have set
the {\twintoo{font}{family}} of all the children of the {\element{omtext}} element
by adding a directive {\snippet{font-family:sans-serif}} there and then override it by
a directive for the property {\snippet{font-family}} in one of the children.

Frequently recurring groups of {\css} directives can be given symbolic names in {\css}
style sheets\twin{CSS}{style sheet}, which can be referenced by the
{\attributeshort{class}} attribute. In {\mylstref{css-basic}} we have made use of this
with the class {\snippet{emphasize}}, which we assume to be defined in the style sheet
{\snippet{style.css}} associated with the document in the ``{\twintoo{style
    sheet}{processing instruction}}'' in the prolog\footnote{i.e. at the very beginning of
  the {\xml} document before the document type declaration} of the {\xml} document
(see~\cite{Clark:assxd99} for details).  Note that an {\omdoc} element can have both
{\attributeshort{class}} and {\attributeshort{style}} attributes, in this case, precedence
is determined by the rules for {\css} style sheets as specified in~\cite{BosHak:css98}. In
our example in {\mylstref{css-basic}} the directives in the {\attributeshort{style}}
attribute take precedence over the {\css} directives in the style sheet referenced by the
{\attributeshort{class}} attribute on the {\element{xhtml:span}} element. As a
consequence, the word ``stylish'' would appear in green, bold italics.
\end{tsubsection}
\end{oldpart}
\end{tsection}

\begin{tsection}[id=sharing,short=Structure Sharing]{Structure Sharing}

  {\omdoc} is a content markup format, from which documents are produced via a
  presentation process. This ``source character'' of {\omdoc} documents allows to utilize
  structure sharing technologies in the markup\footnote{{\omdocv{1.2}} used the
    {\oldelement{ref}{1.2}} element with {\oldattribute{type}{ref}{1.2}}
    {\oldattval{include}{type}{ref}{1.2}} for this purpose. The new
    {\attributeshort{tref}}-based infrastructure supports validation much better.}. For
  structure sharing {\omdoc} uses the {\attributeshort{tref}} attribute: all content
  elements\ednote{MK: what are they? Intuitively, all that allow the id attributes and
    metadata children. If that is true, we can maybe optimize the schema further.} can be
  used with the {\attributeshort{tref}} whose value is a URI reference to an {\omdoc}
  element instead of the normal element models. We call such an element an
  {\twindef{OMDoc}{reference}}\ednote{MK: probably better use \texttt{oref}
    instead}. Semantically, {\omdoc} references are just placeholders for the {\omdoc}
  objects they reference via their {\attributeshort{tref}} attribute. {\omdoc} references
  require {\omdoc} applications to process the document as if the {\omdoc} reference were
  replaced with the {\omdoc} fragment referenced in the {\attributeshort{tref}} attribute.

  \begin{tsubsection}[id=flattning]{Ref-Reduction and Flattening}

\setbox0=\hbox{\begin{minipage}{5.1cm}
\begin{lstlisting}[label=flattena,mathescape,frame=none,numbers=none,index={omgroup,omtext}]
<omgroup xml:id="text" 
         type="sequence">
  <omtext xml:id="t1">$T_1$</omtext>
  <omgroup xml:id="enum" 
            type="enumeration">
    <omtext xml:id="t2">$T_2$</omtext>
    <omtext xml:id="t3">$T_3$</omtext>
  </omgroup>
  <omtext xml:id="t4">$T_4$</omtext>
</omgroup>
\end{lstlisting}
\end{minipage}}
\setbox1=\hbox{\begin{minipage}{5.5cm}
\begin{lstlisting}[label=flattenb,mathescape,frame=none,numbers=none,index={omgroup,omtext}]
<omgroup xml:id="text" type="sequence">
  <omtext tref="#t1"/>
  <omgroup tref="#enum"/>
  <omtext tref="#t4"/>
</omgroup>

<ignore type="targets"
        comment="already referenced"> 
  <omtext xml:id="t1">$T_1$</omtext>
  <omtext xml:id="t2">$T_2$</omtext>
  <omtext xml:id="t3">$T_3$</omtext>
  <omtext xml:id="t4">$T_4$</omtext>

  <omgroup xml:id="enum" 
           type="enumeration">
    <omtext tref="#t2"/>
    <omtext tref="#t3"/>
  </omgroup>
</ignore>
\end{lstlisting}
\end{minipage}}
\begin{myfig}{flatten}{Flattening a Tree Structure}
\fbox{\box0}$\;\leftrightarrow\;$\fbox{\box1}
\end{myfig}

Let $R$ be an {\omdoc} reference, we call the element the URI in the
{\attributeshort{tref}} points to its {\defin{target}}.  We call the process of replacing
an {\omdoc} reference by its target in a document \twindef{reference}{reduction}, and the
document resulting from the process of systematically and recursively reducing all the
{\omdoc} references the {\atwindef{ref}{normal}{form}} of the source document. Note that
ref-normalization may not always be possible, e.g.  if the ref-targets do not exist or are
inaccessible --- or worse yet, if the relation given by the {\omdoc} references is
{\indextoo{cyclic}}. Moreover, even if it is possible to ref-normalize, this may not lead
to a valid {\omdoc} document, e.g.  since {\snippet{ID}} type\twin{ID}{type} attributes
that were unique in the target documents are no longer in the ref-reduced
one. We will call a document {\defin{ref-reducible}}, iff its ref-normal form exists, and
\defin{ref-valid}, iff the ref-normal form exists and is a valid {\omdoc} document.
  
Note that it may make sense to use documents that are not ref-valid for
{\indextoo{narrative-centered}} documents, such as courseware or slides for talks that
only allude to, but do not fully specify the knowledge structure of the mathematical
knowledge involved. For instance the slides discussed in {\mysecref{narrative-structured}}
do not contain the {\element{theory}} elements that would be needed to make the documents
ref-valid.

{\omdoc} references also allow to ``{\indextoo{flatten}}'' the tree structure in a
document into a list of leaves and relation declarations (see {\myfigref{flatten}} for an
example). It also makes it possible to have more than one view on a document using
{\element{omgroup}} structures that reference a shared set of {\omdoc} elements. Note that
we have embedded the ref-targets of the top-level {\element{omgroup}} element
into an {\element{ignore}} comment, so that an {\omdoc} transformation (e.g. to text form)
does not encounter the same content twice.
\end{tsubsection}

\begin{oldpart}{MK: this needs to be reworked, we also need to think about whether we want
    to do similar cascading for metadata.}
\begin{tsubsection}[id=tref-css-cascading]{Cascading of CSS Attributes}
  While the {\omdoc} approach to specifying document structure is a much more flexible
  (database-like) approach to representing structured documents\footnote{The simple tree
    model is sufficient for simple markup of existing mathematical texts and to replay
    them verbatim in a browser, but is insufficient e.g. for generating individualized
    presentations at multiple levels of abstractions from the representation. The {\omdoc}
    text model --- if taken to its extreme --- allows to specify the respective role and
    contributions of smaller text units, even down to the sub-sentence level, and to make
    the structure of mathematical texts machine-understandable. Thus, an advanced
    presentation engine like the {\activemath} system~\cite{SieBen:acgap00} can --- for
    instance --- extract document fragments based on the preferences of the respective
    user.}  than the tree model, it puts a much heavier load on a system for presenting
  the text to humans. In essence the presentation system must be able to recover the left
  representation from the right one in {\myfigref{flatten}}.  Generally, any {\omdoc}
  element defines a fragment of the {\omdoc} it is contained in: everything between the
  start and end tags and (recursively) those elements that are reached from it by
  following the {\omdoc} references.  In particular, the text fragment corresponding to
  the element with {\attribute[ns-attr=xml]{id}{omtext}}={\snippet{"text"}} in the right
  {\omdoc} of~\myfigref{flatten} is just the one on the left.

  In {\mysecref{common-attribs}} we have introduced the {\css}
  attributes\twin{CSS}{attribute} {\attribute{style}{ref}} and {\attribute{class}{ref}},
  which are present on all {\omdoc} elements. In the case of a {\omdoc} reference, there
  is a problem, since the content of these can be incompatible. In general, the rule for
  determining the style information for an element is that we treat the replacement
  element as if it were a child of the reference, and then determine the values of the
  {\css} properties\twin{CSS}{property} of the {\omdoc} reference by inheritance.
\end{tsubsection}
\end{oldpart}
\end{tsection}

\end{tchapter}
%%% Local Variables: 
%%% mode: latex
%%% TeX-master: "omdoc"
%%% End: 


% LocalWords:  omdoc mobj attribs DTD XMLSchema dtd css omtext RGB
% LocalWords:  lst emph xml utf stylesheet href DOCTYPE px CMP serif ns dc attr
% LocalWords:  xmlns creativecommons cmml openmath cc om mtxt rt adt ext pres
% LocalWords:  na xxx es sans mathescape prolog
