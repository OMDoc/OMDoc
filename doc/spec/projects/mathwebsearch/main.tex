%%%%%%%%%%%%%%%%%%%%%%%%%%%%%%%%%%%%%%%%%%%%%%%%%%%%%%%%%%%%%%%%%%%%%%%%%
% This file is part of the LaTeX sources of the OMDoc 1.3 specifiation
% Copyright (c) 2006 Ioan Sucan, Michael Kohlhase
% This work is licensed by the Creative Commons Share-Alike license
% see http://creativecommons.org/licenses/by-sa/2.5/ for details
\svnInfo $Id: main.tex 8453 2009-08-04 09:58:26Z kohlhase $
\svnKeyword $HeadURL: https://svn.omdoc.org/repos/omdoc/branches/omdoc-1.3/doc/spec/projects/mathwebsearch/main.tex $
%%%%%%%%%%%%%%%%%%%%%%%%%%%%%%%%%%%%%%%%%%%%%%%%%%%%%%%%%%%%%%%%%%%%%%%%%

\section{A Search Engine for Mathematical Formulae}
\begin{project}{matwebsearch}{http://search.mathweb.org/}
\pauthors{Ioan Sucan\and Michael Kohlhase}
\pinstitute{Computer Science, International University Bremen}
\end{project}

As the world of information technology grows, being able to quickly search data of
interest becomes one of the most important tasks in any kind of environment, be it
academic or not. We present a search engine for mathematical formulae. The {\mmlsearch}
system harvests the web for content representations of formulae (currently {\mathml} and
{\openmath}) and indexes them with substitution tree indexing, a technique originally
developed for accessing intermediate results in automated theorem provers. For querying,
we present a generic language extension approach that allows to construct queries by
minimally annotating existing representations.

Generally, searching for mathematical formulae is a non-trivial problem --- especially if
we want to be able to search occurrences of the query term as sub-formulae --- for the
following reasons:
\begin{enumerate}
\item {\emph{Mathematical notation is context-dependent}}. For instance, binomial
  coefficients can come in a variety of notations depending on the context: $\left(n\atop
    k\right)$, $_nC^k$, $C^n_k$, and $C^k_n$ all mean the same thing: $n!\over k!(n-k)!$.
  In a formula search we would like to retrieve all forms irrespective of the notations.
\item {\emph{Identical presentations can stand for multiple distinct mathematical
      objects}}, e.g. an integral expression of the form $\int f(x) dx$ can mean a Riemann
  Integral, a Lebesgue Integral, or any other of the 10 to 15 known anti-derivative
  operators.  We would like to be able to restrict the search to the particular integral
  type we are interested in at the moment.
\item {\emph{Certain variations of notations are widely considered irrelevant}}, for
  instance $\int f(x) dx$ means the same as $\int f(y) dy$ (modulo $\alpha$-equivalence),
  so we would like to find both, even if we only query for one of them.
\end{enumerate}
To solve this formula search problem, we concentrate on {\emph{content representations of
    mathematical formulae}}, since they are presentation-independent and disambiguate
mathematical notions.

\subsubsection{A Running Example: The Power of a Signal}\label{sec:runnex}

A standard use case for {\mmlsearch} is that of an engineer trying to solve a mathematical
problem such as finding the power of a given signal $s(t)$.  Of course our engineer is
well-versed in signal processing and remembers that a signal's power has something to do
with integrating its square, but has forgotten the details of how to compute the necessary
integrals. He will therefore call up {\mmlsearch} to search for something that looks like
$\int_?^?s^2(t)dt$ (for the concrete syntax of the query see {\mylstref{runningex-query}}
in {\mysecref{query}}). {\mmlsearch} finds a document about Parseval's Theorem, more
specifically $\frac{1}{T}\int_0^Ts^2(t)dt = \Sigma_{k=-\infty}^{\infty}\mid c_k\mid^2$
where $c_k$ are the Fourier coefficients of the signal.  In short, our engineer found the
exact formula he was looking for (he had missed the factor in front and the integration
limits) and moreover a theorem he may be able to use.


\subsubsection{Indexing Mathematical Formulae}

For indexing mathematical formulae on the web, we will interpret them as first-order
terms. This allows us to use a technique from automated reasoning called {\emph{term
    indexing}}~\cite{Graf:ti96}. This is the process by which a set of terms is stored in
a special purpose data structure (the {\bf{index}}) where common parts of the terms are
potentially shared, so as to minimize access time and storage. The indexing technique we
work with is a form of tree-based indexing called {\emph{substitution-tree indexing}}. A
substitution tree, as the name suggests, is simply a tree where substitutions are the
nodes. A term is constructed by successively applying substitutions along a path in the
tree, the leaves represent the terms stored in the index. Internal nodes of the tree are
{\bf{generic terms}} and represent similarities between terms.

The main advantage of substitution tree indexing is that we only store substitutions, not
the actual terms, and this leads to a small memory footprint.  Adding data to an existing
index is simple and fast, querying the data structure is reduced to performing a walk down
the tree.  Index building is done in similar fashion to~\cite{Graf:ti96}. Once the index
is built, we keep the actual term instead of the substitution at each node, so we do not
have to recompute it with every search. At first glance this may seem to be against the
idea of indexing, as we would store all the terms again, not only the
substitutions. However, due to the tree-like structure of the terms, we can in fact store
only a pool of (sub)terms and define the terms in our index using pointers to elements of
the pool (which are simply other terms). To each of the indexed terms, a data string is
attached --- a string that represents the exact location of the term. We use
XPointer~\cite{GroMal:xf03} to specify this.

Unfortunately, substitution tree indexing does not support subterm search in an elegant
fashion, so when adding a term to the index, we add all its subterms as well. This simple
trick works well: the increase in index size remains manageable and it greatly simplifies
the implementation.

\subsubsection{A Query Language for Content Mathematics}\label{sec:query}

When designing a query language for mathematical formulae, we have to satisfy a couple of
conflicting constraints. The language should be content-oriented and familiar, but it
should not be specialized to a given content representation format.  Our approach to this
problem is to use a simple, generic extension mechanism for {\xml}-based representation
formats rather than a genuine query language itself.  The query extension is very simple,
it adds two new attributes to the respective languages:
{\attributeshort[ns-attr=mq]{generic}} and {\attributeshort[ns-attr=mq]{anyorder}}, where
the prefix {\tt{mq:}} abbreviates the namespace URI \url{http://mathweb.org/MathQuery/}
for {\mmlsearch}.

In this way, the user need not master a new representation language, and we can generate
queries by copy and paste and then make parts of the formulae generic by simply adding
suitable attributes. We will use Content {\mathml}~\cite{CarIon:MathML03} in the example,
but {\mmlsearch} also supports {\openmath} and a shorthand notation that resembles the
internal representation we are using for terms (prefix notation).  The
{\attributeshort[ns-attr=mq]{generic}} attribute takes string values and can be specified
for any element in the query, making it into a (named) query variable: its contents are
ignored and it matches any term in the search process.  

While of searching expressions of the form $A = B$, we might like to find occurrences of
$B = A$ as well. At this point the {\attributeshort[ns-attr=mq]{anyorder}} attribute comes
in.  Inside an {\element{apply}} tag, the first child defines the function to be applied;
if this child has the attribute {\attributeshort[ns-attr=mq]{anyorder}} defined with the
value ``yes'', the order of the subsequent children is ignored.  If we do not want to
specify the function name, we can use the {\attributeshort[ns-attr=mq]{generic}} attribute
again, but this time for the first child of an {\element{apply}} tag.  Given the above,
the query of our running example has the form presented in
{\mylstref{runningex-query}}. Note that we do not know the integration limits or whether
the formula is complete or not.

\begin{lstlisting}[label=lst:runningex-query,caption=Query for Signal Power,numbers=none]
<math xmlns="http://www.w3.org/1998/Math/MathML" 
      xmlns:mq="http://mathweb.org/MathQuery">
  <apply><int/>
    <domainofapplication mq:generic="domain"/> 
    <bvar> <ci mq:generic="time"/> </bvar>
    <apply><power/>
      <apply><ci mq:generic="fun"></ci><ci mq:generic="time"/></apply>
      <cn>2</cn>
    </apply>
  </apply>
</math>
\end{lstlisting}

\subsubsection{Input Processing}\label{subsec:input-processing}

{\mmlsearch} can process any kind of {\xml} representation for content mathematics. The
system is modular and provides an interface which allows to add {\xml}-to-index-term
transformers. . We will discuss input processing for {\cmathml}.

\setbox0=\hbox{\footnotesize\begin{minipage}[t]{3cm}
    \begin{itemize}\raggedright
    \item[1)] Mathematical
     expression:\\
    $f(x) = y$\\[3ex]
  \item[3)] Term
    representation:\\
    $eq(f(x), y)$
  \end{itemize}
\end{minipage} 
\begin{minipage}[t]{3.4cm}
    \begin{itemize}\raggedright
    \item[2)] Content {\mathml}:\\[-1.5ex]
\begin{lstlisting}[frame=none,numbers=none]
<apply><eq/>
  <apply>
    <ci>f</ci>
    <ci>x</ci>
  </apply>
  <ci>y</ci>
</apply>
\end{lstlisting}
\end{itemize}
\end{minipage}}
\begin{wrapfigure}{r}{7cm}
\fbox{\usebox0}
\vspace*{-.6cm}
\end{wrapfigure}
Given an {\xml} document, we create an index term for each of its {\element{math}}
elements. Consider the example on the right: We have the standard mathematical notation of
an equation (1), its Content {\mathml} representation (2), and the term we extract for
indexing (3). As previously stated, any mathematical construct can be represented in a
similar fashion.

Search modulo $\alpha$-renaming becomes available via a very simple input processing
trick: during input processing, we add a {\attributeshort[ns-attr=mq]{generic}} attribute
to every bound variable (but with distinct strings for different variables). Therefore in
our running example the query variable $t$ ({\tt{@time}} in {\mylstref{runningex-query}})
in the query $\int_?^?s^2(t)dt$ is made generic, therefore the query would also find the
variant $\frac{1}{T}\int_0^Ts^2(x)dx = \Sigma_{k=-\infty}^{\infty}\mid c_k\mid^2$.

\subsubsection{Result reporting}\label{subsec:result-reporting}

For a search engine for mathematical formulae we need to augment the set of result items
(usually page title, description, and page link) reported to the user for each hit. As
typical pages contain multiple formulae, we need to report the exact occurrence of the hit
in the page. We do this by supplying an {\sc{XPointer}} reference where possible.
Concretely, we group all occurrences into one page item that can be expanded on demand and
within this we order the groups by number of contained references. For any given result, a
detailed view is available.  This view shows the exact term that was matched (using
Presentation {\mathml}) and the used substitution (a mapping from the query variables
specified by the {\attributeshort[ns-attr=mq]{generic}} attributes to certain subterms) to
match that specific term.

\subsubsection{Case Studies and Results}\label{sec:case-study}

We have tested our implementation on the content repository of the {\connexions} Project,
available via the OAI protocol~\cite{OAI:protocol}. This gives us a set of over 3,200
articles with mathematical expressions to work on. The number of terms represented in
these documents is approximately 53,000 (77,000 including subterms). The average term
depth is 3.6 and the maximal one is 14.  Typical query execution times on this index are
in the range of milliseconds. The search in our running example takes 14 ms for
instance. There are, however, complex searches (e.g. using the
{\attributeshort[ns-attr=mq]{anyorder}} attribute) that internally call the searching
routine multiple times and take up to 200 ms but for realistic examples execution time is
below 50 ms. We are currently building an index of the 86,000 Content {\mathml} formulae
from \url{http://functions.wolfram.com}. Here, term depths are much larger (average term
depth 5.7, maximally around 50) resulting in a much larger index: it is just short of 2
million formulae. First experiments indicate that search times are largely unchanged by
the increase in index size.

In the long run, it would be interesting to interface {\mmlsearch} with a regular web
search engine and create a powerful, specialized, full-feature application. This would
resolve the main disadvantage our implementation has -- it cannot search for simple
text. A simple socket-based search API allows to integrate {\mmlsearch} into other
content-based mathematical software systems.
%%% Local Variables: 
%%% mode: stex
%%% TeX-master: "../../omdoc"
%%% End: 

% LocalWords:  matwebsearch Ioan Sucan nC dx dy dt runningex XPointer ns attr
% LocalWords:  mq anyorder xmlns domainofapplication bvar ci cn eq OAI API
% LocalWords:  screenshots
