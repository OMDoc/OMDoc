\begin{module}[id=theories]
\begin{omgroup}{Simple Theories}
The module {\STmodule{spec}}
presented in this chapter introduces a part of this infrastructure, which can already
address the first two concerns. For the latter, we need the machinery for complex theories
introduced in \sref{complex-theories}.

\begin{definition}[id=theory.def]
  Theories are specified by the {\eldef{theory}} element in \omdoc, which has a required
  \attribute[ns-attr=xml]{id}{theory} attribute for referencing the theory. Furthermore,
  the \element{theory} element can have the \attribute[ns-elt=om]{cdbase}{theory}
  attribute that allows to specify the \attribute{cdbase}{OMS} this theory uses for
  disambiguation on \element[ns-elt=om]{OMS} elements (see \sref{openmath} for a
  discussion).
\end{definition}
Additional information about the theory like a title or a short description can be given
in the \element{metadata} element. AfTer this, any {\indextoo{top-level}} \omdoc
element can occur, including the theory-constitutive elements introduced in
{\srefs{statements-constitutive}{definitions}}, even \element{theory} elements
themselves. Note that theory-constitutive elements may {\emph{only}} occur in
\element{theory} elements.

\begin{definition}[id=tgroup.def]
  Theories can be structured like documents e.g. into sections and the like (see
  \sref{sectioning} for a discussion) via the {\eldef{tgroup}} element, which behaves
  exactly like the \element{omdoc} element introduced in \sref{sectioning} except that
  it also allows theory-constitutive elements, but does not allow a
  \attribute{theory}{omdoc} attribute, since this information is already given by the
  dominating \element{theory} element.\footnote{This element has been introduced to keep
    \omdoc validation manageable: We cannot directly use the \element{omdoc}
    element,since there is no simple, context-free way to determine whether an
    \element{omdoc} is dominated by a \element{theory} element.}
\end{definition}  

\begin{presonly}
\begin{myfig}{simple-thy}{Theories in \omdoc}
\begin{scriptsize}
\begin{tabular}{|>{\tt}l|>{\tt}l|>{\tt}p{5.4truecm}|c|>{\tt}p{2truecm}|}\hline
{\rm Element}& \multicolumn{2}{l|}{Attributes\hspace*{2.25cm}} & M & Content  \\\hline
             & {\rm Req.}  & {\rm Optional}                    & D &           \\\hline\hline
 theory      &             & xml:id, class, style, cdbase, 
                             cdversion, cdrevision, cdstatus, cdurl, 
                             cdreviewdate                      & + & (\llquote{top+thc} | imports)*\\\hline
 imports     & from        & id, type, class, style            & + & \\\hline
 tgroup      &   & xml:id, modules, type, class, style         & +  & (\llquote{top+thc})* \\\hline
 \multicolumn{5}{|p{11cm}|}{where \llquote{top+thc} stands for top-level and
   theory-constitutive elements}\\\hline
\end{tabular}
\end{scriptsize}
\end{myfig}
\end{presonly}

\begin{omgroup}[id=inheritance]{Simple Inheritance}

\element{theory} elements can contain \element{imports} elements (mixed in
with the top-level ones) to specify inheritance: The main idea behind structured theories
and specification is that not all theory-constitutive elements need to be explicitly
stated in a theory; they can be inherited from other theories. Formally, the set of
theory-constitutive elements in a theory is the union of those that are explicitly
specified and those that are imported from other theories. This has consequences later on,
for instance, these are available for use in proofs. See
\sref{proofs.justifications} for details on availability of assertional statements in
proofs and justifications.

\begin{definition}[id=imports.def]
  The meaning of the {\eldef{imports}} element is determined by two attributes:
\begin{description}
\item[\attribute{from}{imports}] The value of this attribute is a
  {\twintoo{URI}{reference}} that specifies the \defii{source}{theory},
    i.e. the theory we import from.  The current theory (the one specified in
    the parent of the \element{imports} element, we will call it the
    \defii{target}{theory}) inherits the constitutive elements from the source
  theory.
\item[\attribute{type}{imports}] This optional attribute can have the values
  \attval{global}{type}{imports} and \attval{local}{type}{imports} (the former is
  assumed, if the attribute is absent): We call constitutive elements \defi{local} to
  the current theory, if they are explicitly defined as children, and else
  \defi{inherited}.  A \defii{local}{import} (an \element{imports} element with
  {\snippet{type="local"}}) only imports the local elements of the source theory, a
  {\indextoo{global}} import also the inherited ones.
\end{description}
\end{definition}

The meaning of nested \element{theory} elements is given in terms of an implicit imports
relation: The inner theory imports from the outer one. Thus
\begin{lstlisting}[label=lst:nested-thy,index={theory}]
<theory xml:id="a.thy">
  <symbol name="aa"/>
  <theory xml:id="b.thy">
    <symbol name="cc"/>
    <definition xml:id="cc.def" for="cc" type="simple">
       <OMS cd="a.thy" name="af"/>
    </definition>
  </theory>
</theory>
\end{lstlisting}
is equivalent to 
\begin{lstlisting}[label=lst:nested-thy-equiv,index={theory}]
<theory xml:id="a.thy"><symbol name="aa"/></theory>
<theory xml:id="b.thy">
  <imports from="#a.thy" type="global"/>
  <symbol name="cc"/>
  <definition xml:id="cc.def" for="cc" type="simple">
     <OMS cd="a.thy" name="af"/>
  </definition>
</theory>
\end{lstlisting}
In particular, the symbol {\snippet{cc}} is visible only in theory {\snippet{b.thy}}, not
in the rest of theory {\snippet{a.thy}} in the first representation.

Note that the inherited elements of the current theory can themselves be inherited in the
source theory. For instance, in the {\mylstref{def-group}} the {\snippet{left-inv}} is the
only local axiom of the theory {\snippetin{group}}, which has the inherited axioms
{\snippet{closed}}, {\snippet{assoc}}, {\snippet{left-unit}}.

In order for this import mechanism to work properly, the
{\twintoo{inheritance}{relation}}, i.e.  the relation on theories induced by the
\element{imports} elements, must be {\indextoo{acyclic}}. There is another, more subtle
constraint on the inheritance relation concerning multiple inheritance.  Consider the
situation in {\mylstref{multiple-inheritance}}: here theories {\snippet{A}} and
{\snippet{B}} import theories with {\snippet{xml:id="mythy"}}, but from different
URIs. Thus we have no guarantee that the theories are identical, and semantic integrity of
the theory {\snippet{C}} is at risk. Note that this situation might in fact be totally
unproblematic, e.g. if both URIs point to the same document, or if the referenced
documents are identical or equivalent. But we cannot guarantee this by content markup
alone, we have to forbid it to be safe.

\begin{lstlisting}[label=lst:multiple-inheritance,
  caption={Problematic Multiple Inheritance},
  index={theory,symbol,axiom,imports}]
<theory xml:id="A">
  <imports from="http://red.com/theories.omdoc#mythy"/>
</theory>
<theory xml:id="B">
  <imports from="http://blue.org/cd/all.omdoc#mythy"/>
</theory>
<theory xml:id="C"><imports from="#A"/><imports from="#B"/></theory>
\end{lstlisting}

Let us now formulate the constraint carefully, 
\begin{definition}[display=flow,id=base.uri.def]
  the \defii{base}{URI} of an {\xml} document is the {\indextoo{URI}} that has been
  used to retrieve it.
\end{definition}
We adapt this to \omdoc theory elements: the base URI of an imported theory is the URI
declared in the \attribute{cdbase}{theory} attribute of the \element{theory} element
(if present) or the base URI of the document which contains it\footnote{Note that the base
  URI of the document is sufficient, since a valid \omdoc document cannot contain more
  than one \element{theory} element for a given
  \attribute[ns-attr=xml]{id}{theory}}. For theories that are imported along a chain of
global imports, which include {\twintoo{relative}{URI}s}, we need to employ
{\twintoo{URI}{normalization}} to compute the {\twintoo{effective}{URI}}.  Now the
constraint is that any two imported theories that have the same value of the
\attribute[ns-attr=xml]{id}{theory} attribute must have the same base URI. Note that
this does not imply a global unicity constraint for \attribute[ns-attr=xml]{id}{theory}
values of \element{theory} elements, it only means that the mapping of theory
identifiers to URIs is unambiguous in the dependency cone of a theory.

In {\mylstref{def-group}} we have specified three algebraic theories that gradually build
up a theory of groups importing theory-constitutive statements (symbols, axioms, and
definitions) from earlier theories and adding their own content. The theory
{\snippetin{semigroup}} provides symbols for an operation {\snippetin{op}} on a base set
{\snippetin{set}} and has the axioms for closure and associativity of
{\snippetin{op}}. The theory of monoids imports these without modification and uses them
to state the {\snippet{left-unit}} axiom. The theory {\snippetin{monoid}} then proceeds to
add a symbol {\snippetin{neut}} and an axiom that states that it acts as a left unit with
respect to {\snippetin{set}} and {\snippetin{op}}.  The theory {\snippetin{group}}
continues this process by adding a symbol {\snippetin{inv}} for the function that gives
inverses and an axiom that states its meaning.

\begin{lstlisting}[label=lst:def-group,escapechar=\%,mathescape,
  caption={A Structured Development of Algebraic Theories in \omdoc},
  index={theory,symbol,axiom,imports}]
<theory xml:id="semigroup">
  <symbol name="set"/><symbol name="op"/>
  <axiom xml:id="closed"> $\ldots$ </axiom><axiom xml:id="assoc"> $\ldots$ </axiom>
</theory>

<theory xml:id="monoid">
  <imports from="#semigroup"/>
  <symbol name="neut"/><symbol name="setstar"/>
  <axiom xml:id="left-unit">
    <h:p>%\tt{neut}% is a left unit for %\tt{op}%.</h:p><FMP>$\allcdot{x\in{\tt{set}}}{{\tt{op}}(x,{\tt{neut}})=x}$</FMP>
  </axiom>
  <definition xml:id="setstar.def" for="setstar" type="implicit">
    <h:p>$\cdot^*$ subtracts the unit from a set </h:p><FMP>$\allcdot{S}{S^*=S\backslash\set{\tt{unit}}}$</FMP>
  </definition>
</theory>

<theory xml:id="group"> 
  <imports from="#monoid"/>
  <symbol name="inv"/>
  <axiom xml:id="left-inv">
    <h:p>For every $X\in\tt{set}$ there is an inverse ${\tt{inv}}(X)$ wrt. %\tt{op}%.</h:p>
  </axiom>
</theory>
\end{lstlisting}

The example in {\mylstref{def-group}} shows that with the notion of theory inheritance it
is possible to re-use parts of theories and add structure to specifications. For instance
it would be very simple to define a theory of {\twintoo{Abelian}{semigroup}s} by adding a
{\twintoo{commutativity}{axiom}}.

\begin{omtext}
  The set of symbols, axioms, and definitions available for use in proofs in the importing
  theory consists of the ones directly specified as \element{symbol}, \element{axiom},
  and \element{definition} elements in the target theory itself (\inlinedef{we speak of
    \defi{local} axioms and definitions in this case} and the ones that are inherited from
  the source theories via \element{imports} elements.  Note that these symbols, axioms,
  and definitions (\inlinedef{we call them \defi{inherited}}) can consist of the local
  ones in the source theories and the ones that are inherited there.
\end{omtext}

The local and inherited symbols, definitions, and axioms are the only ones
available to mathematical statements and proofs. If a symbol is not available in
the home theory (the one given by the dominating \element{theory} element or the
one specified in the \attribute{theory}{statement} attribute of the statement),
then it cannot be used since its semantics is not defined.
\end{omgroup}

\begin{omgroup}[id=identifying]{OMDoc Theories as Content Dictionaries}
\begin{oldpart}{The discussion here depends on the upcoming OM3 standard and MathML3
    recommendation. The material is provisional on the expected outcome and may change in
    the future.}


\begin{omtext}
  In \sref{mobj}, we have introduced the {\openmath} and {\cmathml} representations for
  mathematical objects and formulae. One of the central concepts there was the notion that
  the representation of a symbol includes a pointer to a document that defines its
  meaning.  In the {\openmath} standard, these documents are identified as
  \atwinalt{{\openmath} content dictionaries}{OpenMath}{content}{dictionary}, the
  {\mathml} recommendation is not specific.  In the examples above, we have seen that
  \omdoc documents can contain definitions of mathematical concepts and symbols, thus
  they are also candidates for ``defining documents'' for symbols.  By the {\openmath}2
  standard~\cite{BusCapCar:2oms04} suitable classes of \omdoc documents can act as
  {\openmath} content dictionaries (\inlinedef{we call them \omdoc \adefii{content
      dictionaries}{content dictionary}{OMDoc}}; see \sref{sub-languages.cd}).  The main
  distinguishing feature of \omdoc content dictionaries is that they include
  \element{theory} elements with {\twintoo{symbol}{declaration}s} (see
  \sref{definitions}) that act as the targets for the pointers in the symbol
  representations in {\openmath} and {\cmathml}. The theory name specified in the
  \attribute[ns-attr=xml]{id}{theory} attribute of the \element{theory} element takes
  the place of the {\snippet{CDname}} defined in the \atwinalt{{\openmath} content
    dictionary}{OpenMath}{content}{dictionary}.
\end{omtext}

Furthermore, the {\indextoo{URI}} specified in the \attribute{cdbase}{theory} attribute
is the one used for disambiguation on \element[ns-elt=om]{OMS} elements (see
\sref{openmath} for a discussion).
  
For instance the symbol declaration in {\mylstref{symbol}} can be referenced as\ednote{is
  this really the right {\texttt{cdbase?}}}
\begin{lstlisting}
<OMS cd="elAlg" name="monoid" cdbase="http://omdoc.org/algebra.omdoc"/>
\end{lstlisting}
if it occurs in a theory for elementary algebra whose
\attribute[ns-attr=xml]{id}{theory} attribute has the value {\snippet{elAlg}} and which
occurs in a resource with the URI \url{http://omdoc.org/algebra.omdoc} or if the
\attribute{cdbase}{theory} attribute of the \element{theory} element has the value
\url{http://omdoc.org/algebra.omdoc}.

To be able to act as an {\openmath}2 {\twintoo{content dictionary}{format}}, \omdoc must
be able to express {\twintoo{content dictionary}{metadata}} (see {\mylstref{omcd}} for an
example). For this, the \element{theory} element carries some optional attributes that
allow to specify the administrative metadata of {\openmath} content dictionaries.

\begin{omtext}
  The \attribute{cdstatus}{theory} attribute specifies \inlinedef{the {\defii{content
        dictionary}{status}}}, which can take one of the following values:
  \attval{official}{cdstatus}{theory} (i.e. approved by the {\openmath} Society),
  \attval{experimental}{cdstatus}{theory} (i.e. under development and thus liable to
  change), \attval{private}{cdstatus}{theory} (i.e. used by a private group of
  {\openmath} users) or \attval{obsolete}{cdstatus}{theory} (i.e. only for archival
  purposes). The attributes \attribute{cdversion}{theory} and
  \attribute{cdrevision}{theory} jointly specify \inlinedef{the \defii{content
      dictionary}{version number}, which consists of two parts, a major \defi{version} and
    a \defi{revision}, both of which are non-negative integers}. For details between the
  relation between content dictionary status and versions consult the {\openmath}
  standard~\cite{BusCapCar:2oms04}.
\end{omtext}

\begin{omtext}
Furthermore, the \element{theory} element can have the following attributes:
\begin{description}
\item[\attribute{cdbase}{theory}] for the {\twintoo{content dictionary}{base}} which, when
  combined with the content dictionary name, forms a unique identifier for the content
  dictionary. It may or may not refer to an actual location from which it can be
  retrieved.
\item[\attribute{cdurl}{theory}] for a valid URL where the source file for the content
  dictionary encoding can be found.
\item[\attribute{cdreviewdate}{theory}] for \inlinedef{the \defii{review}{date} of the
    content dictionary, i.e. the date until which the content dictionary is guaranteed to
    remain unchanged}.
\end{description}
\end{omtext}
\end{oldpart}
\end{omgroup}
\end{omgroup}
\end{module}

%%% Local Variables:
%%% mode: latex
%%% TeX-master: t
%%% End:
