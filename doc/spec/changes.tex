%%%%%%%%%%%%%%%%%%%%%%%%%%%%%%%%%%%%%%%%%%%%%%%%%%%%%%%%%%%%%%%%%%%%%%%%%
% This file is part of the LaTeX sources of the OMDoc 1.6 specification
% Copyright (c) 2015 Michael Kohlhase
% This work is licensed by the Creative Commons Share-Alike license
% see http://creativecommons.org/licenses/by-sa/2.5/ for details
% The source original is at https://github.com/KWARC/OMDoc/doc/spec 
%%%%%%%%%%%%%%%%%%%%%%%%%%%%%%%%%%%%%%%%%%%%%%%%%%%%%%%%%%%%%%%%%%%%%%%%%

\begin{omgroup}[creators=miko,id=changelog]{Changes to the specification}

After about 18 Months of development, Version 1.0 of the \omdoc format was released on
November $1^{st}$ 2000 to give users a stable interface to base their documents and
systems on. It was adopted by various projects in {\twintoo{automated}{deduction}},
{\twintoo{algebraic}{specification}}, and {\twintoo{computer-supported}{education}}. The
experience from these projects uncovered a multitude of small deficiencies and extension
possibilities of the format, that have been subsequently discussed in the \omdoc
{\indextoo{community}}.

{\omdocv{1.1}} was released on December $29^{th}$ 2001 as an attempt to roll the
uncontroversial and non-disruptive part of the extensions and corrections into a
consistent language format. The changes to version 1.0 were largely conservative,
adding optional attributes or child elements. Nevertheless, some non-conservative
changes were introduced, but only to less used parts of the format or in order to
remedy design flaws and inconsistencies of version 1.0.

{\omdocv{1.2}} is the mature version in the {\omdocv{1}} series of specifications. It
contains almost no large-scale changes to the document format, except that {\cmathml} is
now allowed as a representation for mathematical objects. But many of the representational
features have been fine-tuned and brought up to date with the maturing {\xml} technology
(e.g. \twinalt{\snippet{ID}}{type}{ID} attributes now follow the XML ID
specification~\cite{XML:id05}, and the Dublin Core elements follow the official
syntax~\cite{DCMI:dmt03}). The main development is that the \omdoc specification, the
DTD, and schema are split into a system of interdependent modules that support independent
development of certain language aspects and simpler specification and deployment of
sub-languages.  {\vomdoc{1.2}} freezes the development so that version 2 can be started
off on the modules.

In the following, we will keep a log on the changes that have occurred in the released
versions of the \omdoc format.  We will briefly tabulate the changes by element
name. For the state of an element we will use the shorthands ``dep'' for
{\indextoo{deprecated}} (i.e. the element is no longer in use in the new \omdoc
version), ``cha'' for {\indextoo{changed}}, if the element is re-structured (i.e.  some
additions and losses), ``new'' if did not exist in the old \omdoc version, ``lib'', if
it was liberalized (e.g. an attribute was made optional) and finally ``aug'' for
{\indextoo{augmented}}, i.e. if it has obtained additional children or attributes in the
new \omdoc version.

All changes will be relative to the previous version, starting out with OMDoc
1.2~\cite{Kohlhase:OMDoc1.2}. For older changes see Appendix A there.

\inspec{changes1.6}
\inspec{changes2.0}
\end{omgroup}
%%% Local Variables: 
%%% mode: latex
%%% TeX-master: "main"
%%% End: 

% LocalWords:  dep cha aug omlet OMFOREIGN ref sortdef def adt IDREF dc attr th
% LocalWords:  classid codebase href trl OMSTR xref URIref omstyle proofobject
% LocalWords:  xslt changelog lib commonname pto crid cr ns mc metacomment rt
% LocalWords:  requation cref OMR cd cdbase larg rarg elt extradata DTD CMP FMP
% LocalWords:  metadata omdoc omtext omdoc dataset cdurl cdreviewdate
% LocalWords:  cdversion cdrevision cdstatus recurse timestamp ren loc
