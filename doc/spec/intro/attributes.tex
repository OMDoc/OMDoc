\begin{omgroup}[id=common-attribs]{Common Attributes in OMDoc}
Generally, the \omdoc format allows any attributes from foreign (i.e. non-\omdoc)
\twinalt{namespaces}{foreign}{namespace} on the \omdoc elements. This is a commonly
found feature that makes the {\xml} encoding of the \omdoc format extensible. Note
that the attributes defined in this specification are in the default (empty)
\twinalt{namespace}{default}{namespace}: they do not carry a namespace prefix. So any
attribute of the form {\snippet{na:xxx}} is allowed as long as it is in the scope of a
suitable {\atwintoo{namespace}{prefix}{declaration}}.

Many \omdoc elements have optional {\attributeshort[ns-attr=xml]{id}} attributes that
can be used as identifiers to reference them. These attributes are of type
\twinalt{\snippet{ID}}{type}{ID}, they must be unique in the document which is important,
since many {\xml} \twinalt{applications}{XML}{application} offer functionality for
referencing and retrieving elements by \twinalt{{\snippet{ID}}-type}{type}{ID} attributes.
Note that unlike other {\snippet{ID}}-attributes, in this special case it is the name
{\attributeshort[ns-attr=xml]{id}}~\cite{XML:id05} that defines the
{\indextoo{referencing}} and {\indextoo{uniqueness}} functionality, not the type
declaration in the {\indextoo{DTD}} or {\twintoo{XML}{schema}} (see
{\extref{book}{xml.validation}} for a discussion).

  Note that in the \omdoc format proper, all {\twintoo{ID}{type}} attributes are of the
  form {\attributeshort[ns-attr=xml]{id}}. However in the older {\openmath} and {\mathml}
  standards, they still have the form {\attributeshort{id}}. The latter are only
  recognized to be of type {\snippet{ID}}, if a document type or {\xml}schema is
  present. Therefore it depends on the application context, whether a DTD should be
  supplied with the \omdoc document.

  For many occasions (e.g. for printing \omdoc documents), authors want to control a
  wide variety of aspects of the presentation. \omdoc is a content-oriented format, and
  as such only supplies an infrastructure to mark up content-relevant information in
  \omdoc elements. To address this dilemma {\xml} offers an interface to 
  {\twintoo{Cascading}{Style Sheet}s} ({\css})~\cite{BosHak:css98}, which allow to specify
  presentational traits like {\twintoo{text}{color}}, {\twintoo{font}{variant}},
  {\indextoo{positioning}}, {\indextoo{padding}}, or {\indextoo{frame}s} of
  {\twintoo{layout}{box}es}, and even {\indextoo{aural}} aspects of the text.

  To make use of {\css}, most \omdoc elements (all that have
  {\attributeshort[ns-attr=xml]{id}} attributes) have {\attributeshort{style}} attributes
  that can be used to specify {\css} \twinalt{directives}{CSS}{directive} for them. In the
  \omdoc fragment in {\mylstref{css-basic}} we have used the \attribute{style}{omtext}
  attribute to specify that the text content of the \element{omtext} element should be
  formatted in a centered box whose width is 80\% of the surrounding box (probably the
  page box), and that has a 2 pixel wide solid frame of the specified RGB color. Generally
  {\css} directives are of the form {\snippet{A:V}}, where {\snippet{A}} is the name of
  the aspect, and {\snippet{V}} is the value, several {\css}
  \twinalt{directives}{CSS}{directive} can be combined in one {\attributeshort{style}}
  attribute as a {\twintoo{semicolon-separated}{list}} (see {\cite{BosHak:css98} and the
    emerging {\css} 3} standard).

\begin{lstlisting}[label=lst:css-basic,mathescape,
   caption={Basic {\css} Directives in a {\attributeshort{style}} Attribute},
   index={style,class}]
<?xml version="1.0" encoding="utf-8"?>
<?xml-stylesheet type="text/css" href="http://example.org/style.css"?>
<omdoc xml:id="stylish">
  $\ldots$
  <omtext xml:id="t1" style="width:80%;align:center;border:2px #006699 solid">
    <h:p>Here comes something 
      <h:span style="font-weight:bold;color:green" class="emphasize">stylish</h:span>!
    </h:p>
  </omtext>
  $\ldots$
</omdoc>
\end{lstlisting}

Note that many {\css} properties of parent elements are inherited by the children,
if they are not explicitly specified in the child. We could for instance have set
the {\twintoo{font}{family}} of all the children of the \element{omtext} element
by adding a directive {\snippet{font-family:sans-serif}} there and then override it by
a directive for the property {\snippet{font-family}} in one of the children.

Frequently recurring groups of {\css} directives can be given symbolic names in {\css}
\twinalt{styles heets}{CSS}{style sheet}, which can be referenced by the
{\attributeshort{class}} attribute. In {\mylstref{css-basic}} we have made use of this
with the class {\snippet{emphasize}}, which we assume to be defined in the style sheet
{\snippet{style.css}} associated with the document in the ``{\twintoo{style
    sheet}{processing instruction}}'' in the prolog\footnote{i.e. at the very beginning of
  the {\xml} document before the document type declaration} of the {\xml} document
(see~\cite{Clark:assxd99} for details).  Note that an \omdoc element can have both
{\attributeshort{class}} and {\attributeshort{style}} attributes, in this case, precedence
is determined by the rules for {\css} style sheets as specified in~\cite{BosHak:css98}. In
our example in {\mylstref{css-basic}} the directives in the {\attributeshort{style}}
attribute take precedence over the {\css} directives in the style sheet referenced by the
{\attributeshort{class}} attribute on the \element{phrase} element. As a consequence,
the word ``stylish'' would appear in green, bold italics.
\end{omgroup}

%%% Local Variables:
%%% mode: latex
%%% TeX-master: t
%%% End:
