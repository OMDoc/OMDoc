%%%%%%%%%%%%%%%%%%%%%%%%%%%%%%%%%%%%%%%%%%%%%%%%%%%%%%%%%%%%%%%%%%%%%%%%%
% This file is part of the LaTeX sources of the OMDoc 1.6 project descriptions
% Copyright (c) 2006 Paul Libbrecht
% This work is licensed by the Creative Commons Share-Alike license
% see http://creativecommons.org/licenses/by-sa/2.5/ for details
% The source original is at https://github.com/KWARC/OMDoc/doc/projects/activemath
%%%%%%%%%%%%%%%%%%%%%%%%%%%%%%%%%%%%%%%%%%%%%%%%%%%%%%%%%%%%%%%%%%%%%%%%%

\begin{omgroup}[id=jeditoqmath,creators=libbrecht]{Authoring Tools for {\activemath}}

\ednote{project page:\url{http://www.activemath.org/projects/jEditOQMath}}

The {\omdoc} content to be delivered by {\activemath} are {\omdoc} documents with
{\openmath} formulae. Experience has shown that writing the {\xml}-source by hand is
feasible and even preferred if the author wants to follow the evolution of content's
structure.  It is similar to {\html} editing.  However, the complexity of {\xml} makes it
hard to keep an overview when writing mathematical expressions. Therefore, the
{\scsys{OQMath}} processor has been implemented: it uses {\scsys{QMath}} for formulae and
leaves the rest of the {\omdoc} written as usual {\xml}.

{\scsys{OQMath}} has been integrated in a supporting {\xml}-editor, jEdit. This editor
provides structural support at writing {\xml}-documents. Authors, even with no
{\xml}-knowledge, can easily write valid document {\scsys{jEditOQMath}}.  This package
includes, in a one-click installer, {\qmath}, {\scsys{OQMath}}, {\scsys{jEdit}}, and
Ant-scripts for publication of the content in {\activemath} knowledge bases.  These
scripts validate the references in the content.  These scripts also provide authors with
short cycles edit-in-{\scsys{jEditOQMath}}-and-test-in-{\activemath}.  More about
{\scsys{jEditOQMath}} can be seen from
{\url{http://www.activemath.org/projects/jEditOQMath}} at~\cite{AM-authoring-from-dev-on}

%Explorations of knowledge navigation and edition using the Protege%
%\footnote{See \url{http://protege.semanticweb.org/}.}
%ontology editor
%have been made but limitations of the visualization library have plagued this first visual
%editor attempts.

%Access to the other {\omdoc} content in order for them to be referenced, used, or inspected
%has turned out to be an essential requirement of authoring tools: 
{\scsys{jEditOQMath}} provides search facilities as well as contextual drops from items
presented in an {\activemath} window.  This way the testing of content in the target
environment and the authoring experience are bound tighter together,
% binds even tighter the test-running of content in the environment where it will be
%delivered to the authoring experience, 
thus making {\scsys{jEditOQMath}} closer to the WYSIWYG paradigm without being limited to
its simple visual incarnation.

%exercise authoring tool for 'authoring by demonstration'
%---- Ian or George here ----
% George: I can not write anything yet, there is no version of a tool releazed

% abundant  experience with authors
To date, more than $10'000$ {\emph{items}} of {\omdoc} content has been written using
these authoring tools in Algebra and Calculus. This experience with authors considerably
improved our understanding of what today's authors need and what different classes of
authors can cope with.

% has allowed us to
% gear features of the authoring tools towards users and has shown gaps still to be filled.

Among the greatest difficulties of authoring content for {\activemath} was the art of
properly choosing mathematical semantic encoding: the mathematical discourse is made of
very fine notation rules along with subtle digressions to these rules...  formalizing
them, as is needed when writing {\openmath} or the {\qmath} formulae for them, turns out
to often be overwhelming.  The usage of the ellipsis in such a formula as $1, \dots, k,
\dots, n$ is a simple example of semantic encoding challenge. The knowledge organization
of {\omdoc} that makes it possible to define one's own {\openmath} symbols has been a key
ingredient to facing this challenge.

Among the features most requested by authors, which we have tried to answer 
as much as possible,  are a short edit-and-test cycle and validation facilities 
taking in account the overall content.

\begin{omgroup}{Validation Tools}

Automated validation of {\omdoc} content has many facets.
% can be done in many respects but little has been done with {\omdoc} documents.
{\xml}-validation with a DTD and Schema is a first step.  However there are still many
structure rules mentioned only as human readable forms in the {\omdoc} specifications.
References between {\omdoc} items is another important facet which has been answered by
{\activemath} knowledge bases and publishing scripts.  Experience has proved that ignoring
such errors has lead repeatedly to authors complaining about the weirdest behaviours of
the overall learning environment.  Many other simple validations could be done in order to
support the author, for example the validation of a picture embedding, or of fine grained
typing of relations (for example, that a definition should only be {\emph{for}} a symbol).

Further validation tools are being investigated, for example, those tuned to particular
pedagogical scenarios.
\end{omgroup}

\begin{omgroup}{Further Authoring Tools for {\activemath}}
  {\scsys{jEditOQMath}} clearly remains for users who feel comfortable with source
  editing. Experience has shown that authors having written {\html} or {\TeX} earlier did
  not find this paradigm problematic.  It is, however, a steep learning slope for beginner
  authors.  A more visual component is being worked upon, able to display and edit
  visually the children of a \element{CMP}, including formulae.\footnote{More about the
    component for {\omdoc} micro-structure can be read from
    \url{http://www.activemath.org/projects/OmdocJdomAuthoring/}.}  This component, along
  with forms and summaries for metadata, should provide a visual environment to edit
  {\omdoc} content for {\activemath} in a relatively accessible way.

Another area where source editing has shown difficulties is in the process of authoring
exercises with many steps... the rich structure of the exercises, along with the non-neglect able
space taken by the display of {\xml}-source has challenged several authors, having
difficulties to overview such sources as 600 Kb of {\scsys{OQMath}} source for a single
exercise. 
A web-based visual authoring environment is under work within the {\activemath} group.
\end{omgroup}
\end{omgroup}
%%% Local Variables: 
%%% mode: stex
%%% TeX-master: "../main"
%%% End: 

% LocalWords:  jeditoqmath Libbrecht GmbH Universit des Saarlandes OQMath QMath
% LocalWords:  jEdit jEditOQMath jEditOQMath jEditOQMath jEditOQMath one's
% LocalWords:  jEditOQMath jEditOQMath jEditOQMath jEditOQMath jEditOQMath
% LocalWords:  jEditOQMath jEditOQMath jEditOQMath metadata jEditOQMath
% LocalWords:  jEditOQMath jEditOQMath jEditOQMath jEditOQMath jEditOQMath
