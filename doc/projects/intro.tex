%%%%%%%%%%%%%%%%%%%%%%%%%%%%%%%%%%%%%%%%%%%%%%%%%%%%%%%%%%%%%%%%%%%%%%%%%
% This file is part of the LaTeX sources of the OMDoc 1.6 project descriptions
% Copyright (c) 2006 Michael Kohlhase
% This work is licensed by the Creative Commons Share-Alike license
% see http://creativecommons.org/licenses/by-sa/2.5/ for details
% The source original is at https://github.com/KWARC/OMDoc/doc/projects
%%%%%%%%%%%%%%%%%%%%%%%%%%%%%%%%%%%%%%%%%%%%%%%%%%%%%%%%%%%%%%%%%%%%%%%%%

\chapter{Introduction}\label{chapx:projeccts-intro}
The text in the project descriptions has been contributed\footnote{If your {\omdoc}
  project is not represented here, please contact \url{m.kohlhase@jacobs-university.de} to
  arrange for inclusion in later editions of this book.} by the authors marked in the
section headings, for questions about the projects or systems, please visit the web-sites
given or contact the authors directly. Note that the material discussed in this chapter is
under continuous development, and the account here only reflects the state of mid-2006,
see \url{http://www.omdoc.org} for more and current information.


\section{Overview}\label{subsec:projects-overview}
The {\omdoc} format as a whole and the applications mentioned above are supported by a
variety of tools for creating, manipulating, and communicating {\omdoc} documents. We can
distinguish four kinds of tools:

\begin{description}
\item[{\emph{Interfaces\index{interface} for Mathematical Software Systems}}] like
  automated theorem provers. These system are usually add-ins that interpret the internal
  representation of formalized mathematical objects in their host systems and recursively
  generate formal {\omdoc} documents as output and communication streams. Some of these
  systems also have input filters for {\omdoc} like the {\verifun} described in
  {\sref{verifun}}, but most rely on the {\omdoc} transformation to their native input
  syntax described in {\extref{processing}{omdoctosys}}.
\item[\emph{Invasive Editors\twin{invasive}{editor}}] i.e. are add-ins or modes that
  ``invade'' common general-purpose editing systems and specialize them to deal with the
  {\omdoc} format.  The {\cpoint} add-in for MS PowerPoint ({\sref{cpoint}}), the
  {\sentido} plugin for {\mozilla}-based browsers, and the plugin for {\texmacs}
  ({\sref{texmacs-omega}}) are examples for this kind of editor. They differ from simple
  output filter in providing editing functionality for {\omdoc} specific information.
\item[\emph{Human-Oriented Frontend Formats\atwin{human-oriented}{frontend}{format}}] for
  instance the {\qmath} project described in {\sref{qmath}} defines an interface
  language for a fragment of {\omdoc}, that is simpler to type by hand, and less verbose
  than the {\omdoc} that can be generated by the {\snippet{qmath}} parser. {\stex} defines
  a human-oriented format for {\omdoc} by extending the {\TeX/\LaTeX} with content markup
  primitives, so that it can be transformed to {\omdoc}. See {\sref{stex}} for
  details.
\item [\emph{Mathematical Knowledge Bases\atwin{mathematical}{knowledge}{base}}] The
  {\mbase} and {\maya} systems described in {\sref{mbase}} and {\sref{maya}} are web-based
  mathematical knowledge bases that offer the infrastructure for a universal, distributed
  repository of formalized mathematics represented in the {\omdoc} format.
\end{description}


\section{Application Roles of the OMDoc Format}\label{subsec:omdoc-roles}
  The applications above support the utilization of the {\omdoc} format in several
  roles. Generally, {\omdoc} can used of as a
\begin{description}
\item[\emph{Communication Standard}\twin{communication}{standard}] between mechanized
reasoning systems.
\item[\emph{Data Format for Controlled Refinement}\twin{controlled}{refinement}] from
  informal presentation to formal specification of mathematical objects and theories.
  Basically, an informal textual presentation can first be marked up, by making its
  structure\twin{structure}{discourse} explicit (classifying text fragments as
  definitions, theorems, proofs, linking text, and their relations), and then formalizing
  the textually given mathematical knowledge in logical formulae (by adding
  {\element{FMP}} elements; see {\extref{spec}{mtxt}}).
\item[\emph{Document Preparation Language}\twin{document}{preparation language}.]  The
  {\omdoc} format makes the large-scale document- and conceptual structures explicit and
  facilitates maintenance on this level. Individual documents can be encoded as
  lightweight narrative structures, which can directly be transformed to e.g.
  {\xhtml}+{\mathml} or {\LaTeX}, which can in turn be published on the Internet.
\item[\emph{Basis for Individualized (Interactive)
    Documents}\twin{document}{interactive}\twin{individualized}{document}.] Personalized
  {\twintoo{narrative}{structure}s} can be generated from {\mbase} content making use of
  the {\twintoo{conceptual}{structure}} encoded in {\mbase} together with a user model.
  For instance, the {\mmiss}, {\MathDox}, and {\activemath} projects described in
  {\sref{MMiSS}} to {\sref{activemath}} use the {\omdoc} infrastructure in an educational
  setting. They make use of the content-orientation and the explicit structural markup of
  the mathematical knowledge to generate on the fly specialized learning materials that
  are adapted to the students prior knowledge, learning goals, and notational tastes.
\item[\emph{Interface for Proof Presentation\twin{proof}{presentation}}.] As the proof
  part of {\omdoc} allows small-grained interleaving of formal ({\element{FMP}}) and
  textual ({\element{CMP}}) presentations in multiple languages (see
  e.g.~\cite{HuangFiedler:pvip97,Fiedler:uacatp99}).
\end{description}

%%% Local Variables: 
%%% mode: latex
%%% TeX-master: "main"
%%% End: 

% LocalWords:  mtext activemath omdocmode cpoint nb frontend qmath maya mtxt
% LocalWords:  omdoctosys mathdox sentido FMP CMP omdoc mbase MMiSS verifun mkm
% LocalWords:  plugin texmacs stex
