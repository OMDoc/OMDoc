\documentclass{article}
\usepackage{a4wide,url}
\usepackage[hyper]{acronyms}
\usepackage{lstomdoc,xmlindex,myref}
\usepackage[show]{ed}
\usepackage[eso-foot,today]{svninfo}
\usepackage{hyperref} 
\svnInfo $Id: note.tex 8556 2009-11-16 08:14:39Z kohlhase $
\svnKeyword $HeadURL: https://svn.omdoc.org/repos/omdoc/trunk/doc/blue/p2s-bootstrap/note.tex $
\lstset{float=htb,columns=flexible,frame=lines,language=[1.6]OMDoc,basicstyle=\scriptsize,
        numbers=left,stepnumber=5,numberstyle=\tiny,showstringspaces=false}

\def\llquote#1{\ensuremath{\langle\kern-.25em\langle\hbox{\sl{#1}}\rangle\kern-.25em\rangle}}
\def\ptos{\ensuremath{\mathcal{P}\kern-.2em2\kern-.2em\mathcal{S}}}
\usepackage[hyperref=auto,style=alphabetic]{biblatex}
\bibliography{kwarc}

\title{Towards Bootstrapping the Pragmatic to Strict Mapping in {\omdoc}}
\author{Michael Kohlhase\\Jacobs University Bremen\\\url{http://kwarc.info/kohlhase}}

\begin{document}
\maketitle
\begin{abstract}
  We propose a representational infrastructure to make the {\omdocv2} pragmatic-to-strict
  mapping that is fixed in {\omdocv{1.6}} configurable. This will allow to legitimize many
  of the pragmatic {\omdoc} features from foundational metalogics and thus make {\omdocv2}
  more flexible and expressive.
\end{abstract}

\section{Introduction}\label{sec:intro}

The {\omdocv{2}} format currently under development at KWARC has two sub-languages:
{\emph{strict {\omdoc}}} --- a conceptually minimal set of mathematically well-understood
representational primitives~\cite{RabKoh:WSMSML09}, and {\emph{pragmatic}} {\omdoc} --- a
representational infrastructure that attempts a more pragmatic balance between theoretical
expressivity and conceptual familiarity. The meaning of pragmatic {\omdoc} is given by the
{\ptos} mapping which interprets all pragmatic language features in terms of strict ones.

The strict/pragmatic language design was pioneered by the author for the upcoming
{\mathml}3 recommendation~\cite{CarlisleEd:MathML09}, and is used in
{\omdocv{1.6}}~\cite{Kohlhase:OMDoc1.6spec}, the first step towards {\omdocv2}. This
language design has the advantage that only a small, regular sublanguage has to be given a
mathematical meaning, but a larger vocabulary that is more intuitive to practitioners of
the field can be used for actual representation. Moreover, semantic services like
validation only need to be implemented for the strict subset and can be extended to the
pragmatic language by translation. Ultimately, a representation format might even have
multiple pragmatic front-ends geared towards different audiences. These are semantically
interoperable by construction.

In {\omdocv{1.6}}, the {\ptos} mapping is fixed by the specification; severely restricting
the benefits of the strict/pragmatic language design. Moreover, the specification revealed
interactions with the respective foundational meta-logics that cannot be accounted for
with a {\ptos} mapping. Consider for instance the pragmatic element of an ``implicit
definition'' which defines a mathematical object by describing it so accurately, that
there is only one object that fits this description: the definiendum. For instance the
exponential function is defined as the (unique) solution of the differential equation
$f=f'$ with $f(0)=1$. This form of definition is extensively used in practical
mathematics, so pragmatic {\omdoc} should offer an infrastructure for it. Strict {\omdoc}
only offers ``simple definitions'' of the form $c:=d$, where $c$ is a new symbol (the
definiendum) and an expression $d$ (the definiens) which does not contain $c$. In the
{\omdocv{1.6}} $\ptos$ mapping the implicit definition for the exponential function is
mapped to the simple definition $e:=\tau f.(f'=f\wedge f(0)=1)$, where $\tau$ is a
``definite description operator'': Given an expression $A$ with free variable $x$, such
that there is a unique $x$ that makes $A$ valid, $\tau x.A$ returns that $x$, otherwise
$\tau x.A$ is undefined. Note the {\ptos} mapping is only justified, if the current meta
logic supplies a definite description operator. In most areas of mathematics, this is
unproblematic, since its existence is implicitly assumed. In more foundational contexts
such as the philosophy of mathematics this poses a problem, since it prevents the
representation of systems without a definite description operator.

To alleviate this and other problems, it seems natural to make the {\ptos} mapping
configurable in the {\omdocv2} format itself. All the more since we can extend an existing
{\omdoc{2}} mechanism for this: format transformations from the notation definition
subsystem.

\section{A Configurable {\ptos} Mapping}\label{sec:main}

In~\cite{KMR:NoLMD08} we have presented a presentation system for {\omdoc}. It has been
implemented in the {\jomdoc} system and became a part of the {\omdocv{1.6}} format. In a
nutshell, the system uses {\emph{notation definitions}} such as the one in
Listing~\ref{lst:bc-notation}. This is essentially a transformation rule that transforms
{\openmath} expressions into presentation {\mathml} ones. Crucially, notation definitions
are treated like statements in {\omdoc}; in particular they are tied to theories with
which they are inherited.

\begin{lstlisting}[label=lst:bc-notation,caption=A notation for a binomial coefficient]
<notation>
  <prototype>
    <om:OMA> 
      <om:OMS cd="arith1" name="binomial"/>
      <expr name="arg1"/>
      <expr name="arg2"/>
    </om:OMA>
  </prototype>
  <rendering format="mathml-presentation+xml">
    <m:mrow>
      <m:mfrac linethickness="0">
        <render name="arg1"/>
        <render name="arg2"/>
      </m:mfrac>
    </m:mrow>
  </rendering>
</notation>
\end{lstlisting}

For a configurable {\ptos} mapping, we propose to use the same mechanism; just for
intra-{\omdoc} transformations. In our example, we would add a {\ptos} transformation rule
to a meta-theory that supplies a description operator. Listing~\ref{lst:desctheory} shows
a fragment of such a theory based on first-order logic.

\begin{lstlisting}[label=lst:desctheory,caption=A Meta Theory that supplies implicit definitions,mathescape]
<theory name="descriptions">
  <import from="fol.omdoc?fol"/>
  <constant name="that" role="binding"/>
  <notation>
    <prototype><om:OMS cd="descriptions" name="that"/></prototype>
    <rendering><m:mo>$\tau$</m:mo></rendering>
  </notation>
  <axiom name="that-ax">$\forall y.\tau x.(x=y)=y$</axiom>
  <p2srule>
    <prototype>
      <definition type="implicit">
        <attribute name="name"><expr name="name"/></attribute>
        <attribute name="var"><expr name="var"/></attribute>
        <expr name="content"/>
      </definition>
    </prototype> 
    <rendering>
     <constant>
       <attribute name="name"><text name="name"/></attribute>
       <definition>
         <om:OMBIND>
           <om:OMS cd="descriptions" name="that"/>
           <om:OMBVAR>
             <om:OMV><attribute name="name"/><text name="var"/></attribute></om:OMV>
           </om:OMBVAR>
           <render name="content"/>
         </om:OMBIND>
       </definition>
    </rendering>
  <p2srule>
</theory>
\end{lstlisting}
The theory {\attvalveryshort{descriptions}} introduces the definite description operator
as a new binding operator $\tau$, and describes its meaning by an axiom $\forall y.\tau
x.(x=y)=y$. The new pragmatic syntax is introduced in the first child of a new element
{\element{p2srule}}, which is just a variant of the {\element{notation}} element the two
kinds of transformation rules will probably have to be handled differently, therefore the
separate elements. This rule will transform the new pragmatic syntax.
\begin{lstlisting}[mathescape]
<definition type="implicit" name="expfun" var="f">$f'=f\wedge f(0)=1$</definition>
\end{lstlisting}
into the strict syntax
\begin{lstlisting}[mathescape]
<constant name="expfun">
 <definition>$\tau f.f'=f\wedge f(0)=1$</definition>
</constant>
\end{lstlisting}
in a situation where the theory {\attvalveryshort{descriptions}} is imported as a
meta-theory. Importantly, strict {\omdoc} supports structural validation, which now
extends to the new syntax. 

Note that while the introduction of implicit definition via a simple first-order
description operator works in principle, it does not capture the structure of implicit
definitions in {\omdocv{1.2}}~\cite{Kohlhase:OMDoc1.2}, which uses attributes
{\attribute{existence}{definition}} and {\attribute{uniqueness}{definition}} as pointers
to assertions stating unique existence. In our example above, unique existence assertions
are also needed to invoke the the axiom {\attvalveryshort{that-ax}}: $\exists^1
f.f'=f\wedge f(0)=1$ can be used to show that the sets $\{f\bigl|f'=f\wedge f(0)=1\}$ and
$\{f\bigl|f=e\}$ where $e$ is the exponential function are equal and thus that $e=\tau
f. f'=f\wedge f(0)=1$. So any further mathematical development of the exponential function
will need this assertion, but the syntax defined by the {\ptos} mapping in
Listing~\ref{lst:desctheory} has no structural information to locate them. In fact in
mathematical texts, the association is only made informally in the accompanying text.

As for machine-processing of mathematical knowledge the accompanying text is opaque, we
will try to re-gain the structural constraints using a more structural meta-logic:
dependently typed first-order logic (a dependently typed meta-logic~\cite{Ban:MizarTypes}
is used in the Mizar system~\cite{mizar:online} one of the larges mathematics formalization
projects). In DFOL\ednote{cite Florian's papers here} we can introduce a definite
description operator $\iota$ of type $\Pi\alpha.\Pi A\colon(\alpha\rightarrow
o).pf(\exists^1A).\alpha$\ednote{check if that is really the right type.} Intuitively,
$\iota$ takes as arguments a type $\alpha$, a set $A$, and a proof that $A$ has a unique
member $a$ (i.e. is a singleton) and returns the unique member of that set. This would
give rise to a translation\ednote{hmmmm, I am using $\lambda$ here, I am not sure that
  this exists in DFOL} like the one in Listing~\ref{iotap2s}.
\begin{lstlisting}[label=lst:iotap2s,caption=A Meta Theory for structured implicit definitions,mathescape]
<p2srule>
  <prototype>
    <definition type="implicit">
      <attribute name="name"><expr name="name"/></attribute>
      <attribute name="var"><expr name="var"/></attribute>
      <attribute name="exunique"><expr name="exunique"/></attribute>
      <expr name="idef"/>
    </definition>
    $\ldots$
    <type>
      <attribute name="for"><expr name="name"/></attribute>
      <expr name="tcont"/>
    </type>
    $\ldots$
    <proof>
      <attribute name="name"><expr name="exunique"/></attribute>
      <expr name="pcont"/>
    </proof>
  </prototype> 
  <rendering>
   <constant>
     <attribute name="name"><text name="name"/></attribute>
     <definition>
       <om:OMA>
        <om:OMS cd="descriptions" name="iota"/>
        <render name="tcont"/>
        <om:OMBIND>
          <om:OMS cd="dfol" name="lambda"/>
          <om:OMBVAR>
            <om:OMV><attribute name="name"/><text name="var"/></attribute></om:OMV>
          </om:OMBVAR>
          <render name="idef"/>
        </om:OMBIND>
        <render name="pcont"/>
      </om:OMA>
    </definition>
  </rendering>
<p2srule>
\end{lstlisting}
This already shows added difficulties for the implementation of the $\ptos$ mapping: The
prototype contains three elements that together correspond to the strict
definition. Naturally, the pragmatic syntax does not want to restrict itself to having
these three elements in a consecutive block (a fact that we have tried to gloss by using
the ellipses $\ldots$ in the {\element{prototype}} element).

\section{Research \& Design Problems}\label{sec:problems}
There are some problems that still have to be solved:

\subsection{Multi-Part Prototypes} 
Like in Listing~\ref{lst:iotap2s}. How are they matched, over what scope?

\subsection{Prioritization \& Scoping \& Generic Rules}
How are the {\ptos} rules prioritized and scoped? This is especially important, if we want
to introduce generic {\ptos} rules, e.g. for attributes. Consider for instance the
pragmatic way of specifying rhetorical metadata in {\omdocv{1.2}} via the
{\attributeshoft{type}} attribute and related ones. For instance in {\omdocv{1.2}} we
would have
\begin{lstlisting}[mathescape]
<omtext type="transition" from="foo" to="bar">$\ldots$</omtext>
\end{lstlisting}
which would be represented as 
\begin{lstlisting}[mathescape]
<omtext>
  <meta property="omdoc:transition"/>
  <link rel="omdoc:transition-from" href="foo"/>
  <link rel="omdoc:transition-to" href="bar"/>
  $\ldots$
</omtext>
\end{lstlisting}
Note that these attributes might give rise to {\ptos} transformation rules in a generic
{\attvalveryshort{omdoc}} theory via 
\begin{lstlisting}[mathescape]
<p2srule>
  <prototype>
    <element name="name">
      <attribute name="type"><text name="type"/></attribute>
      <attributes name="arest"/>
      <exprlist name="crest"/>
    </element>
  </prototype>
  <rendering>
    <element name="name"><attributes name="arest/>
    <meta><attribute name="property"><text>omdoc:</text><text name="type"/></attribute></meta>
    <iterate name="crest"/>
  </rendering>
</p2srule>  
\end{lstlisting}
and similar ones for the {\attributeshort{for}} and {\attributeshort{from}}
attributes. Note that in contrast to the {\ptos} rules shown above, these rules are not
bound to a specific new element, but introduce new attributes\ednote{is that syntax above
  a good way of talking about elements {\texttt{<*>}}}. So we might also contemplate the
following transformation rule syntax:
\begin{lstlisting}[mathescape]
<p2srule>
  <prototype>
    <attribute name="type"><text name="type"/></attribute>
  </prototype>
  <rendering>
    <meta><attribute name="property"><text>omdoc:</text><text name="type"/></attribute></meta>
  </rendering>
</p2srule>  
\end{lstlisting}


\subsection{User Extensions}\label{sec:problems:ue}
Do we allow the user (as opposed to the language designer) to extend the syntax? If we
did, we might end up with a situation like in {\LaTeX}, where we have thousands of
{\omdoc} packages and classes.

\subsection{Documentation}
How are the {\ptos} rules documented for the language user? There should be some form an
explanation system for the currently active transformation, especially, when we have user
extensions as envisioned in section~\ref{sec:problems:ue}

\section{Conclusion}\label{sec:concl}

I have proposed a novel, more semantic way of handling the {\omdocv2} pragmatic to strict
mapping. Even though some practical problems still have to be solved (see
Section~\ref{sec:problems}), the approach points to a much more modular and flexible way
of developing the {\omdoc} language. From all the examples that the author is aware, it
seems to scale to interpreting the whole {\omdocv{1.2}} vocabulary as pragmatic
{\omdocv{2}}. In fact, the new idea of having pragmatic extensions be meta-theory induced
points towards the idea that {\omdoc} metadata and the corresponding metadata
ontologies~\cite{LK:MathOntoAuthDoc09} are actually meta-theories as well (albeit at a
somewhat different level).

\begin{newpart}{this is still quite sketchy; must discuss with Florian; also in terms of
    what this explains about foundations. Indeed the foundations supply type and equality
    judgments; just the relations that are extensions. What would a real foundational MMT
    look like?} In fact we can go even further and interpret the features of types and
  definitions that are built into strict {\omdocv{1.6}}~\cite{RabKoh:WSMSML09} as
  pragmatic extensions of an even more foundational system: definitions are just pragmatic
  notations for axioms of the form $c=t$\ednote{Here Fulya's declaration patterns come in:
    $t$ may not contain $c$.}, which are introduced by the meta-theory that supplies the
  equality relation. Axioms in turn are just type statements of the form $true(A)$, so
  they can be viewed as pragmatic extensions licensed by the meta-theory that introduces
  the truth type $true$. 
\end{newpart}
\printbibliography
\end{document}


% LocalWords:  OMA cd arith expr arg mathml xml mrow mfrac linethickness omdoc
% LocalWords:  srule OMBIND OMBVAR OMV expfun DFOL hmmmm exunique idef tcont
% LocalWords:  pcont dfol omtext href arest exprlist
