\documentclass{article}
\usepackage{a4wide,url}
\usepackage{lstomdoc,xmlindex}
\usepackage[hyper]{acronyms}
\usepackage[show]{ed}
\usepackage{hyperref}
\lstset{language=[1.6]OMDoc,basicstyle=\scriptsize,mathescape}
\def\stex{{\raisebox{-.5ex}S\kern-.5ex\TeX}}
\def\llquote#1{\ensuremath{\langle\kern-.25em\langle{#1}\rangle\kern-.25em\rangle}}
\title{Proposal for a set of pragmatic notation definitions}
\author{Michael Kohlhase}

\begin{document}
\maketitle
\begin{abstract}
  In this note, we propose abbreviations of notation definitions that are simpler to write
  and more declarative as well.
\end{abstract}
\tableofcontents\newpage
\section{Introduction}

The notation definitions presented in~\cite{KMR:NoLMD08} allow to represent a
comprehensive set of notations by pattern matching and has been implemented in the JOMDoc
system~\cite{JOMDoc:webpage}. One of the advantages of this system is that gives common
tasks like bracket elision a straightforward transformation interpretation. However, in
many cases, the notation definitions become unwieldy and clumsy for simple tasks like
specifying that addition is associative with precedence 500; for such cases, we would like
introduce syntactic shortcuts that are simpler to write and maintain. Here the
strict/pragmatic distincition introduced in {\omdocv{1.6}} comes to the rescue, we
consider the notation definitions of~\cite{KMR:NoLMD08} as {\emph{strict OMDoc}} and the
ones we propose here as {\emph{pragmatic OMDoc}}. The idea is that the pragmatic notation
definitions are translated into strict ones via the rules given below during the
presentation process, so that the translation is the only extension necessary to the
presentation process.

\section{Declarative Notation Declarations}

The concrete proposal is based on extensive practice in the {\stex} presentation
package~\cite{Kohlhase:ipsmsl08} that has been used to mark up more than 1700 symbols in
the context of slides and notes for lectures and talks given by the KWARC group.
Surprisingly the proposal in this note comes out relatively close to the notation proposal
of OMDoc1.2~\cite{Kohlhase:OMDoc1.2} and the notation proposal we made earlier
in~\cite{KohLanRab:pmcfe07}.

In a nutshell we use mixfix declarations for symbols referenced by the OpenMath triple
({\attribute{cd}{mixfix}}, {\attribute{name}{mixfix}}, and
{\attribute{cdbase}{mixfix}}). The attribute {\attribute{args}{mixfix}} specifies the
number of arguments, and the attribute {\attribute{assoc}{mixfix}} allows us to specify
one argument as associative.

 the attributes
@p @pi @pii @piii, ...  give outer and argument precedences. Finally, the role is the OM
symbol role. The pragmatic elements are called mixfix* where * is a specification of the
arities of the argumunds, i.e. a string of characters 'i' (arity 1 argument) and one
character 'a' an associative sequence of arguments. Here is the strict-pragmatic
translation for <mixfixai>, which is e.g. used for funtype (I will make the concrete
example later).

\begin{center}\lstset{frame=none,numbers=none,lineskip=-.7ex,aboveskip=-.5em,belowskip=-1em}
  \begin{tabular}{|p{4cm}|p{8.4cm}|}\hline
    pragmatic & strict\\\hline
{
\begin{lstlisting}
<mixfix format="$\llquote{foo}$" 
  args="$\llquote{n}$" assoc="$\llquote{m}$"
  name="$\llquote{nam}$" cd="$\llquote{cd}$" 
  cdbase="$\llquote{URI}$"
  role="application" 
  precedence="$\llquote{p}$">
  $\llquote{rendering}$
</mixfix>
\end{lstlisting}
}&{
\begin{lstlisting}
<notation>
 <prototype>
  <OMA>
   <OMS cd="$\llquote{cd}$" name="$\llquote{nam}$" cdbase="$\llquote{URI}$"/>
   <expr name="arg1"/>$\ldots$<expr name="arg$\llquote{m-1}$"/>
   <exprlist name="arg$\llquote{m}$">
    <expr name="aargs"/>
   </exprlist>
   <expr name="arg$\llquote{m+1}$"/>$\ldots$<expr name="arg$\llquote{n}$"/>
  </OMA>
 </prototype>
 <rendering format="$\llquote{foo}$" precedence="$\llquote{p}$">
  $\llquote{rendering}$
  </rendering>
</notation>      
\end{lstlisting}
}\\\hline 
\end{tabular}
\end{center}
\ednote{@Florian, you had something about implicit arguments. I am not sure how this needs
  to be integrated here; probably for the stuff below only.}

The next level of pragmatism would be another abbreviation, e.g. {\element{infix}} for
{\element{mixfix}}, in general

\begin{center}\lstset{frame=none,numbers=none,lineskip=-.7ex,aboveskip=-.5em,belowskip=-1em}
  \begin{tabular}{|p{4cm}|p{6.5cm}|}\hline
    pragmatic & strict\\\hline
{
\begin{lstlisting}
<infix format="$\llquote{foo}$" 
  name="$\llquote{nam}$" cd="$\llquote{cd}$" 
  cdbase="$\llquote{URI}$"
  precedence="$\llquote{n}$"
  [elide="$\llquote{e}$"]>
  $\llquote{op}$
</infix>
\end{lstlisting}
}&{
\begin{lstlisting}
<mixfix format="$\llquote{foo}$" args="2"
  name="$\llquote{nam}$" cd="$\llquote{cd}$" cdbase="$\llquote{URI}$"
  precedence="$\llquote{n}$" role="application">
  $\llquote{foo-group-open}$
  <render name="arg1" precedence="$\llquote{l}$/>
  $\llquote{op}$
  <render name="arg2" precedence="$\llquote{r}$/>
  $\llquote{foo-group-close}$
</mixfix>
\end{lstlisting}
}\\\hline 
\end{tabular}
\end{center}
Where the {\attributeshort{elide}} attribute is optional and can take the values
{\attvalveryshort{left}} and {\attvalveryshort{right}}. If no {\attributeshort{elide}} is
given, then $\llquote{n}=\llquote{l}=\llquote{r}$, if {\attributeshort{elide}} has the
value {\attvalveryshort{left}}, then $\llquote{l}=\llquote{n}-1$ and
$\llquote{r}=\llquote{n}$ and if if {\attributeshort{elide}} has the value
{\attvalveryshort{right}}, then $\llquote{r}=\llquote{n}-1$ and
$\llquote{l}=\llquote{n}$.\ednote{or the other way around, I am a bit confused at the
  moment.}

This would allow to specifiy the presentation for a binary plus operator that elides
brackets to the left (sometimes called an 'infixl' operator) as 

\begin{lstlisting}
<infix format="TeX" cd="arith1" name="plus" precedence="500" elide="left">+</infix>
\end{lstlisting}

which is really much more pragmatic than the strict verion. Here we made use of the idea
that the argument precedences are inherited from the {\attributeshort{precedence}}
attribute. For prefix operators we have another abbreviation:

\begin{center}\lstset{frame=none,numbers=none,lineskip=-.7ex,aboveskip=-.5em,belowskip=-1em}
  \begin{tabular}{|p{4cm}|p{6.7cm}|}\hline
    pragmatic & strict\\\hline
{
\begin{lstlisting}
<prefix format="$\llquote{foo}$" 
  name="$\llquote{nam}$" cd="$\llquote{cd}$" 
  cdbase="$\llquote{URI}$"
  precedence="$\llquote{n}$" 
  precargs="$\llquote{m}$">
  $\llquote{op}$
</infix>
\end{lstlisting}
}&{
\begin{lstlisting}
<mixfix format="$\llquote{foo}$" args="1" assoc="1"
  name="$\llquote{nam}$" cd="$\llquote{cd}$" cdbase="$\llquote{URI}$"
  precedence="$\llquote{n}$" role="application">
  $\llquote{foo-group-open}$
  $\llquote{op}$
  <iterate name="arg1">
    <render name="aargs"/>
  </iterate>
  $\llquote{foo-group-close}$
</mixfix>
\end{lstlisting}
}\\\hline 
\end{tabular}
\end{center}
and analogously for postfix operators. 

\newpage
\section{Examples}\label{sec:examples}
The example below is mainly used to show some pragmatic notation elememnts and test the
schema. 

 \lstinputlisting{examples/mixfix.omdoc}

\newpage
\section{An extension for the {\omdoc} RelaxNG Schema}\label{sec:rnc}
\lstinputlisting[language=RNC,nolol]{pragpres.rnc}

% \newpage
% \begin{appendix}
%   \section{ The {\omdoc} RelaxNG Schema for Notation Definitions}\label{sec:presrnc}
% We reprint the RelaxNG schema for notation definitions for reference
% \lstinputlisting[language=RNC,nolol]{../../../rnc/omdocpres.rnc}

% \end{appendix}
\bibliographystyle{alphahurl}
\bibliography{kwarc}
\end{document}
