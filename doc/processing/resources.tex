%%%%%%%%%%%%%%%%%%%%%%%%%%%%%%%%%%%%%%%%%%%%%%%%%%%%%%%%%%%%%%%%%%%%%%%%%
% This file is part of the LaTeX sources of the OMDoc 1.6 specifiation
% Copyright (c) 2006 Michael Kohlhase
% This work is licensed by the Creative Commons Share-Alike license
% see http://creativecommons.org/licenses/by-sa/2.5/ for details
% The source original is at https://github.com/KWARC/OMDoc/doc/processing
%%%%%%%%%%%%%%%%%%%%%%%%%%%%%%%%%%%%%%%%%%%%%%%%%%%%%%%%%%%%%%%%%%%%%%%%%

\chapter{OMDoc resources}\label{chap:resources}
In this chapter we will describe various public resources for working with the
{\omdoc} format.

\section{The OMDoc Web Site, Wiki, and Mailing List}\label{sec:website}  The main web site for the {\omdoc} format is \url{http://www.omdoc.org}.  It
  hosts news about developments, applications, collaborators, and events, provides access
  to an list of ``\atwintoo{frequently}{asked}{questions}'' ({\indextoo{FAQ}}), and
  current and old {\omdoc} specifications and provides pre-generated examples from the
  {\omdoc} distribution.
  
  There are two mailing lists for discussion of the {\omdoc} format: 
  \begin{description}
  \item[\url{omdoc@omdoc.org}] is for announcements and discussions of the {\omdoc}
    format on the user level. Direct your questions to this list.
  \item[\url{omdoc-dev@omdoc.org}] is for developer discussions.
  \end{description}
  For subscription and archiving details see the {\omdoc} resources page for mailing
  lists~\cite{OMDoc-mailinglists:URL}.
  
  Finally, the {\omdoc} web site hosts a Wiki~\cite{OMDoc:wiki} for user-driven
  documentation and discussion.


\section{The OMDoc distribution}\label{sec:distribution}  
All resources on the {\omdoc} web site are available from the {\mathweb} {\subversion}
repository~\cite{OMDoc:svn} for anonymous download.  {\subversion} ({\svn}) is a
collaborative version control system -- to support a distributed community of developers
in accessing and developing the {\omdoc} format, software, and documentation,
see~\cite{omdoc.org:svn} for a general introduction to the setup. The resources for
version 1.6 of the {\omdoc} format which is described in this book are accessible on the
web at \url{https://svn.omdoc.org/repos/omdoc/branches/omdoc-1.6}\ednote{For the moment,
  this URI is not valid yet, use {\texttt{https://svn.omdoc.org/repos/omdoc/trunk}}} via a
regular web browser\footnote{Ongoing development of the {\omdoc} format can be accessed
  via the {\twintoo{head}{revision}} of the repository at
  \url{https://svn.omdoc.org/repos/omdoc/trunk}.}.  The {\svn} server allows anonymous
read access to the general public. To check out the {\omdoc} distribution, use\ednote{For
  the moment, this URI is not valid yet, use
  {\texttt{https://svn.omdoc.org/repos/omdoc/trunk}}}
\begin{verbatim}
svn co https://svn.omdoc.org/repos/omdoc/branches/omdoc-1.6
\end{verbatim}
This will create a directory {\snippet{omdoc-1.6}}, with the sub-directories 
\begin{center}\footnotesize
\begin{tabular}{|p{2.6cm}|p{7.5cm}|}\hline
  directories            & content \\\hline\hline
  {\snippet{bin}}, {\snippet{lib}}, {\snippet{oz}}, {\snippet{thirdParty}} &  
  programs and third-party software used in the administration and examples\\\hline
  {\snippet{css}}, {\snippet{xsl}} & style sheets for displaying {\omdoc} documents on
                           the web, see {\mychapref{transform-xsl}} for a
  discussion.\\\hline
  {\snippet{doc}}             & The {\omdoc} documentation, including the
                           specification, papers about a
                           the {\omdoc} format and tools.\\\hline
  {\snippet{dtd}}, {\snippet{rnc}} & The {\omdoc} document type definition and the {\relaxng}
                            schemata for {\omdoc}\\\hline
  {\snippet{examples}}        &  Various example documents in {\omdoc} format. \\\hline
  {\snippet{projects}}        &  various contributed developments for
                                 {\omdoc}. Documentation is usually in their {\snippet{doc}} 
                                 sub-directory\\\hline
\end{tabular}
\end{center}

After the initial check out, the {\omdoc} distribution can be kept up to date by
the command {\snippet{svn -q update}} in the top-level directory from time to time. To obtain write
access contact \url{svnadmin@omdoc.org}.


\section{The OMDoc bug tracker}\label{sec:bugzilla}
\begin{oldpart}{we have a TRAC now, describe that instead}
  \url{MathWeb.org} supplies a BugZilla bug-tracker~\cite{bugzilla:URL} at
  \url{http://bugzilla.mathweb.org:8000} to aid the development of the {\omdoc}
  format. BugZilla is a server-based discussion forum and bug tracking system. We use it
  to track, archive and discuss tasks, software bugs, and enhancements in our
  project. Discussions are centered about threads called "bugs" (which need not be
  software bugs at all), which are numbered, can be searched, and can be referred to by
  their URL. To use {\bugzilla}, just open an account and visit the {\omdoc} content by
  querying for the ``product'' {\omdoc}. For offline use of the bug-tracker we recommend
  the excellent {\scsys{Deskzilla}} application~\cite{deskzilla:URL}, which is free for
  open-source projects like {\omdoc}.

  Further development of the {\omdoc} format will be public and driven by the discussions
  on {\bugzilla}, the {\omdoc} mailing list, and the {\omdoc} Wiki
  (see~\mysecref{website}).
\end{oldpart}

\section{An XML catalog for OMDoc}\label{sec:catalog}  
  Many {\xml} processes use {\twintoo{system}{ID}s} (in practice {\indextoo{URL}s}) to
  locate supporting files like {\indextoo{DTD}s}, {\indextoo{schema}ta},
  {\indextoo{style sheet}s}. To make them more portable, {\omdoc} documents will often
  reference the files on the \url{omdoc.org} web server, even in situations, where
  they are accessible locally e.g. from the {\omdoc} distribution. This practice not only
  puts considerable load on this server, but also slows down or even blocks document
  processing, since the {\xml} processors have to retrieve these files over the Internet.

  Many processors can nowadays use \twinalt{{\xml} catalogs}{XML}{catalog} to remap
  {\twintoo{public}{identifier}s} and {\indextoo{URL}s} as an alternative to explicit
  {\twintoo{system}{identifier}s}. A catalog can convert public {\snippet{ID}}s like the
  one for the {\omdoc} DTD ({\snippet{-//OMDoc//DTD OMDoc V1.6//EN}}) into absolute URLs
  like \url{http://www.omdoc.org/dtd/omdoc.dtd}. Moreover, it can replace remote URLs like
  this one with local URLs like \url{file:///home/kohlhase/omdoc/dtd/omdoc.dtd}.  This
  offers fast, reliable access to the DTDs and schemata without making the documents less
  portable across systems and networks.

To facilitate the use of catalogs, the {\omdoc} distribution provides a catalog file
\url{lib/omdoc.cat}. This catalog file can either be imported into the system's
catalog\footnote{This catalog is usually at \url{file:///etc/xml/catalog} on {\unix}
  systems; unfortunately there is no default location for {\windows} machines.} using a
{\snippet{nextCatalog}} element of the form
\begin{lstlisting}[language=XML]
<nextCatalog xml:id="omdoc.cat" catalog="file:///home/kohlhase/omdoc/lib/omdoc.cat"/>
\end{lstlisting}
or by making it known directly to the \twinalt{{\xml} processor}{XML}{processor} by an
application-specific method. For instance for {\snippetin{libxml2}} based tools like
{\snippetin{xsltproc}} or {\snippetin{xmllint}}, it is sufficient to include the path to
{\snippetin{omdoc.cat}} in the value of the {\snippetin{XML\_CATALOG\_FILES}} environment
variable (it contains a whitespace-separated list of FILES).


\section{External Resources}\label{sec:external-resources}
The {\omdoc} format has been used on a variety of projects. {\Mychapref{projects}}
gives an overview over some of the projects (use the project home pages given
there for details), a up to date list of links to {\omdoc} projects can be found
at \url{http://omdoc.org/projects.html}. These projects have
contributed tools, code, and documentation to the {\omdoc} format, often stressing
their special vantage points and applications of the format.


%%% Local Variables: 
%%% mode: latex
%%% TeX-master: "main"
%%% End: 

% LocalWords:  lib thirdParty oz xsl doc dtd nextCatalog rnc svn bugzilla omdoc
% LocalWords:  libxml xsltproc xmllint website ta Wiki mailinglists Deskzilla
% LocalWords:  css MathWeb BugZilla BugZilla mathweb mathweb mathweb mathweb
% LocalWords:  mathweb mathweb
